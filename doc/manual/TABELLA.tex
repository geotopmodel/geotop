%*******************Dichiarazione della classe di documento e pacchetti usati***********
\documentclass[11pt, a4paper, landscape]{article}  %uso carta a3 e dimnesione font: 11 punti,  tipo di documento
\usepackage[latin1]{inputenc}
%\usepackage[utf8x]{inputenc}                    %
%\usepackage[italian]{babel}
\usepackage{fancyhdr}
\usepackage{longtable}
\usepackage{caption}
\usepackage[usenames]{color}
\usepackage{times}
\usepackage{setspace}
\usepackage{amsmath}
\usepackage{textcomp}
\usepackage{float}
\usepackage{fancyhdr}
\usepackage{amsfonts}
\usepackage{amssymb}
\usepackage{float}
\usepackage{natbib}
\usepackage{rotating}
\usepackage{graphicx}
%
\usepackage[babel]{csquotes}                    %
\usepackage[T1]{fontenc}                        %
\usepackage{indentfirst}                        %
\usepackage{lmodern}                            %
%\usepackage[hang,  footnotesize,  it]{caption}    % 
%\usepackage{tikz}                               %
%\usepackage{vmargin}                            %
%\usepackage{booktabs}                           %
%\usepackage{cite}                         
%%%%%%%%%%%%%%%%%%%%%%%%%%%%%%%%%%%%%%%%%%%%%%%%%%%

\newif\ifpdf
\ifx\pdfoutput\undefined
\pdffalse % we are not running PDFLaTeX
\else
\pdfoutput=1 % we are running PDFLaTeX
\pdftrue
\fi

%\ifpdf
%\usepackage[pdftex]{graphicx}
%\else
%\usepackage{graphicx}
%\usepackage{graphicx}
%\fi

%*****************Tutta una serie di comandi di formattazione**************************
\textwidth=25cm %larghezza corpo testo (l'intero paragrafo,  non le parole singole)
\textheight=15cm %altezza del corpo del testo (spazio occupato)
\headsep=1cm %distanza del testo dall'intestazione in alto
\topmargin=1.5cm %distanza dell'intestazione dal limite superiore (che non e' il margine)*
\footskip=1cm %distanza tra il testo e il piede di pagina inferiore
\voffset=-2cm %zona tra il limite superiore e il margine del foglio (vedi manuale)
\evensidemargin=-0.50cm
\oddsidemargin=0cm


%\providecommand{\geotop}{\lower-.25em\hbox{G}E\lower.25em\hbox{O}\lower-.25em\hbox{T}O\lower-.25em\hbox{P}\@}
\providecommand{\geotop}{\lower-.25em\hbox{GEO}\lower-.25em\hbox{top}\@}

%*****************Inizio descrizione del testo: indice,  capitoli,  elenchi**************

\begin{document}
%%% TITOLO%%%%%%%%%%%%%%%%%%%%%%%%%%%%%%%

\begin{titlepage}
\thispagestyle{empty}
\topmargin=0cm

%\begin{figure}[htbp]
%\begin{center}
%\includegraphics[width=5cm]{./pic/logoPATgeologico.png}\\
%\vspace{0.2cm}
%\end{center}
%\end{figure}

%\begin{figure}
%\begin{minipage}[htp]{0.5\linewidth} % A minipage that covers half the page
%\centering
%\includegraphics[height=2.5cm]{./pic/logopermanet.pdf}
%\end{minipage}
%\hspace{0.1 cm} % To get a little bit of space between the figures
%\begin{minipage}[!tbp]{0.5\linewidth}
%\centering
%\includegraphics[height=2.5cm]{./pic/logoAlpineSpace.png}
%\vspace{0.5cm}
%\end{minipage}
%{\it The PermaNET project is part of the European Territorial Cooperation and co-funded by the European Regional Development Fund (ERDF) in the scope of the Alpine Space Programme www.alpine-space.eu}
%%\label{U_psi}
%\end{figure}


%\vspace{0.2cm}
%\begin{center}
%{\LARGE Report on the activity on permafrost modeling in the Autonomous Province of Trento}\\
%\end{center}
%\vspace{0.2 cm}
%\begin{center}
%{ \large Realized by} \\
%\vspace{0.5cm}
%{\LARGE \bf{Mountain-eering S.r.l.} }\\
%\vspace{1cm}
% ing. Filippo Zambon\\
% \vspace{0.1cm}
% Dr. Matteo Dall'Amico,  PhD\\
%\begin{figure}[t, b]
%\includegraphics[width=0.2\textwidth]{./firma_matteo.png}
%\end{figure}
%\end{center}


%\noindent{\large Report on the determination of the thermal parameters of the Cime Bianche borehole through physically based hydroL modeling}

\vspace{0.2cm}
\hfill{May 2011}\\


%\newpage
\mbox{}
\clearpage

%\thispagestyle{empty}
%\vspace{17cm}
%%\begin{figure}[H]
%%\includegraphics[width=8cm]{./pic/logo_R.pdf}\\
%%\end{figure}
%\noindent {\bf Mountain-eering S.r.l.} Societ\`a di Ingegneria \\
%Spin Off Universit\`a degli Studi di Trento \\
%Start Up del Technology Innovation Suedtirol (TIS) di Bolzano \\
%Sede legale: Siemens str. 19 via Siemens,  Bozen 39100 Bolzano - Italy \\
%Tel: 0471 068226 Fax: 0471 068229 \\
%Ufficio Tecnico: Via Giusti 10,  38100 Trento - Italy \\


\mbox{ } \thispagestyle{empty}

\end{titlepage}


%*********************Fine modifica titolo******************************************

\baselineskip=5.5mm %formattazione distanza tra le linee del testo

%\newpage
%\mbox{}
%\clearpage

%\pagenumbering{roman} \setcounter{page}{1}
%\tableofcontents %questo genera l'indice
%\clearpage


%**************************NUOVA FORMATTAZIONE TITOLI E PARAGRAFI E SPAZIATURA TRA PARAGRAFI*****
%\makeatletter %setto la @ come font comando

%\renewcommand{\chapter}{\@startsection  %formattazione titolo capitolo
 %      {chapter}                            %nome
%       {0}                                  %livello
%       {0mm}                                %rientro
%       {10cm}                               %spazio prima
%       {1cm}                              %spazio dopo
%       {\normalfont\huge\bfseries\upshape}} %formattazione carattere

%\renewcommand{\section}{\@startsection  %formattazione titolo sezione
%       {section}                            %nome
%       {1}                                  %livello
%       {0mm}                                %rientro
%       {2cm}                                %spazio prima
%       {1cm}                                %spazio dopo
%       {\normalfont\LARGE\bfseries\upshape}}%formattazione carattere

%\renewcommand{\subsection}{\@startsection  %formattazione titolo subsezione
%       {subsection}                           %nome
%       {2}                                         %livello
%       {0mm}                                  %rientro
%      {1.5cm}                                %spazio prima
%       {0.5cm}                                %spazio dopo
 %      {\normalfont\Large\bfseries\upshape}}  %formattazione carattere

%\renewcommand{\subsubsection}{\@startsection  %formattazione titolo subsubsezione
%       {subsubsection}                        %nome
%       {3}                                    %livello
%       {0mm}                                  %rientro
%       {1.5cm}                                %spazio prima
%       {0.5cm}                                %spazio dopo
%       {\normalfont\large\bfseries\upshape}}  %formattazione carattere

%\renewcommand{\paragraph}{\@startsection      %formattazione titolo paragraph
%       {paragraph}                            %nome
%       {4}                                    %livello
%       {0mm}                                  %rientro
%       {1cm}                                  %spazio prima
%       {0.7cm}                                %spazio dopo
%       {\normalfont\normalsize\itshape}}      %formattazione carattere

%\makeatother %riporto la @ a font normale
%**************************FINE FORMATTAZIONE TITOLI E PARAGRAFI E SPAZIATURA TRA PARAGRAFI*****

\parindent=0.5cm %formattaz. rientro dell'inizio paragrafo
%\newpage
%\mbox{ }  \thispagestyle{empty}
\pagenumbering{arabic} \setcounter{page}{0} %numerazioni pagine (testo)

%\clearpage
%\newpage
\pagestyle{fancy} %stile di formattazione di pagina (per l'intestazione eil pi� del documento)
\fancyhead[LE, RO]{\footnotesize \slshape {}} %formattazione intestazione (vedi man.)
\fancyhead[LO, RE]{\footnotesize \slshape \leftmark} %come sopra (vedi manuale)
%\renewcommand{\chaptermark}[1]{\markboth{\thechapter.\ #1}{}} %come sopra
\renewcommand{\sectionmark}[1]{\markright{\thesection.\ #1}}
\renewcommand{\captionfont}{\footnotesize} %formattazione didascalie foto e tabelle
\renewcommand{\headrulewidth}{0.2pt}

%\part{Relazione}
\tableofcontents %questo genera l'indice










\clearpage
\begin{center}
\begin{longtable}{|p {6.5 cm}|p {4.5 cm}|p {3 cm}|p{3 cm}|p{1.5 cm}|p{1.5 cm}|p{2 cm}|}
\hline
\textbf{Keyword} & \textbf{Description} & \textbf{M. U.} & \textbf{range} & \textbf{Default Value} & \textbf{Scalar / Vector} & \textbf{Logical / Numeric} \\ \hline
\endfirsthead
\hline
\multicolumn{7}{| c |}{continued from previous page} \\
\hline
\textbf{Keyword} & \textbf{Description} & \textbf{M. U.} & \textbf{range} & \textbf{Default Value} & \textbf{Scalar / Vector} & \textbf{Logical / Numeric} \\ \hline
\endhead
\hline
\multicolumn{7}{| c |}{{continued on next page}}\\ 
\hline
\endfoot
\endlastfoot
\hline
InitSWE & Initial snow water equivalent (SWE) - used if no snow map is given & kg m$^{-2}$ &  & 0 & sca & num \\ \hline
InitSnowDensity & INITIAL SNOW DENSITY - uniform with depth & kg m$^{-3}$ &  & 200 & sca & num \\ \hline
InitSnowTemp & INITIAL SNOW TEMPERATURE - uniform with depth & $^\circ$C &  & -3 & sca & num \\ \hline
InitSnowAge & INITIAL SNOW AGE & days &  & 0 & sca & num \\ \hline
InitGlacierDepth & GLACIER DEPTH - used if no snow map is given & mm &  & 0 & sca & num \\ \hline
InitGlacierDensity & INITIAL GLACIER DENSITY - uniform with depth & kg m$^{-3}$ &  & 800 & sca & num \\ \hline
InitGlacierTemp & INITIAL GLACIER TEMPERATURE - uniform with depth & $^\circ$C &  & -3 & sca & num \\ \hline
InitWaterTableHeightOverTopoSurface & initial condition on water table depth (positive downwards from ground surface). Used if InitSoilPressure is void & mm &  & 0 & sca & num \\ \hline
InitSoilPressure &  & mm &  & NA & vec & num \\ \hline
InitSoilTemp &  & $^\circ$C &  & 5 & vec & num \\ \hline
InitSoilPressureBedrock &  & mm &  & NA & vec & num \\ \hline
InitSoilTempBedrock &  & $^\circ$C &  & 5 & vec & num \\ \hline
\caption{Table of initial condition  (numeric)}
\label{IC1d_numeric}
\end{longtable}
\end{center}


\clearpage
\begin{center}
\begin{longtable}{|p {6.5 cm}|p {4.5 cm}|p {3 cm}|p{3 cm}|p{1.5 cm}|p{1.5 cm}|p{2 cm}|}
\hline
\textbf{Keyword} & \textbf{Description} & \textbf{M. U.} & \textbf{range} & \textbf{Default Value} & \textbf{Scalar / Vector} & \textbf{Logical / Numeric} \\ \hline
\endfirsthead
\hline
\multicolumn{7}{| c |}{continued from previous page} \\
\hline
\textbf{Keyword} & \textbf{Description} & \textbf{M. U.} & \textbf{range} & \textbf{Default Value} & \textbf{Scalar / Vector} & \textbf{Logical / Numeric} \\ \hline
\endhead
\hline
\multicolumn{7}{| c |}{{continued on next page}}\\ 
\hline
\endfoot
\endlastfoot
\hline
Iobsint & Let Micromet determine an appropriate "radius of influence" (=0), or define the "radius of influence" you want the model to use (=1). 1=use obs interval below, 0=use model generated interval. & - &  & 1 & sca & num \\ \hline
Dn & The "radius of influence" or "observation interval" you want the model to use for the interpolation.  In units of deltax, deltay. & - &  & 1 & sca & num \\ \hline
SlopeWeight & Weight assigned to the slope (as tangent when it is < 1) in the spatial distribution of the wind speed & - &  & 0 & sca & num \\ \hline
CurvatureWeight & Weight assigned to the curvature (as second derivative of the topographic surface) in the spatial distribution of the wind speed & - &  & 0 & sca & num \\ \hline
SlopeWeightD &  &  &  & 0 & sca & num \\ \hline
CurvatureWeightD &  &  &  & 0 & sca & num \\ \hline
SlopeWeightI &  &  &  & 0 & sca & num \\ \hline
CurvatureWeightI &  &  &  & 0 & sca & num \\ \hline
\caption{Table of spatial distribution method parameters  (numeric)}
\label{meteodistr1d_numeric}
\end{longtable}
\end{center}




\clearpage
\begin{center}
\begin{longtable}{|p {6.5 cm}|p {4.5 cm}|p {3 cm}|p{3 cm}|p{1.5 cm}|p{1.5 cm}|p{2 cm}|}
\hline
\textbf{Keyword} & \textbf{Description} & \textbf{M. U.} & \textbf{range} & \textbf{Default Value} & \textbf{Scalar / Vector} & \textbf{Logical / Numeric} \\ \hline
\endfirsthead
\hline
\multicolumn{7}{| c |}{continued from previous page} \\
\hline
\textbf{Keyword} & \textbf{Description} & \textbf{M. U.} & \textbf{range} & \textbf{Default Value} & \textbf{Scalar / Vector} & \textbf{Logical / Numeric} \\ \hline
\endhead
\hline
\multicolumn{7}{| c |}{{continued on next page}}\\ 
\hline
\endfoot
\endlastfoot
\hline
ThetaResBedrock &  & - &  & 0.05 & vec & num \\ \hline
WiltingPointBedrock &  & - &  & 0.15 & vec & num \\ \hline
FieldCapacityBedrock &  & - &  & 0.25 & vec & num \\ \hline
ThetaSatBedrock &  & - &  & 0.5 & vec & num \\ \hline
AlphaVanGenuchtenBedrock &  & mm$^{-1}$ &  & 0.004 & vec & num \\ \hline
NVanGenuchtenBedrock &  & - &  & 1.3 & vec & num \\ \hline
VMualemBedrock &  & - &  & 0.5 & vec & num \\ \hline
ThermalConductivitySoilSolidsBedrock & thermal conductivity of the bedrock & W m$^{-1}$ K$^{-1}$ &  & 2.5 & vec & num \\ \hline
ThermalCapacitySoilSolidsBedrock & thermal capacity of the bedrock & J m$^{-3}$ K$^{-1}$ &  & 1.00E+06 & vec & num \\ \hline
SpecificStorativityBedrock &  & mm$^{-1}$ &  & 1.00E-07 & vec & num \\ \hline
\caption{Table of rock parameters  (numeric)}
\label{rock1d_numeric}
\end{longtable}
\end{center}






\clearpage
\begin{center}
\begin{longtable}{|p {6.5 cm}|p {4.5 cm}|p {3 cm}|p{3 cm}|p{1.5 cm}|p{1.5 cm}|p{2 cm}|}
\hline
\textbf{Keyword} & \textbf{Description} & \textbf{M. U.} & \textbf{range} & \textbf{Default Value} & \textbf{Scalar / Vector} & \textbf{Logical / Numeric} \\ \hline
\endfirsthead
\hline
\multicolumn{7}{| c |}{continued from previous page} \\
\hline
\textbf{Keyword} & \textbf{Description} & \textbf{M. U.} & \textbf{range} & \textbf{Default Value} & \textbf{Scalar / Vector} & \textbf{Logical / Numeric} \\ \hline
\endhead
\hline
\multicolumn{7}{| c |}{{continued on next page}}\\ 
\hline
\endfoot
\endlastfoot
\hline
InitDateDDMMYYYYhhmm & Date and time of the simulation start in date12 format (MANDATORY) & format DDMMYYhhmm & 01/01/1800 00:00, 01/01/2500 00:00 & NA & vec & str \\ \hline
EndDateDDMMYYYYhhmm & Date and time of the simulation start in date12 format (MANDATORY) & format DDMMYYhhmm & 01/01/1800 00:00, 01/01/2500 00:00 & NA & vec & str \\ \hline
NumSimulationTimes & How many times the simulation is run (if >1, it uses the final condition as initial conditions of the new simulation) & - & 0, inf & 1 & vec & num \\ \hline
StandardTimeSimulation & Standard time to which all the output data are referred (difference respect UMT, in hours): GMT + x [h] & h & 0, 12 & 0 & sca & num \\ \hline
PointSim & Point simulation (=1), distributed simulation (=0) & - & 0, 1 & 0 & sca & opt \\ \hline
RecoverSim & Simulation recovered (=number of saving point you want to start from), otherwise (=0) & - & 0, 1 & 0 & sca & opt \\ \hline
WaterBalance & Activate water balance (Yes=1, No=0) & - &  & 0 & sca & opt \\ \hline
EnergyBalance & Activate energy balance (Yes=1, No=0) &  &  & 0 & sca & opt \\ \hline
PixelCoordinates & Write 1 IF ALL point COORDINATES ARE IN FORMAT (EAST,NORTH) in meters, Or 0 IF IN FORMAT ROW AND COLUMS (r,c) of the dem map & - &  & 1 & sca & opt \\ \hline
SavingPoints &  & - &  & NA & vec & num \\ \hline
SoilLayerTypes & Number of types of soil types, corresponding to different soil stratigraphies & - &  & 1 & sca & num \\ \hline
DefaultSoilTypeLand & given a multiple number of type of soil, this relates to the default given to the land type type & - &  & 1 & sca & num \\ \hline
DefaultSoilTypeChannel & given a multiple number of type of soil, this relates to the default given to the channel type & - &  & 1 & sca & num \\ \hline
\caption{Table of general parameters  (numeric)}
\label{general_numeric}
\end{longtable}
\end{center}



\clearpage
\begin{center}
\begin{longtable}{|p {6.5 cm}|p {4.5 cm}|p {3 cm}|p{3 cm}|p{1.5 cm}|p{1.5 cm}|p{2 cm}|}
\hline
\textbf{Keyword} & \textbf{Description} & \textbf{M. U.} & \textbf{range} & \textbf{Default Value} & \textbf{Scalar / Vector} & \textbf{Logical / Numeric} \\ \hline
\endfirsthead
\hline
\multicolumn{7}{| c |}{continued from previous page} \\
\hline
\textbf{Keyword} & \textbf{Description} & \textbf{M. U.} & \textbf{range} & \textbf{Default Value} & \textbf{Scalar / Vector} & \textbf{Logical / Numeric} \\ \hline
\endhead
\hline
\multicolumn{7}{| c |}{{continued on next page}}\\ 
\hline
\endfoot
\endlastfoot
\hline
LWinParameterization & Which formula for incoming longwave radiation:  1 (Brutsaert, 1975), 2 (Satterlund, 1979), 3 (Idso, 1981), 4(Idso+Hodges),  5 (Koenig-Langlo \& Augstein, 1994), 6 (Andreas \& Ackley, 1982), 7 (Konzelmann, 1994), 8 (Prata, 1996), 9 (Dilley 1998) &  & 1, 2, .., 9 & 9 & sca & opt \\ \hline
MoninObukhov & Atmospherical stability parameter: 1 stability and instability considered, 2 stability not considered, 3 instability not considered, 4 always neutrality &  &  & 1 & sca & num \\ \hline
Surroundings & Yes(1), No(0) & - &  & 0 & sca & opt \\ \hline
\caption{Table of surface energy flux parameters  (numeric)}
\label{surfaceenergyfluxes_numeric}
\end{longtable}
\end{center}


\clearpage
\begin{center}
\begin{longtable}{|p {6.5 cm}|p {4.5 cm}|p {3 cm}|p{3 cm}|p{1.5 cm}|p{1.5 cm}|p{2 cm}|}
\hline
\textbf{Keyword} & \textbf{Description} & \textbf{M. U.} & \textbf{range} & \textbf{Default Value} & \textbf{Scalar / Vector} & \textbf{Logical / Numeric} \\ \hline
\endfirsthead
\hline
\multicolumn{7}{| c |}{continued from previous page} \\
\hline
\textbf{Keyword} & \textbf{Description} & \textbf{M. U.} & \textbf{range} & \textbf{Default Value} & \textbf{Scalar / Vector} & \textbf{Logical / Numeric} \\ \hline
\endhead
\hline
\multicolumn{7}{| c |}{{continued on next page}}\\ 
\hline
\endfoot
\endlastfoot
\hline
Latitude & Average latitude of the basin, positive means north, negative means south (MANDATORY) & degree & -90, 90 & 45 & sca & num \\ \hline
Longitude & Average longitude of the basin, eastwards from 0 meridiane (MANDATORY) & degree & 0, 180 & 0 & sca & num \\ \hline
PointID & identification code for the point of simulation &  &  & NA & sca & num \\ \hline
CoordinatePointX & coordinate X if PixelCoordinates is 1, number of row of the matrix if PixelCoordinates is 0 & m (according to the geographical projection of the maps) &  & NA & vec & num \\ \hline
CoordinatePointY & coordinate Y if PixelCoordinates is 1, number of column of the matrix if PixelCoordinates is 1 & m (according to the geographical projection of the maps) &  & NA & vec & num \\ \hline
PointElevation & elevation of the point of simulation & m a.s.l. &  & NA & vec & num \\ \hline
PointSlope & Slope steepness of the simulation point & degree &  & NA & vec & num \\ \hline
PointAspect & Aspect of the simulation point & degree &  & NA & vec & num \\ \hline
PointSkyViewFactor & Sky View Factor of the simulation point & - &  & NA & vec & num \\ \hline
PointCurvatureNorthSouthDirection & N-S curvature of the simulation point & m$^{-1}$ &  & NA & vec & num \\ \hline
PointCurvatureWestEastDirection & W-E curvature of the simulation point & m$^{-1}$ &  & NA & vec & num \\ \hline
PointCurvatureNorthwest SoutheastDirection & N-W curvature of the simulation point & m$^{-1}$ &  & NA & vec & num \\ \hline
PointCurvatureNortheast SouthwestDirection & N-E curvature of the simulation point & m$^{-1}$ &  & NA & vec & num \\ \hline
PointDrainageLateralDistance & Lateral Drainage distance of the simulation point & m &  & NA & vec & num \\ \hline
PointLatitude & Latitude of the simulation point & degree &  & NA & sca & num \\ \hline
PointLongitude & Longitude of the simulation point & degree &  & NA & sca & num \\ \hline
\caption{Table of topographic parameters  (numeric)}
\label{topo3d_numeric}
\end{longtable}
\end{center}



\clearpage
\section{3D OUTPUT NUMERIC}


\clearpage
\begin{center}
\begin{longtable}{|p {6.5 cm}|p {4.5 cm}|p {3 cm}|p{3 cm}|p{1.5 cm}|p{1.5 cm}|p{2 cm}|}
\hline
\textbf{Keyword} & \textbf{Description} & \textbf{M. U.} & \textbf{range} & \textbf{Default Value} & \textbf{Scalar / Vector} & \textbf{Logical / Numeric} \\ \hline
\endfirsthead
\hline
\multicolumn{7}{| c |}{continued from previous page} \\
\hline
\textbf{Keyword} & \textbf{Description} & \textbf{M. U.} & \textbf{range} & \textbf{Default Value} & \textbf{Scalar / Vector} & \textbf{Logical / Numeric} \\ \hline
\endhead
\hline
\multicolumn{7}{| c |}{{continued on next page}}\\ 
\hline
\endfoot
\endlastfoot
\hline
DtPlotPoint & Plotting Time step (in hour) of THE OUTPUT FOR SPECIFIED PIXELS (0 means the it is not plotted) & h & 0, inf & 0 & vec & num \\ \hline
DatePoint & column number in which one would like to visualize the Date12[DDMMYYYYhhmm]    	 & - & 1, 76 & -1 & sca & num \\ \hline
JulianDayFromYear0Point & column number in which one would like to visualize the JulianDayFromYear0[days]   	 & - & 1, 76 & -1 & sca & num \\ \hline
TimeFromStartPoint & column number in which one would like to visualize the TimeFromStart[days]  & - & 1, 76 & -1 & sca & num \\ \hline
PeriodPoint & column number in which one would like to visualize the Simulation\_Period & - & 1, 76 & -1 & sca & num \\ \hline
RunPoint & column number in which one would like to visualize the Run	 & - & 1, 76 & -1 & sca & num \\ \hline
IDPointPoint & column number in which one would like to visualize the IDpoint  & - & 1, 76 & -1 & sca & num \\ \hline
PsnowPoint & column number in which one would like to visualize the Psnow\_over\_canopy[mm]      & - & 1, 76 & -1 & sca & num \\ \hline
PrainPoint & column number in which one would like to visualize the Prain\_over\_canopy[mm] 	 & - & 1, 76 & -1 & sca & num \\ \hline
PsnowNetPoint & column number in which one would like to visualize the Psnow\_under\_canopy[mm]  & - & 1, 76 & -1 & sca & num \\ \hline
PrainNetPoint & column number in which one would like to visualize the Prain\_under\_canopy[mm] 	 & - & 1, 76 & -1 & sca & num \\ \hline
PrainOnSnowPoint & column number in which one would like to visualize the Prain\_rain\_on\_snow[mm] & - & 1, 76 & -1 & sca & num \\ \hline
WindSpeedPoint & column number in which one would like to visualize the Wind\_speed[m/s]          & - & 1, 76 & -1 & sca & num \\ \hline
WindDirPoint & column number in which one would like to visualize the Wind\_direction[deg]   & - & 1, 76 & -1 & sca & num \\ \hline
RHPoint & column number in which one would like to visualize the Relative\_Humidity[-]     & - & 1, 76 & -1 & sca & num \\ \hline
AirPressPoint & column number in which one would like to visualize the Pressure[mbar]     & - & 1, 76 & -1 & sca & num \\ \hline
AirTempPoint & column number in which one would like to visualize the Tair[\textcelsius]     & - & 1, 76 & -1 & sca & num \\ \hline
TDewPoint & column number in which one would like to visualize the Tdew[\textcelsius]   & - & 1, 76 & -1 & sca & num \\ \hline
TsurfPoint & column number in which one would like to visualize the Tsurface[\textcelsius]     & - & 1, 76 & -1 & sca & num \\ \hline
TvegPoint & column number in which one would like to visualize the Tvegetation[\textcelsius]     & - & 1, 76 & -1 & sca & num \\ \hline
TCanopyAirPoint & column number in which one would like to visualize the Tcanopyair[\textcelsius]     & - & 1, 76 & -1 & sca & num \\ \hline
SurfaceEBPoint & column number in which one would like to visualize the Surface\_Energy\_balance [W/m2]     & - & 1, 76 & -1 & sca & num \\ \hline
SoilHeatFluxPoint & column number in which one would like to visualize the Soil\_heat\_flux[W/m2]      & - & 1, 76 & -1 & sca & num \\ \hline
SWinPoint & column number in which one would like to visualize the SWin[W/m2]   & - & 1, 76 & -1 & sca & num \\ \hline
SWbeamPoint & column number in which one would like to visualize the SWbeam[W/m2]    & - & 1, 76 & -1 & sca & num \\ \hline
SWdiffPoint & column number in which one would like to visualize the SWdiff[W/m2]   & - & 1, 76 & -1 & sca & num \\ \hline
LWinPoint & column number in which one would like to visualize the LWin[W/m2]  & - & 1, 76 & -1 & sca & num \\ \hline
LWinMinPoint & column number in which one would like to visualize the LWin\_min[W/m2]  & - & 1, 76 & -1 & sca & num \\ \hline
LWinMaxPoint & column number in which one would like to visualize the LWin\_max[W/m2] & - & 1, 76 & -1 & sca & num \\ \hline
SWNetPoint & column number in which one would like to visualize the SWnet[W/m2]      & - & 1, 76 & -1 & sca & num \\ \hline
LWNetPoint & column number in which one would like to visualize the LWnet[W/m2]      & - & 1, 76 & -1 & sca & num \\ \hline
HPoint & column number in which one would like to visualize the H[W/m2]       & - & 1, 76 & -1 & sca & num \\ \hline
LEPoint & column number in which one would like to visualize the LE[W/m2]      & - & 1, 76 & -1 & sca & num \\ \hline
CanopyFractionPoint & column number in which one would like to visualize the Canopy\_fraction[-]      & - & 1, 76 & -1 & sca & num \\ \hline
LSAIPoint & column number in which one would like to visualize the LSAI[m2/m2]    & - & 1, 76 & -1 & sca & num \\ \hline
z0vegPoint & column number in which one would like to visualize the z0veg[m]     & - & 1, 76 & -1 & sca & num \\ \hline
d0vegPoint & column number in which one would like to visualize the d0veg[m]     & - & 1, 76 & -1 & sca & num \\ \hline
EstoredCanopyPoint & column number in which one would like to visualize the Estored\_canopy[W/m2]    & - & 1, 76 & -1 & sca & num \\ \hline
SWvPoint & column number in which one would like to visualize the SWv[W/m2]     & - & 1, 76 & -1 & sca & num \\ \hline
LWvPoint & column number in which one would like to visualize the LWv[W/m2]     & - & 1, 76 & -1 & sca & num \\ \hline
HvPoint & column number in which one would like to visualize the Hv[W/m2]      & - & 1, 76 & -1 & sca & num \\ \hline
LEvPoint & column number in which one would like to visualize the LEv[W/m2]     & - & 1, 76 & -1 & sca & num \\ \hline
HgUnvegPoint & column number in which one would like to visualize the Hg\_unveg[W/m2]     & - & 1, 76 & -1 & sca & num \\ \hline
LEgUnvegPoint & column number in which one would like to visualize the LEg\_unveg[W/m2]    & - & 1, 76 & -1 & sca & num \\ \hline
HgVegPoint & column number in which one would like to visualize the Hg\_veg[W/m2]     & - & 1, 76 & -1 & sca & num \\ \hline
LEgVegPoint & column number in which one would like to visualize the LEg\_veg[W/m2]    & - & 1, 76 & -1 & sca & num \\ \hline
EvapSurfacePoint & column number in which one would like to visualize the Evap\_surface[mm]   & - & 1, 76 & -1 & sca & num \\ \hline
TraspCanopyPoint & column number in which one would like to visualize the Trasp\_canopy[mm]     & - & 1, 76 & -1 & sca & num \\ \hline
WaterOnCanopyPoint & column number in which one would like to visualize the Water\_on\_canopy[mm] & - & 1, 76 & -1 & sca & num \\ \hline
SnowOnCanopyPoint & column number in which one would like to visualize the Snow\_on\_canopy[mm] & - & 1, 76 & -1 & sca & num \\ \hline
QVegPoint & column number in which one would like to visualize the specific humidity near the vegetation (grams vapour/grams air)  & - & 1, 76 & -1 & sca & num \\ \hline
QSurfPoint & column number in which one would like to visualize the specific humidity at the surface (grams vapour/grams air)  & - & 1, 76 & -1 & sca & num \\ \hline
QAirPoint & column number in which one would like to visualize the specific humidity at air (grams vapour/grams air)  & - & 1, 76 & -1 & sca & num \\ \hline
QCanopyAirPoint & column number in which one would like to visualize the specific humidity at the canopy-air interface (grams vapour/grams air)  & - & 1, 76 & -1 & sca & num \\ \hline
LObukhovPoint & column number in which one would like to visualize the LObukhov[m] & - & 1, 76 & -1 & sca & num \\ \hline
LObukhovCanopyPoint & column number in which one would like to visualize the LObukhovcanopy[m] & - & 1, 76 & -1 & sca & num \\ \hline
WindSpeedTopCanopyPoint & column number in which one would like to visualize the Wind\_speed\_top\_canopy [m/s]     & - & 1, 76 & -1 & sca & num \\ \hline
DecayKCanopyPoint & column number in which one would like to visualize the Decay\_of\_K\_in\_canopy[-]    & - & 1, 76 & -1 & sca & num \\ \hline
SWupPoint & column number in which one would like to visualize the SWup[W/m$^{2}$]    & - & 1, 76 & -1 & sca & num \\ \hline
LWupPoint & column number in which one would like to visualize the LWup[W/m$^{2}$]    & - & 1, 76 & -1 & sca & num \\ \hline
HupPoint & column number in which one would like to visualize the Hup[W/m$^{2}$]     & - & 1, 76 & -1 & sca & num \\ \hline
LEupPoint & column number in which one would like to visualize the LEup[W/m$^{2}$]    & - & 1, 76 & -1 & sca & num \\ \hline
SnowDepthPoint & column number in which one would like to visualize the snow\_depth[mm]  & - & 1, 76 & -1 & sca & num \\ \hline
SWEPoint & column number in which one would like to visualize the snow\_water\_equivalent [mm]  & - & 1, 76 & -1 & sca & num \\ \hline
SnowDensityPoint & column number in which one would like to visualize the snow\_density[kg/m$^{3}$]  & - & 1, 76 & -1 & sca & num \\ \hline
SnowTempPoint & column number in which one would like to visualize the snow\_temperature[\textcelsius]  & - & 1, 76 & -1 & sca & num \\ \hline
SnowMeltedPoint & column number in which one would like to visualize the snow\_melted[mm]  & - & 1, 76 & -1 & sca & num \\ \hline
SnowSublPoint & column number in which one would like to visualize the snow\_subl[mm]  & - & 1, 76 & -1 & sca & num \\ \hline
SWEBlownPoint & column number in which one would like to visualize the snow\_blown\_away[mm]  & - & 1, 76 & -1 & sca & num \\ \hline
SWESublBlownPoint & column number in which one would like to visualize the snow\_subl\_while\_blown [mm] & - & 1, 76 & -1 & sca & num \\ \hline
GlacDepthPoint & column number in which one would like to visualize the glac\_depth[mm]  & - & 1, 76 & -1 & sca & num \\ \hline
GWEPoint & column number in which one would like to visualize the glac\_water\_equivalent[mm]  & - & 1, 76 & -1 & sca & num \\ \hline
GlacDensityPoint & column number in which one would like to visualize the glac\_density[kg/m$^{3}$]  & - & 1, 76 & -1 & sca & num \\ \hline
GlacTempPoint & column number in which one would like to visualize the glac\_temperature[\textcelsius]  & - & 1, 76 & -1 & sca & num \\ \hline
GlacMeltedPoint & column number in which one would like to visualize the glac\_melted[mm]  & - & 1, 76 & -1 & sca & num \\ \hline
GlacSublPoint & column number in which one would like to visualize the glac\_subl[mm]  & - & 1, 76 & -1 & sca & num \\ \hline
ThawedSoilDepthPoint & column number in which one would like to visualize the thawed\_soil\_depth[mm]  & - & 1, 76 & -1 & sca & num \\ \hline
WaterTableDepthPoint & column number in which one would like to visualize the water\_table\_depth[mm]  & - & 1, 76 & -1 & sca & num \\ \hline
DefaultPoint & 0: use personal setting, 1:use default & - & 0, 1 & 1 & sca & opt \\ \hline
\caption{Table of point output  (numeric)}
\label{point3d_numeric}
\end{longtable}
\end{center}











\clearpage
\begin{center}
\begin{longtable}{|p {6.5 cm}|p {4.5 cm}|p {3 cm}|p{3 cm}|p{1.5 cm}|p{1.5 cm}|p{2 cm}|}
\hline
\textbf{Keyword} & \textbf{Description} & \textbf{M. U.} & \textbf{range} & \textbf{Default Value} & \textbf{Scalar / Vector} & \textbf{Logical / Numeric} \\ \hline
\endfirsthead
\hline
\multicolumn{7}{| c |}{continued from previous page} \\
\hline
\textbf{Keyword} & \textbf{Description} & \textbf{M. U.} & \textbf{range} & \textbf{Default Value} & \textbf{Scalar / Vector} & \textbf{Logical / Numeric} \\ \hline
\endhead
\hline
\multicolumn{7}{| c |}{{continued on next page}}\\ 
\hline
\endfoot
\endlastfoot
\hline
OutputMeteoMaps & frequency (h) of printing of the results of the meteo maps & h &  & 0 & sca & num \\ \hline
SpecialPlotBegin & date of begin of plotting of the special output & format DDMMYYhhmm & 01/01/1800 00:00, 01/01/2500 00:00 & 0 & vec & str \\ \hline
SpecialPlotEnd & date of end of plotting of the special output & format DDMMYYhhmm & 01/01/1800 00:00, 01/01/2500 00:00 & 0 & vec & str \\ \hline
\caption{Table of meteo output  (numeric)}
\label{meteo_numeric}
\end{longtable}
\end{center}


\clearpage
\begin{center}
\begin{longtable}{|p {6.5 cm}|p {4.5 cm}|p {3 cm}|p{3 cm}|p{1.5 cm}|p{1.5 cm}|p{2 cm}|}
\hline
\textbf{Keyword} & \textbf{Description} & \textbf{M. U.} & \textbf{range} & \textbf{Default Value} & \textbf{Scalar / Vector} & \textbf{Logical / Numeric} \\ \hline
\endfirsthead
\hline
\multicolumn{7}{| c |}{continued from previous page} \\
\hline
\textbf{Keyword} & \textbf{Description} & \textbf{M. U.} & \textbf{range} & \textbf{Default Value} & \textbf{Scalar / Vector} & \textbf{Logical / Numeric} \\ \hline
\endhead
\hline
\multicolumn{7}{| c |}{{continued on next page}}\\ 
\hline
\endfoot
\endlastfoot
\hline
OutputSnowMaps & frequency (h) of printing of the results of the snow maps & h &  & 0 & sca & num \\ \hline
DateSnow & column number in which one would like to visualize the Date12[DDMMYYYY hhmm]    	 & - &  & -1 & sca & num \\ \hline
JulianDayFromYear0Snow & column number in which one would like to visualize the JulianDayFromYear0[days]   	 & - &  & -1 & sca & num \\ \hline
TimeFromStartSnow & column in which one would like to visualize the TimeFromStart[days]     & - &  & -1 & sca & num \\ \hline
PeriodSnow & Column number to write the period number & - &  & -1 & sca & num \\ \hline
RunSnow & Column number to write the run number & - &  & -1 & sca & num \\ \hline
IDPointSnow & column number in which one would like to visualize the IDpoint  & - &  & -1 & sca & num \\ \hline
WaterEquivalentSnow & column number in which one would like the water equivalent of the snow & - &  & -1 & sca & num \\ \hline
DepthSnow & column number in which one would like to visualize the depth of the snow & - &  & -1 & sca & num \\ \hline
DensitySnow & column number in which one would like to visualize the density of the snow & - &  & -1 & sca & num \\ \hline
TempSnow & column number in which one would like to visualize the temperature of the snow  & - &  & -1 & sca & num \\ \hline
IceContentSnow & column number in which one would like to visualize the ice content of the snow  & - &  & -1 & sca & num \\ \hline
WatContentSnow & column number in which one would like to visualize the water content of the snow  & - &  & -1 & sca & num \\ \hline
DefaultSnow & 0: use personal setting, 1:use default & - & 0, 1 & 1 & sca & opt \\ \hline
SnowPlotDepths & depth at which one wants the data on the snow to be plotted & - &  & NA & vec & num \\ \hline
\caption{Table of snow output  (numeric)}
\label{snow_numeric}
\end{longtable}
\end{center}





\clearpage
\begin{center}
\begin{longtable}{|p {6.5 cm}|p {4.5 cm}|p {3 cm}|p{3 cm}|p{1.5 cm}|p{1.5 cm}|p{2 cm}|}
\hline
\textbf{Keyword} & \textbf{Description} & \textbf{M. U.} & \textbf{range} & \textbf{Default Value} & \textbf{Scalar / Vector} & \textbf{Logical / Numeric} \\ \hline
\endfirsthead
\hline
\multicolumn{7}{| c |}{continued from previous page} \\
\hline
\textbf{Keyword} & \textbf{Description} & \textbf{M. U.} & \textbf{range} & \textbf{Default Value} & \textbf{Scalar / Vector} & \textbf{Logical / Numeric} \\ \hline
\endhead
\hline
\multicolumn{7}{| c |}{{continued on next page}}\\ 
\hline
\endfoot
\endlastfoot
\hline
OutputVegetationMaps & frequency (h) of printing of the results of the vegetation maps & h &  & 0 & sca & num \\ \hline
\caption{Table of vegetation output  (numeric)}
\label{vegetation_numeric}
\end{longtable}
\end{center}

\clearpage



\section{1D OUTPUT NUMERIC}







\clearpage
\begin{center}
\begin{longtable}{|p {6.5 cm}|p {4.5 cm}|p {3 cm}|p{3 cm}|p{1.5 cm}|p{1.5 cm}|p{2 cm}|}
\hline
\textbf{Keyword} & \textbf{Description} & \textbf{M. U.} & \textbf{range} & \textbf{Default Value} & \textbf{Scalar / Vector} & \textbf{Logical / Numeric} \\ \hline
\endfirsthead
\hline
\multicolumn{7}{| c |}{continued from previous page} \\
\hline
\textbf{Keyword} & \textbf{Description} & \textbf{M. U.} & \textbf{range} & \textbf{Default Value} & \textbf{Scalar / Vector} & \textbf{Logical / Numeric} \\ \hline
\endhead
\hline
\multicolumn{7}{| c |}{{continued on next page}}\\ 
\hline
\endfoot
\endlastfoot
\hline
DateSnow & column number in which one would like to visualize the Date12 [DDMMYYYYhhmm]    	 & - &  & -1 & sca & num \\ \hline
JulianDayFromYear0Snow & column number in which one would like to visualize the JulianDayFromYear0[days]   	 & - &  & -1 & sca & num \\ \hline
TimeFromStartSnow & column in which one would like to visualize the TimeFromStart[days]     & - &  & -1 & sca & num \\ \hline
PeriodSnow & Column number to write the period number & - &  & -1 & sca & num \\ \hline
RunSnow & Column number to write the run number & - &  & -1 & sca & num \\ \hline
IDPointSnow & column number in which one would like to visualize the IDpoint  & - &  & -1 & sca & num \\ \hline
WaterEquivalentSnow & column number in which one would like the water equivalent of the snow & - &  & -1 & sca & num \\ \hline
DepthSnow & column number in which one would like to visualize the depth of the snow & - &  & -1 & sca & num \\ \hline
DensitySnow & column number in which one would like to visualize the density of the snow & - &  & -1 & sca & num \\ \hline
TempSnow & column number in which one would like to visualize the temperature of the snow  & - &  & -1 & sca & num \\ \hline
IceContentSnow & column number in which one would like to visualize the ice content of the snow  & - &  & -1 & sca & num \\ \hline
WatContentSnow & column number in which one would like to visualize the water content of the snow  & - &  & -1 & sca & num \\ \hline
DefaultSnow & 0: use personal setting, 1:use default & - & 0, 1 & 1 & sca & opt \\ \hline
SnowPlotDepths & depth at which one wants the data on the snow to be plotted & - &  & NA & vec & num \\ \hline
\caption{Table of snow output  (numeric)}
\label{snow1d_numeric}
\end{longtable}
\end{center}

\clearpage
\begin{center}
\begin{longtable}{|p {6.5 cm}|p {4.5 cm}|p {3 cm}|p{3 cm}|p{1.5 cm}|p{1.5 cm}|p{2 cm}|}
\hline
\textbf{Keyword} & \textbf{Description} & \textbf{M. U.} & \textbf{range} & \textbf{Default Value} & \textbf{Scalar / Vector} & \textbf{Logical / Numeric} \\ \hline
\endfirsthead
\hline
\multicolumn{7}{| c |}{continued from previous page} \\
\hline
\textbf{Keyword} & \textbf{Description} & \textbf{M. U.} & \textbf{range} & \textbf{Default Value} & \textbf{Scalar / Vector} & \textbf{Logical / Numeric} \\ \hline
\endhead
\hline
\multicolumn{7}{| c |}{{continued on next page}}\\ 
\hline
\endfoot
\endlastfoot
\hline
DateSoil & column number in which one would like to visualize the Date12 [DDMMYYYYhhmm]    	 & - &  & -1 & sca & num \\ \hline
JulianDayFromYear0Soil & column number in which one would like to visualize the JulianDayFromYear0[days]   	 & - &  & -1 & sca & num \\ \hline
TimeFromStartSoil & column number in which one would like to visualize the time from the start of the soil & - &  & -1 & sca & num \\ \hline
PeriodSoil & Column number to write the period number & - &  & -1 & sca & num \\ \hline
RunSoil & Column number to write the run number & - &  & -1 & sca & num \\ \hline
IDPointSoil & column number in which one would like to visualize the IDpoint  & - &  & -1 & sca & num \\ \hline
DefaultSoil & 0: use personal setting, 1:use default & - & 0, 1 & 1 & sca & opt \\ \hline
SoilPlotDepths & depth at which one wants the data on the snow to be plotted & m &  & NA & vec & num \\ \hline
\caption{Table of snow output  (numeric)}
\label{soil1d_numeric}
\end{longtable}
\end{center}

\section{1D INPUT CHARACTER}

\begin{center}
\begin{longtable}{|p {7 cm}|p {7 cm}|p {3 cm}|p {4 cm}|}
\hline
\textbf{Keyword} & \textbf{Description} & \textbf{Associated file} & \textbf{type (file, header)} \\ \hline
\endfirsthead
\hline
\multicolumn{4}{| c |}{continued from previous page} \\
\hline
\textbf{Keyword} & \textbf{Description} & \textbf{Associated file} & \textbf{type (file, header)} \\ \hline
\endhead
\hline
\multicolumn{4}{| c |}{{continued on next page}}\\ 
\hline
\endfoot
\endlastfoot
\hline
HeaderSoilInitPres & column name in the file SoilParFile for the initial total pressure head & SoilParFile & header \\ \hline
HeaderSoilInitTemp & column name in the file SoilParFile for the initial temperature & SoilParFile & header \\ \hline
\caption{Table of initial conditions (character)}
\label{IC1D_data}
\end{longtable}
\end{center}
\clearpage



\begin{center}
\begin{longtable}{|p {7 cm}|p {7 cm}|p {3 cm}|p {4 cm}|}
\hline
\textbf{Keyword} & \textbf{Description} & \textbf{Associated file} & \textbf{type (file, header)} \\ \hline
\endfirsthead
\hline
\multicolumn{4}{| c |}{continued from previous page} \\
\hline
\textbf{Keyword} & \textbf{Description} & \textbf{Associated file} & \textbf{type (file, header)} \\ \hline
\endhead
\hline
\multicolumn{4}{| c |}{{continued on next page}}\\ 
\hline
\endfoot
\endlastfoot
\hline
HeaderPointDepthFreeSurface & column name in the file PointFile for the depth of the free surface of the point & PointFile & header \\ \hline
\caption{Table of runoff parameters (character)}
\label{runoff1D_data}
\end{longtable}
\end{center}
\clearpage


\begin{center}
\begin{longtable}{|p {7 cm}|p {7 cm}|p {3 cm}|p {4 cm}|}
\hline
\textbf{Keyword} & \textbf{Description} & \textbf{Associated file} & \textbf{type (file, header)} \\ \hline
\endfirsthead
\hline
\multicolumn{4}{| c |}{continued from previous page} \\
\hline
\textbf{Keyword} & \textbf{Description} & \textbf{Associated file} & \textbf{type (file, header)} \\ \hline
\endhead
\hline
\multicolumn{4}{| c |}{{continued on next page}}\\ 
\hline
\endfoot
\endlastfoot
\hline
HeaderPointMaxSWE & column name in the file PointFile for the max SWE  of the point & PointFile & header  \\ \hline
\caption{Table of snow parameters (character)}
\label{snow1D_data}
\end{longtable}
\end{center}
\clearpage

\begin{center}
\begin{longtable}{|p {7 cm}|p {7 cm}|p {3 cm}|p {4 cm}|}
\hline
\textbf{Keyword} & \textbf{Description} & \textbf{Associated file} & \textbf{type (file, header)} \\ \hline
\endfirsthead
\hline
\multicolumn{4}{| c |}{continued from previous page} \\
\hline
\textbf{Keyword} & \textbf{Description} & \textbf{Associated file} & \textbf{type (file, header)} \\ \hline
\endhead
\hline
\multicolumn{4}{| c |}{{continued on next page}}\\ 
\hline
\endfoot
\endlastfoot
\hline
HeaderPointSoilType & column name in the file PointFile for the soil type of the point & PointFile & header \\ \hline
HeaderSoilDz & column name in the file SoilParFile for the layers thickness & SoilParFile & header \\ \hline
HeaderNormalHydrConductivity & column name in the file SoilParFile for the normal hydraulic conductivity & SoilParFile & header \\ \hline
HeaderLateralHydrConductivity & column name in the file SoilParFile for the lateral hydraulic conductivity & SoilParFile & header \\ \hline
HeaderThetaRes & column name in the file SoilParFile for the residual water content & SoilParFile & header \\ \hline
HeaderWiltingPoint & column name in the file SoilParFile for the soil wilting point & SoilParFile & header \\ \hline
HeaderFieldCapacity & column name in the file SoilParFile for the field capacity & SoilParFile & header \\ \hline
HeaderThetaSat & column name in the file SoilParFile for the saturated water content & SoilParFile & header \\ \hline
HeaderAlpha & column name in the file alpha parameter of Van Genuchten & SoilParFile & header \\ \hline
HeaderN & column name in the file N parameter of Van Genuchten & SoilParFile & header \\ \hline
HeaderV & column name in the file V parameter of Van Genuchten & SoilParFile & header \\ \hline
HeaderKthSoilSolids & column name in the file thermal conductivity of the soil grains & SoilParFile & header \\ \hline
HeaderCthSoilSolids & column name in the file thermal capacity of the soil grains & SoilParFile & header \\ \hline
HeaderSpecificStorativity & column name in the file specific storativity & SoilParFile & header \\ \hline

\caption{Table of soil parameters (character)}
\label{soil1D_data}
\end{longtable}
\end{center}
\clearpage


\begin{center}
\begin{longtable}{|p {7 cm}|p {7 cm}|p {3 cm}|p {4 cm}|}
\hline
\textbf{Keyword} & \textbf{Description} & \textbf{Associated file} & \textbf{type (file, header)} \\ \hline
\endfirsthead
\hline
\multicolumn{4}{| c |}{continued from previous page} \\
\hline
\textbf{Keyword} & \textbf{Description} & \textbf{Associated file} & \textbf{type (file, header)} \\ \hline
\endhead
\hline
\multicolumn{4}{| c |}{{continued on next page}}\\ 
\hline
\endfoot
\endlastfoot
\hline
HeaderPointLandCoverType & column name in the file PointFile for the land cover of the point & PointFile & header \\ \hline
\caption{Table of soil surface parameters (character)}
\label{soilsur1D_data}
\end{longtable}
\end{center}
\clearpage



\section{3D INPUT CHARACTER}



\begin{center}
\begin{longtable}{|p {7 cm}|p {7 cm}|p {3 cm}|p {4 cm}|}
\hline
\textbf{Keyword} & \textbf{Description} & \textbf{Associated file} & \textbf{type (file, header)} \\ \hline
\endfirsthead
\hline
\multicolumn{4}{| c |}{continued from previous page} \\
\hline
\textbf{Keyword} & \textbf{Description} & \textbf{Associated file} & \textbf{type (file, header)} \\ \hline
\endhead
\hline
\multicolumn{4}{| c |}{{continued on next page}}\\ 
\hline
\endfoot
\endlastfoot
\hline
TimeDependentVegetationParameterFile & name of the file providing the time dependent vegetation parameters & / & file \\ \hline
\caption{Table of vegetation parameters (character)}
\label{vegetation3d_data}
\end{longtable}
\end{center}
\clearpage

\section{3D INPUT CHARACTER}

\begin{center}
\begin{longtable}{|p {7 cm}|p {7 cm}|p {3 cm}|p {4 cm}|}
\hline
\textbf{Keyword} & \textbf{Description} & \textbf{Associated file} & \textbf{type (file, header)} \\ \hline
\endfirsthead
\hline
\multicolumn{4}{| c |}{continued from previous page} \\
\hline
\textbf{Keyword} & \textbf{Description} & \textbf{Associated file} & \textbf{type (file, header)} \\ \hline
\endhead
\hline
\multicolumn{4}{| c |}{{continued on next page}}\\ 
\hline
\endfoot
\endlastfoot
\hline
HeaderDateDDMMYYYYhhmmMeteo & column name in the file MeteoFile for the variable DateDDMMYYYhhmmMeteo & MeteoFile & header \\ \hline
HeaderJulianDayfrom0Meteo & column name in the file MeteoFile for the variable julian day from 0 & MeteoFile & header \\ \hline
HeaderIPrec & column name in the file MeteoFile for the variable precipitation & MeteoFile & header \\ \hline
HeaderWindVelocity & column name in the file MeteoFile for the variable wind speed & MeteoFile & header \\ \hline
HeaderWindDirection & column name in the file MeteoFile for the variable wind direction & MeteoFile & header \\ \hline
HeaderWindX & column name in the file MeteoFile for the variable wind X & MeteoFile & header \\ \hline
HeaderWindY & column name in the file MeteoFile for the variable wind Y & MeteoFile & header \\ \hline
HeaderRH & column name in the file MeteoFile for the variable Relative humidity & MeteoFile & header \\ \hline
HeaderAirTemp & column name in the file MeteoFile for the variable Air Temperature & MeteoFile & header \\ \hline
HeaderDewTemp & column name in the file MeteoFile for the variable Dew temperature & MeteoFile & header \\ \hline
HeaderAirPress & column name in the file MeteoFile for the variable Air Pressure & MeteoFile & header \\ \hline
HeaderSWglobal & column name in the file MeteoFile for the variable SW global & MeteoFile & header \\ \hline
HeaderSWdirect & column name in the file MeteoFile for the variable Swdirect & MeteoFile & header \\ \hline
HeaderSWdiffuse & column name in the file MeteoFile for the variable Swdiffuse & MeteoFile & header \\ \hline
HeaderCloudSWTransmissivity & column name in the file MeteoFile for the variable transmissivity of SW through cloud & MeteoFile & header \\ \hline
HeaderCloudFactor & column name in the file MeteoFile for the variable cloud factor & MeteoFile & header \\ \hline
HeaderLWin & column name in the file MeteoFile for the variable LW in & MeteoFile & header \\ \hline
HeaderSWnet & column name in the file MeteoFile for the variable SW net & MeteoFile & header \\ \hline
\caption{Table of meteorological forcing (meteo data - character)}
\label{meteo_data}
\end{longtable}
\end{center}
\clearpage

\begin{center}
\begin{longtable}{|p {7 cm}|p {7 cm}|p {3 cm}|p {4 cm}|}
\hline
\textbf{Keyword} & \textbf{Description} & \textbf{Associated file} & \textbf{type (file, header)} \\ \hline
\endfirsthead
\hline
\multicolumn{4}{| c |}{continued from previous page} \\
\hline
\textbf{Keyword} & \textbf{Description} & \textbf{Associated file} & \textbf{type (file, header)} \\ \hline
\endhead
\hline
\multicolumn{4}{| c |}{{continued on next page}}\\ 
\hline
\endfoot
\endlastfoot
\hline
HeaderSoilDz & column name in the file SoilParFile for the layers thickness & SoilParFile & header \\ \hline
HeaderNormalHydrConductivity & column name in the file SoilParFile for the normal hydraulic conductivity & SoilParFile & header \\ \hline
HeaderLateralHydrConductivity & column name in the file SoilParFile for the lateral hydraulic conductivity & SoilParFile & header \\ \hline
HeaderThetaRes & column name in the file SoilParFile for the residual water content & SoilParFile & header \\ \hline
HeaderWiltingPoint & column name in the file SoilParFile for the soil wilting point & SoilParFile & header \\ \hline
HeaderFieldCapacity & column name in the file SoilParFile for the field capacity & SoilParFile & header \\ \hline
HeaderThetaSat & column name in the file SoilParFile for the saturated water content & SoilParFile & header \\ \hline
HeaderAlpha & column name in the file alpha parameter of Van Genuchten & SoilParFile & header \\ \hline
HeaderN & column name in the file N parameter of Van Genuchten & SoilParFile & header \\ \hline
HeaderV & column name in the file V parameter of Van Genuchten & SoilParFile & header \\ \hline
HeaderKthSoilSolids & column name in the file thermal conductivity of the soil grains & SoilParFile & header \\ \hline
HeaderCthSoilSolids & column name in the file thermal capacity of the soil grains & SoilParFile & header \\ \hline
HeaderSpecificStorativity & column name in the file specific storativity & SoilParFile & header \\ \hline
\caption{Table of soil (character)}
\label{soil3d_data}
\end{longtable}
\end{center}
\clearpage

\begin{center}
\begin{longtable}{|p {7 cm}|p {7 cm}|p {3 cm}|p {4 cm}|}
\hline
\textbf{Keyword} & \textbf{Description} & \textbf{Associated file} & \textbf{type (file, header)} \\ \hline
\endfirsthead
\hline
\multicolumn{4}{| c |}{continued from previous page} \\
\hline
\textbf{Keyword} & \textbf{Description} & \textbf{Associated file} & \textbf{type (file, header)} \\ \hline
\endhead
\hline
\multicolumn{4}{| c |}{{continued on next page}}\\ 
\hline
\endfoot
\endlastfoot
\hline
InitWaterTableHeightOverTopoSurface MapFile & name of the file providing the initial condition on the water table height map & / & map \\ \hline
InitSnowDepthMapFile & name of the file providing the initial condition on the snow depth map & / & map \\ \hline
InitSnowAgeMapFile & name of the file providing the initial condition on the snow age map & / & map \\ \hline
InitGlacierDepthMapFile & name of the file providing the initial condition on the glacier depth map & / & map \\ \hline
HeaderSoilInitPres & column name in the file SoilParFile for the initial total pressure head & SoilParFile & header \\ \hline
HeaderSoilInitTemp & column name in the file SoilParFile for the initial temperature & SoilParFile & header \\ \hline


\caption{Table of initial condition (character)}
\label{init3d_data}
\end{longtable}
\end{center}
\clearpage








\section{1D OUTPUT CHARACTER}


\clearpage


\begin{center}
\begin{longtable}{|p {7 cm}|p {7 cm}|p {3 cm}|p {4 cm}|}
\hline
\textbf{Keyword} & \textbf{Description} & \textbf{Associated file} & \textbf{type (file, header)} \\ \hline
\endfirsthead
\hline
\multicolumn{4}{| c |}{continued from previous page} \\
\hline
\textbf{Keyword} & \textbf{Description} & \textbf{Associated file} & \textbf{type (file, header)} \\ \hline
\endhead
\hline
\multicolumn{4}{| c |}{{continued on next page}}\\ 
\hline
\endfoot
\endlastfoot
\hline
SoilTempProfileFile & name of the output file providing the Soil/rock instantaneous temperature values at various depths & / & file \\ \hline
SoilTempProfileFileWriteEnd & name of the output file providing the Soil/rock instantaneous temperature values at various depths written just once at the end & / & file \\ \hline
SoilAveragedTempProfileFile & name of the output file providing the Soil/rock average (in DtPlotPoint) temperature values at various depths & / & file \\ \hline
SoilAveragedTempProfileFileWriteEnd & name of the output file providing the Soil/rock average (in DtPlotPoint) temperature values at various depths written just once at the end & / & file \\ \hline
SoilLiqWaterPressProfileFile & name of the output file providing the Soil/rock instantaneous liquid water pressure head values at various depths & / & file \\ \hline
SoilLiqWaterPressProfileFileWriteEnd & name of the output file providing the Soil/rock instantaneous liquid water pressure head values at various depths written just once at the end & / & file \\ \hline
SoilTotWaterPressProfileFile & name of the output file providing the Soil/rock instantaneous total (water+ice) pressure head values at various depths & / & file \\ \hline
SoilTotWaterPressProfileFileWriteEnd & name of the output file providing the Soil/rock instantaneous total (water+ice) pressure head values at various depths written just once at the end & / & file \\ \hline
SoilLiqContentProfileFile & name of the output file providing the Soil/rock instantaneous liquid water content values at various depths & / & file \\ \hline
SoilLiqContentProfileFileWriteEnd & name of the output file providing the Soil/rock instantaneous liquid water content values at various depths written just once at the end & / & file \\ \hline
SoilAveragedLiqContentProfileFile & name of the output file providing the Soil/rock average (in DtPlotPoint) liquid water content values at various depths & / & file \\ \hline
SoilAveragedLiqContentProfileFile WriteEnd & name of the output file providing the Soil/rock average (in DtPlotPoint) liquid water content values at various depths written just once at the end & / & file \\ \hline
SoilIceContentProfileFile & name of the output file providing the Soil/rock instantaneous ice content values at various depths & / & file \\ \hline
SoilIceContentProfileFileWriteEnd & name of the output file providing the Soil/rock instantaneous ice content values at various depths written just once at the end & / & file \\ \hline
SoilAveragedIceContentProfileFile & name of the output file providing the Soil/rock average (in DtPlotPoint) ice content values at various depths & / & file \\ \hline
SoilAveragedIceContentProfile FileWriteEnd & name of the output file providing the Soil/rock average (in DtPlotPoint) ice content values at various depths written just once at the end & / & file \\ \hline
HeaderDateSoil & column name in the file PointOutputFile for the variable Date &  & header \\ \hline
HeaderJulianDayFromYear0Soil & column name in the file PointOutputFile for the variable Julian Day from 0 &  & header \\ \hline
HeaderTimeFromStartSoil & column name in the file PointOutputFile for the variable Time from start &  & header \\ \hline
HeaderPeriodSoil & column name in the file PointOutputFile for the variable Simulation period &  & header \\ \hline
HeaderRunSoil & column name in the file PointOutputFile for the variable Run &  & header \\ \hline
HeaderIDPointSoil & column name in the file PointOutputFile for the variable IDPoint &  & header \\ \hline
HeaderThawedSoilDepthPoint & column name in the file PointOutputFile for the variable ThawedSoilDepthPoint & PointOutputFile & header \\ \hline
HeaderWaterTableDepthPoint & column name in the file PointOutputFile for the variable WaterTableDepthPoint & PointOutputFile & header \\ \hline
\caption{Table of meteorological parameters (character)}
\label{soil1d_data}
\end{longtable}
\end{center}
\clearpage







\begin{center}
\begin{longtable}{|p {7 cm}|p {7 cm}|p {3 cm}|p {4 cm}|}
\hline
\textbf{Keyword} & \textbf{Description} & \textbf{Associated file} & \textbf{type (file, header)} \\ \hline
\endfirsthead
\hline
\multicolumn{4}{| c |}{continued from previous page} \\
\hline
\textbf{Keyword} & \textbf{Description} & \textbf{Associated file} & \textbf{type (file, header)} \\ \hline
\endhead
\hline
\multicolumn{4}{| c |}{{continued on next page}}\\ 
\hline
\endfoot
\endlastfoot
\hline
SnowProfileFile & name of the output file providing the snow instantaneous values at various depths & / & file \\ \hline
SnowProfileFileWriteEnd & name of the output file providing the snow instantaneous values at various depths written just once at the end & / & file \\ \hline
SnowCoveredAreaFile & Name of the output file containing the percentage of the area covered by snow & / & file \\ \hline
HeaderDateSnow & column name in the file SnowProfileFile for the variable Date & SnowProfileFile & header \\ \hline
HeaderJulianDayFromYear0Snow & column name in the file SnowProfileFile for the variable Julian Day from 0 & SnowProfileFile & header \\ \hline
HeaderTimeFromStartSnow & column name in the file SnowProfileFile for the variable Time from start & SnowProfileFile & header \\ \hline
HeaderPeriodSnow & column name in the file SnowProfileFile for the variable Simulation period & SnowProfileFile & header \\ \hline
HeaderRunSnow & column name in the file SnowProfileFile for the variable Run & SnowProfileFile & header \\ \hline
HeaderIDPointSnow & column name in the file SnowProfileFile for the variable IDPoint & SnowProfileFile & header \\ \hline
HeaderTempSnow & column name in the file SnowProfileFile for the variable temperature & SnowProfileFile & header \\ \hline
HeaderIceContentSnow & column name in the file SnowProfileFile for the variable ice content & SnowProfileFile & header \\ \hline
HeaderWatContentSnow & column name in the file SnowProfileFile for the variable liquid content & SnowProfileFile & header \\ \hline
HeaderDepthSnow & column name in the file SnowProfileFile for the variable Depth & SnowProfileFile & header \\ \hline
HeaderPsnowNetPoint & column name in the file PointOutputFile for the variable PsnowNetPoint & PointOutputFile & header \\ \hline
HeaderSnowDepthPoint & column name in the file PointOutputFile for the variable SnowDepthPoint & PointOutputFile & header \\ \hline
HeaderSWEPoint & column name in the file PointOutputFile for the variable SWEPoint & PointOutputFile & header \\ \hline
HeaderSnowDensityPoint & column name in the file PointOutputFile for the variable SnowDensityPoint & PointOutputFile & header \\ \hline
HeaderSnowTempPoint & column name in the file PointOutputFile for the variable SnowTempPoint & PointOutputFile & header \\ \hline
HeaderSnowMeltedPoint & column name in the file PointOutputFile for the variable SnowMeltedPoint & PointOutputFile & header \\ \hline
HeaderSnowSublPoint & column name in the file PointOutputFile for the variable SnowSublPoint & PointOutputFile & header \\ \hline
HeaderSWEBlownPoint & column name in the file PointOutputFile for the variable SWEBlownPoint & PointOutputFile & header \\ \hline
HeaderSWESublBlownPoint & column name in the file PointOutputFile for the variable SWESublBlownPoint & PointOutputFile & header \\ \hline
\caption{Table of snow parameters (character)}
\label{snow1d_data}
\end{longtable}
\end{center}
\clearpage


\begin{center}
\begin{longtable}{|p {7 cm}|p {7 cm}|p {3 cm}|p {4 cm}|}
\hline
\textbf{Keyword} & \textbf{Description} & \textbf{Associated file} & \textbf{type (file, header)} \\ \hline
\endfirsthead
\hline
\multicolumn{4}{| c |}{continued from previous page} \\
\hline
\textbf{Keyword} & \textbf{Description} & \textbf{Associated file} & \textbf{type (file, header)} \\ \hline
\endhead
\hline
\multicolumn{4}{| c |}{{continued on next page}}\\ 
\hline
\endfoot
\endlastfoot
\hline
HeaderSurfaceEBPoint & column name in the file PointOutputFile for the variable SurfaceEBPoint & PointOutputFile & header \\ \hline
HeaderSoilHeatFluxPoint & column name in the file PointOutputFile for the variable SoilHeatFluxPoint & PointOutputFile & header \\ \hline
HeaderSWinPoint & column name in the file PointOutputFile for the variable SWinPoint & PointOutputFile & header \\ \hline
HeaderSWbeamPoint & column name in the file PointOutputFile for the variable SWbeamPoint & PointOutputFile & header \\ \hline
HeaderSWdiffPoint & column name in the file PointOutputFile for the variable SWdiffPoint & PointOutputFile & header \\ \hline
HeaderLWinPoint & column name in the file PointOutputFile for the variable LWinPoint & PointOutputFile & header \\ \hline
HeaderLWinMinPoint & column name in the file PointOutputFile for the variable LWinMinPoint & PointOutputFile & header \\ \hline
HeaderLWinMaxPoint & column name in the file PointOutputFile for the variable LWinMaxPoint & PointOutputFile & header \\ \hline
HeaderSWNetPoint & column name in the file PointOutputFile for the variable SWNetPoint & PointOutputFile & header \\ \hline
HeaderLWNetPoint & column name in the file PointOutputFile for the variable LWNetPoint & PointOutputFile & header \\ \hline
HeaderHPoint & column name in the file PointOutputFile for the variable HPoint & PointOutputFile & header \\ \hline
HeaderLEPoint & column name in the file PointOutputFile for the variable LEPoint & PointOutputFile & header \\ \hline
HeaderQSurfPoint & column name in the file PointOutputFile for the variable specific humidity near the soil surface & PointOutputFile & header \\ \hline
HeaderQAirPoint & column name in the file PointOutputFile for the variable specific humidity of the air & PointOutputFile & header \\ \hline
HeaderLObukhovPoint & column name in the file PointOutputFile for the variable LObukhovPoint & PointOutputFile & header \\ \hline
HeaderSWupPoint & column name in the file PointOutputFile for the variable SWupPoint & PointOutputFile & header \\ \hline
HeaderLWupPoint & column name in the file PointOutputFile for the variable LWupPoint & PointOutputFile & header \\ \hline
HeaderHupPoint & column name in the file PointOutputFile for the variable HupPoint & PointOutputFile & header \\ \hline
HeaderLEupPoint & column name in the file PointOutputFile for the variable LEupPoint & PointOutputFile & header \\ \hline
\caption{Table of surface energy flux parameters (character)}
\label{surfaceenergyflux1d_data}
\end{longtable}
\end{center}
\clearpage


\begin{center}
\begin{longtable}{|p {7 cm}|p {7 cm}|p {3 cm}|p {4 cm}|}
\hline
\textbf{Keyword} & \textbf{Description} & \textbf{Associated file} & \textbf{type (file, header)} \\ \hline
\endfirsthead
\hline
\multicolumn{4}{| c |}{continued from previous page} \\
\hline
\textbf{Keyword} & \textbf{Description} & \textbf{Associated file} & \textbf{type (file, header)} \\ \hline
\endhead
\hline
\multicolumn{4}{| c |}{{continued on next page}}\\ 
\hline
\endfoot
\endlastfoot
\hline
HeaderTvegPoint & column name in the file PointOutputFile for the variable TvegPoint & PointOutputFile & header \\ \hline
HeaderTCanopyAirPoint & column name in the file PointOutputFile for the variable TCanopyAirPoint & PointOutputFile & header \\ \hline
HeaderLSAIPoint & column name in the file PointOutputFile for the variable LSAIPoint & PointOutputFile & header \\ \hline
Headerz0vegPoint & column name in the file PointOutputFile for the variable z0vegPoint & PointOutputFile & header \\ \hline
Headerd0vegPoint & column name in the file PointOutputFile for the variable d0vegPoint & PointOutputFile & header \\ \hline
HeaderEstoredCanopyPoint & column name in the file PointOutputFile for the variable EstoredCanopyPoint & PointOutputFile & header \\ \hline
HeaderSWvPoint & column name in the file PointOutputFile for the variable SWvPoint & PointOutputFile & header \\ \hline
HeaderLWvPoint & column name in the file PointOutputFile for the variable LWvPoint & PointOutputFile & header \\ \hline
HeaderHvPoint & column name in the file PointOutputFile for the variable HvPoint & PointOutputFile & header \\ \hline
HeaderLEvPoint & column name in the file PointOutputFile for the variable LEvPoint & PointOutputFile & header \\ \hline
HeaderHgUnvegPoint & column name in the file PointOutputFile for the variable HgUnvegPoint & PointOutputFile & header \\ \hline
HeaderLEgUnvegPoint & column name in the file PointOutputFile for the variable LEgUnvegPoint & PointOutputFile & header \\ \hline
HeaderHgVegPoint & column name in the file PointOutputFile for the variable HgVegPoint & PointOutputFile & header \\ \hline
HeaderLEgVegPoint & column name in the file PointOutputFile for the variable LEgVegPoint & PointOutputFile & header \\ \hline
HeaderEvapSurfacePoint & column name in the file PointOutputFile for the variable EvapSurfacePoint & PointOutputFile & header \\ \hline
HeaderTraspCanopyPoint & column name in the file PointOutputFile for the variable TraspCanopyPoint & PointOutputFile & header \\ \hline
HeaderWaterOnCanopyPoint & column name in the file PointOutputFile for the variable WaterOnCanopyPoint & PointOutputFile & header \\ \hline
HeaderSnowOnCanopyPoint & column name in the file PointOutputFile for the variable SnowOnCanopyPoint & PointOutputFile & header \\ \hline
HeaderQVegPoint & column name in the file PointOutputFile for the variable specific humidity near the vegetation & PointOutputFile & header \\ \hline
HeaderLObukhovCanopyPoint & column name in the file PointOutputFile for the variable LObukhovCanopyPoint & PointOutputFile & header \\ \hline
HeaderWindSpeedTopCanopyPoint & column name in the file PointOutputFile for the variable WindSpeedTopCanopyPoint & PointOutputFile & header \\ \hline
HeaderDecayKCanopyPoint & column name in the file PointOutputFile for the variable DecayKCanopyPoint & PointOutputFile & header \\ \hline
\caption{Table of vegetation parameters (character)}
\label{vegetation1d_data}
\end{longtable}
\end{center}
\clearpage



\section{3D OUTPUT CHARACTER}

\begin{center}
\begin{longtable}{|p {7 cm}|p {7 cm}|p {3 cm}|p {4 cm}|}
\hline
\textbf{Keyword} & \textbf{Description} & \textbf{Associated file} & \textbf{type (file, header)} \\ \hline
\endfirsthead
\hline
\multicolumn{4}{| c |}{continued from previous page} \\
\hline
\textbf{Keyword} & \textbf{Description} & \textbf{Associated file} & \textbf{type (file, header)} \\ \hline
\endhead
\hline
\multicolumn{4}{| c |}{{continued on next page}}\\ 
\hline
\endfoot
\endlastfoot
\hline
SuccessfulRunFile & column name of the file that summarizes if the simulation has arrived to the end & / & file \\ \hline
FailedRunFile & column name of the file that summarizes if the simulation has failed & / & file \\ \hline
PointOutputFile & name of the output file providing the Point values & / & file \\ \hline
PointOutputFileWriteEnd & name of the output file providing the Point values written just once at the end & / & file \\ \hline
HeaderDatePoint & column name in the file PointOutputFile for the variable DatePoint & PointOutputFile & header \\ \hline
HeaderJulianDayFromYear0Point & column name in the file PointOutputFile for the variable JulianDayFromYear0Point & PointOutputFile & header \\ \hline
HeaderTimeFromStartPoint & column name in the file PointOutputFile for the variable TimeFromStartPoint & PointOutputFile & header \\ \hline
HeaderPeriodPoint & column name in the file PointOutputFile for the variable PeriodPoint & PointOutputFile & header \\ \hline
HeaderRunPoint & column name in the file PointOutputFile for the variable RunPoint & PointOutputFile & header \\ \hline
HeaderIDPointPoint & column name in the file PointOutputFile for the variable IDPointPoint & PointOutputFile & header \\ \hline
HeaderCanopyFractionPoint & column name in the file PointOutputFile for the variable CanopyFractionPoint & PointOutputFile & header \\ \hline
\caption{Table of general parameters (character)}
\label{general1d_data}
\end{longtable}
\end{center}
\clearpage



\begin{center}
\begin{longtable}{|p {7 cm}|p {7 cm}|p {3 cm}|p {4 cm}|}
\hline
\textbf{Keyword} & \textbf{Description} & \textbf{Associated file} & \textbf{type (file, header)} \\ \hline
\endfirsthead
\hline
\multicolumn{4}{| c |}{continued from previous page} \\
\hline
\textbf{Keyword} & \textbf{Description} & \textbf{Associated file} & \textbf{type (file, header)} \\ \hline
\endhead
\hline
\multicolumn{4}{| c |}{{continued on next page}}\\ 
\hline
\endfoot
\endlastfoot
\hline
DischargeFile & name of the output file providing the discharge values & / & file \\ \hline
\caption{Table of channel flow parameters (character)}
\label{channelflow3d_data}
\end{longtable}
\end{center}
\clearpage



\begin{center}
\begin{longtable}{|p {7 cm}|p {7 cm}|p {3 cm}|p {4 cm}|}
\hline
\textbf{Keyword} & \textbf{Description} & \textbf{Associated file} & \textbf{type (file, header)} \\ \hline
\endfirsthead
\hline
\multicolumn{4}{| c |}{continued from previous page} \\
\hline
\textbf{Keyword} & \textbf{Description} & \textbf{Associated file} & \textbf{type (file, header)} \\ \hline
\endhead
\hline
\multicolumn{4}{| c |}{{continued on next page}}\\ 
\hline
\endfoot
\endlastfoot
\hline
SnowDepthMapFile & name of the output file providing the Snow depth map & / & map \\ \hline
SnowMeltedMapFile & name of the output file providing the Snow melted map & / & map \\ \hline
SnowSublMapFile & name of the output file providing the Snow sublimated map & / & map \\ \hline
SWEMapFile & name of the output file providing the Snow water equivalent (SWE) map & / & map \\ \hline
AveragedSnowDepthMapFile & name of the output file providing the Average snow depth map & / & map \\ \hline
SpecificPlotSnowDepthMapFile & name of the output file providing the snow depth map at high temporal resolution during specific days & / & map \\ \hline
SnowProfileFile & name of the output file providing the snow instantaneous values at various depths & / & file \\ \hline
SnowProfileFileWriteEnd & name of the output file providing the snow instantaneous values at various depths written just once at the end & / & file \\ \hline
SnowCoveredAreaFile & Name of the output file containing the percentage of the area covered by snow & / & file \\ \hline
HeaderDateSnow & column name in the file SnowProfileFile for the variable Date & SnowProfileFile & header \\ \hline
HeaderJulianDayFromYear0Snow & column name in the file SnowProfileFile for the variable Julian Day from 0 & SnowProfileFile & header \\ \hline
HeaderTimeFromStartSnow & column name in the file SnowProfileFile for the variable Time from start & SnowProfileFile & header \\ \hline
HeaderPeriodSnow & column name in the file SnowProfileFile for the variable Simulation period & SnowProfileFile & header \\ \hline
HeaderRunSnow & column name in the file SnowProfileFile for the variable Run & SnowProfileFile & header \\ \hline
HeaderIDPointSnow & column name in the file SnowProfileFile for the variable IDPoint & SnowProfileFile & header \\ \hline
HeaderTempSnow & column name in the file SnowProfileFile for the variable temperature & SnowProfileFile & header \\ \hline
HeaderIceContentSnow & column name in the file SnowProfileFile for the variable ice content & SnowProfileFile & header \\ \hline
HeaderWatContentSnow & column name in the file SnowProfileFile for the variable liquid content & SnowProfileFile & header \\ \hline
HeaderDepthSnow & column name in the file SnowProfileFile for the variable Depth & SnowProfileFile & header \\ \hline
HeaderPsnowNetPoint & column name in the file PointOutputFile for the variable PsnowNetPoint & PointOutputFile & header \\ \hline
HeaderSnowDepthPoint & column name in the file PointOutputFile for the variable SnowDepthPoint & PointOutputFile & header \\ \hline
HeaderSWEPoint & column name in the file PointOutputFile for the variable SWEPoint & PointOutputFile & header \\ \hline
HeaderSnowDensityPoint & column name in the file PointOutputFile for the variable SnowDensityPoint & PointOutputFile & header \\ \hline
HeaderSnowTempPoint & column name in the file PointOutputFile for the variable SnowTempPoint & PointOutputFile & header \\ \hline
HeaderSnowMeltedPoint & column name in the file PointOutputFile for the variable SnowMeltedPoint & PointOutputFile & header \\ \hline
HeaderSnowSublPoint & column name in the file PointOutputFile for the variable SnowSublPoint & PointOutputFile & header \\ \hline
HeaderSWEBlownPoint & column name in the file PointOutputFile for the variable SWEBlownPoint & PointOutputFile & header \\ \hline
HeaderSWESublBlownPoint & column name in the file PointOutputFile for the variable SWESublBlownPoint & PointOutputFile & header \\ \hline
\caption{Table of snow parameters (character)}
\label{snow3d_data}
\end{longtable}
\end{center}
\clearpage






\begin{center}
\begin{longtable}{|p {7 cm}|p {7 cm}|p {3 cm}|p {4 cm}|}
\hline
\textbf{Keyword} & \textbf{Description} & \textbf{Associated file} & \textbf{type (file, header)} \\ \hline
\endfirsthead
\hline
\multicolumn{4}{| c |}{continued from previous page} \\
\hline
\textbf{Keyword} & \textbf{Description} & \textbf{Associated file} & \textbf{type (file, header)} \\ \hline
\endhead
\hline
\multicolumn{4}{| c |}{{continued on next page}}\\ 
\hline
\endfoot
\endlastfoot
\hline
CanopyInterceptedWaterMapFile & name of the output file providing the canopy intercepted water map & / & map \\ \hline
SpecificPlotVegSensibleHeatFluxMapFile & name of the output file providing the vegetation sensible heat flux map at high temporal resolution during specific days & / & map \\ \hline
SpecificPlotVegLatentHeatFluxMapFile & name of the output file providing the vegetation latent heat flux map at high temporal resolution during specific days & / & map \\ \hline
SpecificPlotNetVegShortwaveRadMapFile & name of the output file providing the vegetation Swnet flux map at high temporal resolution during specific days & / & map \\ \hline
SpecificPlotNetVegLongwaveRadMapFile & name of the output file providing the vegetation Lwnet map at high temporal resolution during specific days & / & map \\ \hline
SpecificPlotCanopyAirTempMapFile & name of the output file providing the canopy air temperature map at high temporal resolution during specific days & / & map \\ \hline
SpecificPlotVegTempMapFile & name of the output file providing the vegetation temperature map at high temporal resolution during specific days & / & map \\ \hline
SpecificPlotAboveVegAirTempMapFile & name of the output file providing the above vegetation air temperature map at high temporal resolution during specific days & / & map \\ 
HeaderTvegPoint & column name in the file PointOutputFile for the variable TvegPoint & PointOutputFile & header \\ \hline
HeaderTCanopyAirPoint & column name in the file PointOutputFile for the variable TCanopyAirPoint & PointOutputFile & header \\ \hline
HeaderLSAIPoint & column name in the file PointOutputFile for the variable LSAIPoint & PointOutputFile & header \\ \hline
Headerz0vegPoint & column name in the file PointOutputFile for the variable z0vegPoint & PointOutputFile & header \\ \hline
Headerd0vegPoint & column name in the file PointOutputFile for the variable d0vegPoint & PointOutputFile & header \\ \hline
HeaderEstoredCanopyPoint & column name in the file PointOutputFile for the variable EstoredCanopyPoint & PointOutputFile & header \\ \hline
HeaderSWvPoint & column name in the file PointOutputFile for the variable SWvPoint & PointOutputFile & header \\ \hline
HeaderLWvPoint & column name in the file PointOutputFile for the variable LWvPoint & PointOutputFile & header \\ \hline
HeaderHvPoint & column name in the file PointOutputFile for the variable HvPoint & PointOutputFile & header \\ \hline
HeaderLEvPoint & column name in the file PointOutputFile for the variable LEvPoint & PointOutputFile & header \\ \hline
HeaderHgUnvegPoint & column name in the file PointOutputFile for the variable HgUnvegPoint & PointOutputFile & header \\ \hline
HeaderLEgUnvegPoint & column name in the file PointOutputFile for the variable LEgUnvegPoint & PointOutputFile & header \\ \hline
HeaderHgVegPoint & column name in the file PointOutputFile for the variable HgVegPoint & PointOutputFile & header \\ \hline
HeaderLEgVegPoint & column name in the file PointOutputFile for the variable LEgVegPoint & PointOutputFile & header \\ \hline
HeaderEvapSurfacePoint & column name in the file PointOutputFile for the variable EvapSurfacePoint & PointOutputFile & header \\ \hline
HeaderTraspCanopyPoint & column name in the file PointOutputFile for the variable TraspCanopyPoint & PointOutputFile & header \\ \hline
HeaderWaterOnCanopyPoint & column name in the file PointOutputFile for the variable WaterOnCanopyPoint & PointOutputFile & header \\ \hline
HeaderSnowOnCanopyPoint & column name in the file PointOutputFile for the variable SnowOnCanopyPoint & PointOutputFile & header \\ \hline
HeaderQVegPoint & column name in the file PointOutputFile for the variable specific humidity near the vegetation & PointOutputFile & header \\ \hline
HeaderLObukhovCanopyPoint & column name in the file PointOutputFile for the variable LObukhovCanopyPoint & PointOutputFile & header \\ \hline
HeaderWindSpeedTopCanopyPoint & column name in the file PointOutputFile for the variable WindSpeedTopCanopyPoint & PointOutputFile & header \\ \hline
HeaderDecayKCanopyPoint & column name in the file PointOutputFile for the variable DecayKCanopyPoint & PointOutputFile & header \\ \hline
\caption{Table of vegetation parameters (character)}
\label{vegetation3d_data}
\end{longtable}
\end{center}
\clearpage

\section{RECOVERY 3D CHARACTER}

\begin{center}
\begin{longtable}{|p {7 cm}|p {7 cm}|p {3 cm}|p {4 cm}|}
\hline
\textbf{Keyword} & \textbf{Description} & \textbf{Associated file} & \textbf{type (file, header)} \\ \hline
\endfirsthead
\hline
\multicolumn{4}{| c |}{continued from previous page} \\
\hline
\textbf{Keyword} & \textbf{Description} & \textbf{Associated file} & \textbf{type (file, header)} \\ \hline
\endhead
\hline
\multicolumn{4}{| c |}{{continued on next page}}\\ 
\hline
\endfoot
\endlastfoot
\hline
RecoverSoilWatPresChannel & name of the recovery file of SoiWatPresChannel & / & file  \\ \hline
RecoverSoilIceContChannel & name of the recovery file of SoiIceContChannel & / & file  \\ \hline
RecoverSoilTempChannel & name of the recovery file of SoilTempChannel & / & file  \\ \hline
\caption{Table of recovery parameters for channel flow (character)}
\label{recoverychannelflow_data}
\end{longtable}
\end{center}
\clearpage

\begin{center}
\begin{longtable}{|p {7 cm}|p {7 cm}|p {3 cm}|p {4 cm}|}
\hline
\textbf{Keyword} & \textbf{Description} & \textbf{Associated file} & \textbf{type (file, header)} \\ \hline
\endfirsthead
\hline
\multicolumn{4}{| c |}{continued from previous page} \\
\hline
\textbf{Keyword} & \textbf{Description} & \textbf{Associated file} & \textbf{type (file, header)} \\ \hline
\endhead
\multicolumn{4}{| c |}{{continued on next page}}\\ 
\hline
\endfoot
\endlastfoot
\hline
RecoverGlacierLayerThick & name of the recovery file of GlacierLayerThick & / & file  \\ \hline
RecoverGlacierLiqMass & name of the recovery file of GlacieLiqMass & / & file \\ \hline
RecoverGlacierIceMass & name of the recovery file of GlacieIceMass & / & file  \\ \hline
RecoverGlacierTemp & name of the recovery file of GlacieTemp & / & file  \\ \hline
RecoverGlacierLayerNumber & name of the recovery file of GacierLayerNumber & / & file  \\ \hline
\caption{Table of recovery parameters for glacier (character)}
\label{recoverychannelflow_data}
\end{longtable}
\end{center}
\clearpage

\begin{center}
\begin{longtable}{|p {7 cm}|p {7 cm}|p {3 cm}|p {4 cm}|}
\hline
\textbf{Keyword} & \textbf{Description} & \textbf{Associated file} & \textbf{type (file, header)} \\ \hline
\endfirsthead
\hline
\multicolumn{4}{| c |}{continued from previous page} \\
\hline
\textbf{Keyword} & \textbf{Description} & \textbf{Associated file} & \textbf{type (file, header)} \\ \hline
\endhead
\hline
\multicolumn{4}{| c |}{{continued on next page}}\\ 
\hline
\endfoot
\endlastfoot
\hline
RecoverLandSurfaceWaterDepth & name of the recovery file of LandSurfaceWaterDepth & / & file \\ \hline
\caption{Table of recovery parameters for runoff (character)}
\label{recoveryrunoff_data}
\end{longtable}
\end{center}
\clearpage

\begin{center}
\begin{longtable}{|p {7 cm}|p {7 cm}|p {3 cm}|p {4 cm}|}
\hline
\textbf{Keyword} & \textbf{Description} & \textbf{Associated file} & \textbf{type (file, header)} \\ \hline
\endfirsthead
\hline
\multicolumn{4}{| c |}{continued from previous page} \\
\hline
\textbf{Keyword} & \textbf{Description} & \textbf{Associated file} & \textbf{type (file, header)} \\ \hline
\endhead
\hline
\multicolumn{4}{| c |}{{continued on next page}}\\ 
\hline
\endfoot
\endlastfoot
\hline
RecoverSnowLiqMass & name of the recovery file of SnowLiqMass & / & file \\ \hline
RecoverSnowIceMass & name of the recovery file of SnowIceMass & / & file \\ \hline
RecoverSnowTemp & name of the recovery file of SnowTemp & / & file \\ \hline
RecoverSnowLayerNumber & name of the recovery file of SnowLayerNumber & / & file \\ \hline
RecoverNonDimensionalSnowAge & name of the recovery file of NonDimensionalSnowAge & / & file \\ \hline
RecoverDimensionalSnowAge & name of the recovery file of DimensionalSnowAge & / & file \\ \hline
\caption{Table of recovery parameters for snow (character)}
\label{recoverysnow_data}
\end{longtable}
\end{center}
\clearpage

\begin{center}
\begin{longtable}{|p {7 cm}|p {7 cm}|p {3 cm}|p {4 cm}|}
\hline
\textbf{Keyword} & \textbf{Description} & \textbf{Associated file} & \textbf{type (file, header)} \\ \hline
\endfirsthead
\hline
\multicolumn{4}{| c |}{continued from previous page} \\
\hline
\textbf{Keyword} & \textbf{Description} & \textbf{Associated file} & \textbf{type (file, header)} \\ \hline
\endhead
\hline
\multicolumn{4}{| c |}{{continued on next page}}\\ 
\hline
\endfoot
\endlastfoot
\hline
RecoverSoilWatPres & name of the recovery file of SoilWatPres & / & file \\ \hline
RecoverSoilIceCont & name of the recovery file of SoilIceCont & / & file \\ \hline
RecoverSoilTemp & name of the recovery file of SoilTemp & / & file \\ \hline
\caption{Table of recovery parameters for soil (character)}
\label{recoverysoil_data}
\end{longtable}
\end{center}
\clearpage


\begin{center}
\begin{longtable}{|p {7 cm}|p {7 cm}|p {3 cm}|p {4 cm}|}
\hline
\textbf{Keyword} & \textbf{Description} & \textbf{Associated file} & \textbf{type (file, header)} \\ \hline
\endfirsthead
\hline
\multicolumn{4}{| c |}{continued from previous page} \\
\hline
\textbf{Keyword} & \textbf{Description} & \textbf{Associated file} & \textbf{type (file, header)} \\ \hline
\endhead
\hline
\multicolumn{4}{| c |}{{continued on next page}}\\ 
\hline
\endfoot
\endlastfoot
\hline
RecoverLiqWaterOnCanopy & name of the recovery file of LiqWaterOnCanopy & / & file \\ \hline
RecoverSnowOnCanopy & name of the recovery file of SnowOnCanopy & / & file \\ \hline
RecoverVegTemp & name of the recovery file of VegetationTemperature & / & file \\ \hline
\caption{Table of recovery parameters for vegetation (character)}
\label{recoveryvegetation_data}
\end{longtable}
\end{center}
\clearpage


\end{document}  