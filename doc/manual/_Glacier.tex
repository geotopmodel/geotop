\chapter{Glacier}


\section{Input}

\subsection{Parameters}

\begin{center}
\begin{longtable}{|p {4. cm}|p {4. cm}|p {1. cm}|p{0.6 cm}|p{1.4 cm}|p{0.8 cm}|p{1.3 cm}|}
\hline
\textbf{Keyword} & \textbf{Description} & \textbf{M. U.} & \textbf{range} & \textbf{Default Value} & \textbf{Sca / Vec} & \textbf{Str / Num / Opt} \\ \hline
\endfirsthead
\hline
\multicolumn{7}{| c |}{continued from previous page} \\
\hline
\textbf{Keyword} & \textbf{Description} & \textbf{M. U.} & \textbf{range} & \textbf{Default Value} & \textbf{Sca / Vec} & \textbf{Str / Num / Opt} \\ \hline
\endhead
\hline
\multicolumn{7}{| c |}{{continued on next page}}\\ 
\hline
\endfoot
\endlastfoot
\hline
IrriducibleWatSatGlacier \index{IrriducibleWatSatGlacier} & irreducible water saturation for glacier & - &  & 0.02 & sca & num \\ \hline
MaxWaterEqGlacLayerContent \index{MaxWaterEqGlacLayerContent} & maximum water equivalent admitted in a snow layer &  &  & 5 & sca & num \\ \hline
MaxGlacLayerNumber \index{MaxGlacLayerNumber} & maximum layers of snow to use (suggested $>$5) &  &  & 0 & sca & num \\ \hline
ThickerGlacLayers \index{ThickerGlacLayers} & Layer numbers that can become thicker than admitted by the threshold given by MaxGlacLayerNumber (from the bottom up). They can be more than one &  &  & Max Glac Layer Number/2 & vec & num \\ \hline

\caption{Keywords of glacier input parametrs configurable in geotop.inpts file.}
\label{glacier_parameters}
\end{longtable}
\end{center}


\section{Output}



\subsection{Point output}

\subsubsection{Files}

\begin{center}
\begin{longtable}{|p {3.3 cm}|p {9 cm}|p {2 cm}|p {1 cm}|}
\hline
\textbf{Keyword} & \textbf{Description}  \\ \hline
\endfirsthead
\hline
\multicolumn{4}{| c |}{continued from previous page} \\
\hline
\textbf{Keyword} & \textbf{Description}   \\ \hline
\endhead
\hline
\multicolumn{4}{| c |}{{continued on next page}}\\ 
\hline
\endfoot
\endlastfoot
\hline
GlacierProfileFile & name of the output file providing the glacier instantaneous values at various depths  \\ \hline
GlacierProfileFileWriteEnd & name of the output file providing the glacier instantaneous values at various depths written just once at the end  \\ \hline
PointOutputFile & name of the file providing the properties for the simulation point \\ \hline
PointOutputFileWriteEnd & name of the output file providing the Point values written just once at the end \\ \hline
\caption{Keywords of file related to glacier}
\label{glacierpointoutput_file}
\end{longtable}
\end{center}


\subsubsection{Headers}

\begin{center}
\begin{longtable}{|p {4.2 cm}|p {6.5 cm}|p {2.5 cm}|}
\hline
\textbf{Keyword} & \textbf{Description} & \textbf{Associated file}  \\ \hline
\endfirsthead
\hline
\multicolumn{3}{| c |}{continued from previous page} \\
\hline
\textbf{Keyword} & \textbf{Description} & \textbf{Associated file}  \\ \hline
\endhead
\hline
\multicolumn{3}{| c |}{{continued on next page}}\\ 
\hline
\endfoot
\endlastfoot
\hline
HeaderDateGlac \index{HeaderDateGlac} & column name in the file GlacierProfileFile for the variable Date & GlacierProfileFile \\ \hline
HeaderJulianDayFromYear0Glac \index{HeaderJulianDayFromYear0Glac} & column name in the file GlacierProfileFile for the variable Julian Day from 0 & GlacierProfileFile  \\ \hline
HeaderTimeFromStartGlac \index{HeaderTimeFromStartGlac} & column name in the file GlacierProfileFile for the variable Time from start & GlacierProfileFile  \\ \hline
HeaderPeriodGlac \index{HeaderPeriodGlac} & column name in the file GlacierProfileFile for the variable Simulation period & GlacierProfileFile  \\ \hline
HeaderRunGlac \index{HeaderRunGlac} & column name in the file GlacierProfileFile for the variable Run & GlacierProfileFile  \\ \hline
HeaderIDPointGlac \index{HeaderIDPointGlac} & column name in the file GlacierProfileFile for the variable IDPoint & GlacierProfileFile  \\ \hline
HeaderTempGlac \index{HeaderTempGlac} & column name in the file GlacierProfileFile for the variable temperature & GlacierProfileFile  \\ \hline
HeaderIceContentGlac \index{HeaderIceContentGlac} & column name in the file GlacierProfileFile for the variable ice content & GlacierProfileFile  \\ \hline
HeaderWatContentGlac \index{HeaderWatContentGlac} & column name in the file GlacierProfileFile for the variable liquid content & GlacierProfileFile  \\ \hline
HeaderDepthGlac \index{HeaderDepthGlac} & column name in the file GlacierProfileFile for the variable Depth & GlacierProfileFile  \\ \hline
\caption{Keywords of the personalized header for the file GlacierProfileFile}
\label{glacierheaders_data}
\end{longtable}
\end{center}


\begin{center}
\begin{longtable}{|p {3.9 cm}|p {7 cm}|p {2.5 cm}|}
\hline
\textbf{Keyword} & \textbf{Description} & \textbf{Associated file}  \\ \hline
\endfirsthead
\hline
\multicolumn{3}{| c |}{continued from previous page} \\
\hline
\textbf{Keyword} & \textbf{Description} & \textbf{Associated file}  \\ \hline
\endhead
\hline
\multicolumn{3}{| c |}{{continued on next page}}\\ 
\hline
\endfoot
\endlastfoot
\hline
HeaderGlacDepthPoint \index{HeaderGlacDepthPoint} & column name in the file PointOutputFile for the variable GlacDepthPoint & PointOutputFile  \\ \hline
HeaderGWEPoint \index{HeaderGWEPoint} & column name in the file PointOutputFile for the variable GWEPoint & PointOutputFile  \\ \hline
HeaderGlacDensityPoint \index{HeaderGlacDensityPoint} & column name in the file PointOutputFile for the variable GlacDensityPoint & PointOutputFile  \\ \hline
HeaderGlacTempPoint \index{HeaderGlacTempPoint} & column name in the file PointOutputFile for the variable GlacTempPoint & PointOutputFile  \\ \hline
HeaderGlacMeltedPoint \index{HeaderGlacMeltedPoint} & column name in the file PointOutputFile for the variable GlacMeltedPoint & PointOutputFile  \\ \hline
HeaderGlacSublPoint \index{HeaderGlacSublPoint} & column name in the file PointOutputFile for the variable GlacSublPoint & PointOutputFile  \\ \hline
\caption{Keywords of the personalized header for the file PointOutputFile}
\label{glacierheaderpoint_data}
\end{longtable}
\end{center}



\subsubsection{Parameters}

\begin{center}
\begin{longtable}{|p {3.4 cm}|p {4.7 cm}|p {1. cm}|p{0.8 cm}|p{1.4 cm}|p{0.8 cm}|p{1.3 cm}|}
\hline
\textbf{Keyword} & \textbf{Description} & \textbf{M. U.} & \textbf{range} & \textbf{Default Value} & \textbf{Sca / Vec} & \textbf{Str / Num / Opt} \\ \hline
\endfirsthead
\hline
\multicolumn{7}{| c |}{continued from previous page} \\
\hline
\textbf{Keyword} & \textbf{Description} & \textbf{M. U.} & \textbf{range} & \textbf{Default Value} & \textbf{Sca / Vec} & \textbf{Str / Num / Opt} \\ \hline
\endhead
\hline
\multicolumn{7}{| c |}{{continued on next page}}\\ 
\hline
\endfoot
\endlastfoot
\hline
DefaultGlac \index{DefaultGlac} & 0: use personal setting, 1:use default & - & 0, 1 & 1 & sca & opt \\ \hline
GlacPlotDepths \index{GlacPlotDepths} & depths of the glacier where one wants to write the results & - &  & NA & vec & num \\ \hline
DateGlac \index{DateGlac} & column number in which one would like to visualize the Date12 [DDMMYYYYhhmm]    	 & - &  & -1 & sca & num \\ \hline
JulianDayFromYear0Glac \index{JulianDayFromYear0Glac} & column number in which one would like to visualize the JulianDayFromYear0[days]   	 & - &  & -1 & sca & num \\ \hline
TimeFromStartGlac \index{TimeFromStartGlac} & column in which one would like to visualize the TimeFromStart[days]     & - &  & -1 & sca & num \\ \hline
PeriodGlac \index{PeriodGlac} & Column number to write the period number & - &  & -1 & sca & num \\ \hline
RunGlac \index{RunGlac} & Column number to write the run number & - &  & -1 & sca & num \\ \hline
IDPointGlac \index{IDPointGlac} & column number in which one would like to visualize the IDpoint  & - &  & -1 & sca & num \\ \hline
WaterEquivalentGlac \index{WaterEquivalentGlac} & column number in which one would like the water equivalent of the glacier & - &  & -1 & sca & num \\ \hline
DepthGlac \index{DepthGlac} & column number in which one would like to visualize the depth of the glacier & - &  & -1 & sca & num \\ \hline
DensityGlac \index{DensityGlac} & column number in which one would like to visualize the density of the glacier & - &  & -1 & sca & num \\ \hline
TempGlac \index{TempGlac} & column number in which one would like to visualize the temperature of the glacier  & - &  & -1 & sca & num \\ \hline
IceContentGlac \index{IceContentGlac} & column number in which one would like to visualize the ice content of the glacier  & - &  & -1 & sca & num \\ \hline
WatContentGlac \index{WatContentGlac} & column number in which one would like to visualize the water content of the glacier  & - &  & -1 & sca & num \\ \hline
\caption{Keywords defining the column number where printing the desired variable in the GlacierProfileFile}
\label{glaciercolumn_numeric}
\end{longtable}
\end{center}

\begin{center}
\begin{longtable}{|p {3.2 cm}|p {5.3 cm}|p {1 cm}|p{1. cm}|p{1.1 cm}|p{1. cm}|p{1 cm}|}
\hline
\textbf{Keyword} & \textbf{Description} & \textbf{M. U.} & \textbf{range} & \textbf{Default Value} & \textbf{Sca / Vec} & \textbf{Log / Num} \\ \hline
\endfirsthead
\hline
\multicolumn{7}{| c |}{continued from previous page} \\
\hline
\textbf{Keyword} & \textbf{Description} & \textbf{M. U.} & \textbf{range} & \textbf{Default Value} & \textbf{Sca / Vec} & \textbf{Log / Num} \\ \hline
\endhead
\hline
\multicolumn{7}{| c |}{{continued on next page}}\\ 
\hline
\endfoot
\endlastfoot
\hline
DtPlotPoint \index{DtPlotPoint} & Plotting Time step (in hour) of the output for specified pixels (0 means that it is not plotted) & h & 0, inf & 0 & vec & num \\ \hline
DatePoint \index{DatePoint} & column number in which one would like to visualize the Date12 [DDMMYYYY hhmm]    	 & - & 1, 76 & -1 & sca & num \\ \hline
JulianDayFromYear0Point \index{JulianDayFromYear0Point} & column number in which one would like to visualize the JulianDayFromYear0[days]   	 & - & 1, 76 & -1 & sca & num \\ \hline
TimeFromStartPoint \index{TimeFromStartPoint} & column number in which one would like to visualize the TimeFromStart[days]  & - & 1, 76 & -1 & sca & num \\ \hline
PeriodPoint \index{PeriodPoint} & column number in which one would like to visualize the Simulation\_Period & - & 1, 76 & -1 & sca & num \\ \hline
RunPoint \index{RunPoint} & column number in which one would like to visualize the Run	 & - & 1, 76 & -1 & sca & num \\ \hline
IDPointPoint \index{IDPointPoint} & column number in which one would like to visualize the IDpoint  & - & 1, 76 & -1 & sca & num \\ \hline
GlacDepthPoint \index{GlacDepthPoint} & column number in which one would like to visualize the glacier depth [mm]  & - & 1, 76 & -1 & sca & num \\ \hline
GWEPoint \index{GWEPoint} & column number in which one would like to visualize the glacier water equivalent [mm]  & - & 1, 76 & -1 & sca & num \\ \hline
GlacDensityPoint \index{GlacDensityPoint} & column number in which one would like to visualize the glacier density [kg m$^{-3}$]  & - & 1, 76 & -1 & sca & num \\ \hline
GlacTempPoint \index{GlacTempPoint} & column number in which one would like to visualize the glacier temperature [\textcelsius]  & - & 1, 76 & -1 & sca & num \\ \hline
GlacMeltedPoint \index{GlacMeltedPoint} & column number in which one would like to visualize the glac\_melted [mm]  & - & 1, 76 & -1 & sca & num \\ \hline
GlacSublPoint  \index{GlacSublPoint} & column number in which one would like to visualize the glacier sublimated depth [mm]  & - & 1, 76 & -1 & sca & num \\ \hline
\caption{Keywords defining the column number where to print the desired variable in the PointOutputFile}
\label{glaciercolumnpoint_numeric}
\end{longtable}
\end{center}

\subsection{Map Output}

\subsubsection{Parameters}

\begin{center}
\begin{longtable}{|p {3.4 cm}|p {4.7 cm}|p {1. cm}|p{0.8 cm}|p{1.4 cm}|p{0.8 cm}|p{1.3 cm}|}
\hline
\textbf{Keyword} & \textbf{Description} & \textbf{M. U.} & \textbf{range} & \textbf{Default Value} & \textbf{Sca / Vec} & \textbf{Str / Num / Opt} \\ \hline
\endfirsthead
\hline
\multicolumn{7}{| c |}{continued from previous page} \\
\hline
\textbf{Keyword} & \textbf{Description} & \textbf{M. U.} & \textbf{range} & \textbf{Default Value} & \textbf{Sca / Vec} & \textbf{Str / Num / Opt} \\ \hline
\endhead
\hline
\multicolumn{7}{| c |}{{continued on next page}}\\ 
\hline
\endfoot
\endlastfoot
\hline
DefaultGlac \index{DefaultGlac} & 0: use personal setting, 1:use default & - & 0, 1 & 1 & sca & opt \\ \hline
GlacPlotDepths \index{GlacPlotDepths} & depths of the glacier where one wants to write the results & - &  & NA & vec & num \\ \hline
OutputGlacierMaps \index{OutputGlacierMaps} & frequency (h) of printing of the results of the glacier maps & h &  & 0 & sca & num \\ \hline
\caption{Keywords of frequency for printing glacier output maps}
\label{glac_numeric}
\end{longtable}
\end{center}















