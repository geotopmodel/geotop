\chapter{Soil/Rock Infiltration}



\section{Input}

\subsection{File}
\begin{center}
\begin{longtable}{|p {4.2 cm}|p {6 cm}|p {2 cm}|p {1.5 cm}|}
\hline
\textbf{Keyword} & \textbf{Description} & \textbf{Associated file} & \textbf{type (file, header)} \\ \hline
\endfirsthead
\hline
\multicolumn{4}{| c |}{continued from previous page} \\
\hline
\textbf{Keyword} & \textbf{Description} & \textbf{Associated file} & \textbf{type (file, header)} \\ \hline
\endhead
\hline
\multicolumn{4}{| c |}{{continued on next page}}\\ 
\hline
\endfoot
\endlastfoot
\hline
SoilParFile \index{SoilParFile} & name of the file providing the soil parameters & / & file \\ \hline
\caption{Keywords of file related to soil and rock parameters}
\label{soil_key}
\end{longtable}
\end{center}


\subsection{Headers}

\begin{center}
\begin{longtable}{|p {4.1 cm}|p {8.5 cm}|p {2 cm}|}
\hline
\textbf{Keyword} & \textbf{Description} & \textbf{Associated file}  \\ \hline
\endfirsthead
\hline
\multicolumn{3}{| c |}{continued from previous page} \\
\hline
\textbf{Keyword} & \textbf{Description} & \textbf{Associated file}  \\ \hline
\endhead
\hline
\multicolumn{3}{| c |}{{continued on next page}}\\ 
\hline
\endfoot
\endlastfoot
\hline
HeaderPointSoilType \index{HeaderPointSoilType} & column name in the file PointFile for the soil type of the point & PointFile  \\ \hline
HeaderSoilDz \index{HeaderSoilDz} & column name in the file SoilParFile for the layers thickness & SoilParFile  \\ \hline
HeaderNormalHydrConductivity \index{HeaderNormalHydrConductivity} & column name in the file SoilParFile for the normal hydraulic conductivity & SoilParFile  \\ \hline
HeaderLateralHydrConductivity \index{HeaderLateralHydrConductivity} & column name in the file SoilParFile for the lateral hydraulic conductivity & SoilParFile  \\ \hline
HeaderThetaRes \index{HeaderThetaRes} & column name in the file SoilParFile for the residual water content & SoilParFile  \\ \hline
HeaderWiltingPoint \index{HeaderWiltingPoint} & column name in the file SoilParFile for the soil wilting point & SoilParFile  \\ \hline
HeaderFieldCapacity \index{HeaderFieldCapacity} & column name in the file SoilParFile for the field capacity & SoilParFile  \\ \hline
HeaderThetaSat \index{HeaderThetaSat} & column name in the file SoilParFile for the saturated water content & SoilParFile  \\ \hline
HeaderAlpha \index{HeaderAlpha} & column name in the file alpha parameter of Van Genuchten & SoilParFile  \\ \hline
HeaderN \index{HeaderN} & column name in the file N parameter of Van Genuchten & SoilParFile  \\ \hline
HeaderV \index{HeaderV} & column name in the file V parameter of Van Genuchten & SoilParFile  \\ \hline
HeaderSpecificStorativity \index{HeaderSpecificStorativity} & column name in the file specific storativity & SoilParFile  \\ \hline
\caption{Keywords of headers related to soil}
\label{header_soil}
\end{longtable}
\end{center}



\subsection{Parameters}

\begin{center}
\begin{longtable}{|p {4.0 cm}|p {4.7 cm}|p {0.8 cm}|p{0.8 cm}|p{1. cm}|p{0.8 cm}|p{1.3 cm}|}
\hline
\textbf{Keyword} & \textbf{Description} & \textbf{M. U.} & \textbf{range} & \textbf{Default Value} & \textbf{Sca / Vec} & \textbf{Str / Num / Opt} \\ \hline
\endfirsthead
\hline
\multicolumn{7}{| c |}{continued from previous page} \\
\hline
\textbf{Keyword} & \textbf{Description} & \textbf{M. U.} & \textbf{range} & \textbf{Default Value} & \textbf{Sca / Vec} & \textbf{Str / Num / Opt} \\ \hline
\endhead
\hline
\multicolumn{7}{| c |}{{continued on next page}}\\ 
\hline
\endfoot
\endlastfoot
\hline
FrozenSoilHydrCondReduction \index{FrozenSoilHydrCondReduction} & $\Omega$: Reduction factor of the hydraulic conductivity in partially frozen soil ($K=K_{no\_ice}*10^{\Omega Q}$, where Q is the ice ratio & - & 0, 7 & 2 & sca & num \\ \hline
\caption{Keywords for the description of soil}
\label{soil_parameter}
\end{longtable}
\end{center}



\begin{center}
\begin{longtable}{|p {4.2 cm}|p {4. cm}|p {1. cm}|p{0.8 cm}|p{1.4 cm}|p{0.8 cm}|p{1.3 cm}|}
\hline
\textbf{Keyword} & \textbf{Description} & \textbf{M. U.} & \textbf{range} & \textbf{Default Value} & \textbf{Sca / Vec} & \textbf{Str / Num / Opt} \\ \hline
\endfirsthead
\hline
\multicolumn{7}{| c |}{continued from previous page} \\
\hline
\textbf{Keyword} & \textbf{Description} & \textbf{M. U.} & \textbf{range} & \textbf{Default Value} & \textbf{Sca / Vec} & \textbf{Str / Num / Opt} \\ \hline
\endhead
\hline
\multicolumn{7}{| c |}{{continued on next page}}\\ 
\hline
\endfoot
\endlastfoot
\hline
NormalHydrConductivity \index{NormalHydrConductivity} &  & mm s$^{-1}$ &  & 1.00E-04 & vec & num \\ \hline
LateralHydrConductivity \index{LateralHydrConductivity} &  & mm s$^{-1}$ &  & 1.00E-04 & vec & num \\ \hline
ThetaRes \index{ThetaRes} &  & - &  & 0.05 & vec & num \\ \hline
WiltingPoint \index{WiltingPoint} &  & - &  & 0.15 & vec & num \\ \hline
FieldCapacity \index{FieldCapacity} &  & - &  & 0.25 & vec & num \\ \hline
ThetaSat \index{ThetaSat} &  & - &  & 0.5 & vec & num \\ \hline
AlphaVanGenuchten \index{AlphaVanGenuchten} &  & mm$^{-1}$ &  & 0.004 & vec & num \\ \hline
NVanGenuchten \index{NVanGenuchten} &  & - &  & 1.3 & vec & num \\ \hline
VMualem \index{VMualem} &  & - &  & 0.5 & vec & num \\ \hline
SpecificStorativity \index{SpecificStorativity} &  & mm$^{-1}$ &  & 1.00E-07 & vec & num \\ \hline
\caption{Keywords of soil input parameters settable in geotop.inpts}
\label{soil_par}
\end{longtable}
\end{center}



\begin{center}
\begin{longtable}{|p {5.0 cm}|p {3.6 cm}|p {1. cm}|p{0.8 cm}|p{1.4 cm}|p{0.8 cm}|p{1.3 cm}|}
\hline
\textbf{Keyword} & \textbf{Description} & \textbf{M. U.} & \textbf{range} & \textbf{Default Value} & \textbf{Sca / Vec} & \textbf{Str / Num / Opt} \\ \hline
\endfirsthead
\hline
\multicolumn{7}{| c |}{continued from previous page} \\
\hline
\textbf{Keyword} & \textbf{Description} & \textbf{M. U.} & \textbf{range} & \textbf{Default Value} & \textbf{Sca / Vec} & \textbf{Str / Num / Opt} \\ \hline
\endhead
\hline
\multicolumn{7}{| c |}{{continued on next page}}\\ 
\hline
\endfoot
\endlastfoot
\hline
FrozenSoilHydrCondReduction \index{FrozenSoilHydrCondReduction} & Reduction factor of the hydraulic conductivity in partially frozen soil ($K=K_{no\_ice}*10^{impedence Q}$, where Q is the ice ratio & - & 0, 7 & 2 & sca & num \\ \hline
PointSoilType \index{PointSoilType} & Soil type of the simulation point & - &  & NA & vec & num \\ \hline
NormalHydrConductivityBedrock \index{NormalHydrConductivityBedrock} &  & mm s$^{-1}$ &  & 1.00E-04 & vec & num \\ \hline
LateralHydrConductivityBedrock \index{LateralHydrConductivityBedrock} &  & mm s$^{-1}$ &  & 1.00E-04 & vec & num \\ \hline
ThetaResBedrock \index{ThetaResBedrock} &  & - &  & 0.05 & vec & num \\ \hline
WiltingPointBedrock \index{WiltingPointBedrock} &  & - &  & 0.15 & vec & num \\ \hline
FieldCapacityBedrock \index{FieldCapacityBedrock} &  & - &  & 0.25 & vec & num \\ \hline
ThetaSatBedrock \index{ThetaSatBedrock} &  & - &  & 0.5 & vec & num \\ \hline
AlphaVanGenuchtenBedrock \index{AlphaVanGenuchtenBedrock} &  & mm$^{-1}$ &  & 0.004 & vec & num \\ \hline
NVanGenuchtenBedrock \index{NVanGenuchtenBedrock} &  & - &  & 1.3 & vec & num \\ \hline
VMualemBedrock \index{VMualemBedrock} &  & - &  & 0.5 & vec & num \\ \hline
SpecificStorativityBedrock \index{SpecificStorativityBedrock} &  & mm$^{-1}$ &  & 1.00E-07 & vec & num \\ \hline
\caption{Keywords of soil input parameters settable in geotop.inpts}
\label{rock_par}
\end{longtable}
\end{center}

\subsubsection{Numerics}

\begin{center}
\begin{longtable}{|p {3.5 cm}|p {5 cm}|p {1 cm}|p{1 cm}|p{1.1 cm}|p{1.cm}|p{1 cm}|}
\hline
\textbf{Keyword} & \textbf{Description} & \textbf{M. U.} & \textbf{range} & \textbf{Default Value} & \textbf{Scalar / Vector} & \textbf{Logical / Numeric} \\ \hline
\endfirsthead
\hline
\multicolumn{7}{| c |}{continued from previous page} \\
\hline
\textbf{Keyword} & \textbf{Description} & \textbf{M. U.} & \textbf{range} & \textbf{Default Value} & \textbf{Sca / Vec} & \textbf{Log / Num} \\ \hline
\endhead
\hline
\multicolumn{7}{| c |}{{continued on next page}}\\ 
\hline
\endfoot
\endlastfoot
\hline
RichardTol \index{RichardTol} & Absolute Tolerance for the integration of Richards' equation on the Euclidean norm of residuals (mass balance)  & mm & 1E-20, inf & 1.00E-08 & sca & num \\ \hline
RichardMaxIter \index{RichardMaxIter}  & Max iterations for the integration of Richards' equation (mass balance equation) & - & 1, inf & 100 & sca & num \\ \hline
RichardInitForc \index{RichardInitForc} & Initial forcing term of Newton method (mass balance equation) & - &  & 0.01 & sca & num \\ \hline
%MinTimeStepSupFlow \index{MinTimeStepSupFlow} & minimum integration time step for the integration (surface flow equation)  &  &  & 0.01 & sca & num \\ \hline
%CanopyMaxIter \index{CanopyMaxIter} & Max number of iterations for (vegetation energy balance equation) &  &  & 3 & sca & num \\ \hline
%LocMaxIter \index{LocMaxIter} & Max number of iterations for the calculation of the within-canopy Monin-Obukhov length (vegetation energy balance equation) & - &  & 3 & sca & num \\ \hline
%TsMaxIter \index{TsMaxIter} & Max number of iterations for the calculation of canopy air temperature (vegetation energy balance equation) & - &  & 2 & sca & num \\ \hline
%CanopyStabCorrection \index{CanopyStabCorrection} & Use of the stability corrections within canopy (=1), otherwise (=0) & - &  & 1 & sca & opt \\ \hline
%BusingerMaxIter \index{BusingerMaxIter} & Max number of iterations for Monin-Obulhov stability algorithm -Businger parameterization (surface energy balance equation) & - &  & 5 & sca & num \\ \hline
\caption{Keywords of input numeric parameters for the energy and mass balance equation settable in geotop.inpts}
\label{numeric1d_numeric}
\end{longtable}
\end{center}



\section{Output}


\subsection{Point output}
\subsubsection{Files}

\begin{center}
\begin{longtable}{|p {6.3 cm}|p {7.5 cm}|}
\hline
\textbf{Keyword} & \textbf{Description}  \\ \hline
\endfirsthead
\hline
\multicolumn{2}{| c |}{continued from previous page} \\
\hline
\textbf{Keyword} & \textbf{Description}   \\ \hline
\endhead
\hline
\multicolumn{2}{| c |}{{continued on next page}}\\ 
\hline
\endfoot
\endlastfoot
\hline
PointOutputFile \index{PointOutputFile} & name of the file providing the properties for the simulation point \\ \hline
PointOutputFileWriteEnd \index{PointOutputFileWriteEnd} & name of the output file providing the Point values written just once at the end \\ \hline
SoilLiqWaterPressProfileFile \index{SoilLiqWaterPressProfileFile} & name of the output file providing the Soil/rock instantaneous liquid water pressure head values at various depths  \\ \hline
SoilLiqWaterPressProfileFileWriteEnd \index{SoilLiqWaterPressProfileFileWriteEnd} & name of the output file providing the Soil/rock instantaneous liquid water pressure head values at various depths written just once at the end  \\ \hline
SoilTotWaterPressProfileFile \index{SoilTotWaterPressProfileFile} & name of the output file providing the Soil/rock instantaneous total (water+ice) pressure head values at various depths  \\ \hline
SoilTotWaterPressProfileFileWriteEnd \index{SoilTotWaterPressProfileFileWriteEnd} & name of the output file providing the Soil/rock instantaneous total (water+ice) pressure head values at various depths written just once at the end  \\ \hline
SoilLiqContentProfileFile \index{SoilLiqContentProfileFile} & name of the output file providing the Soil/rock instantaneous liquid water content values at various depths  \\ \hline
SoilLiqContentProfileFileWriteEnd \index{SoilLiqContentProfileFileWriteEnd} & name of the output file providing the Soil/rock instantaneous liquid water content values at various depths written just once at the end  \\ \hline
SoilAveragedLiqContentProfileFile \index{SoilAveragedLiqContentProfileFile} & name of the output file providing the Soil/rock average (in DtPlotPoint) liquid water content values at various depths  \\ \hline
SoilAveragedLiqContentProfileFileWriteEnd \index{SoilAveragedLiqContentProfileFileWriteEnd} & name of the output file providing the Soil/rock average (in DtPlotPoint) liquid water content values at various depths written just once at the end  \\ \hline
\caption{Keywords of output file related to soil}
\label{soil_file}
\end{longtable}
\end{center}



\subsubsection{Parameters}

\begin{center}
\begin{longtable}{|p {3.4 cm}|p {4.7 cm}|p {1. cm}|p{0.8 cm}|p{1.4 cm}|p{0.8 cm}|p{1.3 cm}|}
\hline
\textbf{Keyword} & \textbf{Description} & \textbf{M. U.} & \textbf{range} & \textbf{Default Value} & \textbf{Sca / Vec} & \textbf{Str / Num / Opt} \\ \hline
\endfirsthead
\hline
\multicolumn{7}{| c |}{continued from previous page} \\
\hline
\textbf{Keyword} & \textbf{Description} & \textbf{M. U.} & \textbf{range} & \textbf{Default Value} & \textbf{Sca / Vec} & \textbf{Str / Num / Opt} \\ \hline
\endhead
\hline
\multicolumn{7}{| c |}{{continued on next page}}\\ 
\hline
\endfoot
\endlastfoot
\hline
DefaultSoil \index{DefaultSoil} & 0: use personal setting, 1:use default & - & 0, 1 & 1 & sca & opt \\ \hline
SoilPlotDepths \index{SoilPlotDepths} & depth at which one wants the data on the snow to be plotted & m &  & NA & vec & num \\ \hline
DateSoil \index{DateSoil} & column number in which one would like to visualize the Date12[DDMMYYYY hhmm]    	 & - &  & -1 & sca & num \\ \hline
JulianDayFromYear0Soil \index{JulianDayFromYear0Soil} & column number in which one would like to visualize the JulianDayFromYear0[days]   	 & - &  & -1 & sca & num \\ \hline
TimeFromStartSoil \index{TimeFromStartSoil} & column number in which one would like to visualize the time from the start of the soil & - &  & -1 & sca & num \\ \hline
PeriodSoil \index{PeriodSoil} & Column number to write the period number & - &  & -1 & sca & num \\ \hline
RunSoil \index{RunSoil} & Column number to write the run number & - &  & -1 & sca & num \\ \hline
IDPointSoil \index{IDPointSoil} & column number in which one would like to visualize the IDpoint  & - &  & -1 & sca & num \\ \hline
\caption{Keywords defining the column number where to print the desired variable in the output files for the soil variables }
\label{soil_column}
\end{longtable}
\end{center}

\begin{center}
\begin{longtable}{|p {3.2 cm}|p {5.3 cm}|p {1 cm}|p{1. cm}|p{1.1 cm}|p{1. cm}|p{1 cm}|}
\hline
\textbf{Keyword} & \textbf{Description} & \textbf{M. U.} & \textbf{range} & \textbf{Default Value} & \textbf{Sca / Vec} & \textbf{Log / Num} \\ \hline
\endfirsthead
\hline
\multicolumn{7}{| c |}{continued from previous page} \\
\hline
\textbf{Keyword} & \textbf{Description} & \textbf{M. U.} & \textbf{range} & \textbf{Default Value} & \textbf{Sca / Vec} & \textbf{Log / Num} \\ \hline
\endhead
\hline
\multicolumn{7}{| c |}{{continued on next page}}\\ 
\hline
\endfoot
\endlastfoot
\hline
DefaultPoint \index{DefaultPoint} & 0: use personal setting, 1:use default & - & 0, 1 & 1 & sca & opt \\ \hline
DtPlotPoint \index{DtPlotPoint} & Plotting Time step (in hour) of the output for specified pixels (0 means the it is not plotted) & h & 0, inf & 0 & vec & num \\ \hline
DatePoint \index{DatePoint} & column number in which one would like to visualize the Date12[DDMMYYYY hhmm]    	 & - & 1, 76 & -1 & sca & num \\ \hline
JulianDayFromYear0Point \index{JulianDayFromYear0Point} & column number in which one would like to visualize the JulianDayFromYear0[days]   	 & - & 1, 76 & -1 & sca & num \\ \hline
TimeFromStartPoint \index{TimeFromStartPoint} & column number in which one would like to visualize the TimeFromStart[days]  & - & 1, 76 & -1 & sca & num \\ \hline
PeriodPoint \index{PeriodPoint} & column number in which one would like to visualize the Simulation\_Period & - & 1, 76 & -1 & sca & num \\ \hline
RunPoint \index{RunPoint} & column number in which one would like to visualize the Run	 & - & 1, 76 & -1 & sca & num \\ \hline
IDPointPoint \index{IDPointPoint} & column number in which one would like to visualize the IDpoint  & - & 1, 76 & -1 & sca & num \\ \hline
WaterTableDepthPoint \index{WaterTableDepthPoint} & column number in which one would like to visualize the water\_table\_depth [mm]  & - & 1, 76 & -1 & sca & num \\ \hline
\caption{Keywords defining the column number where to print the desired variable in the PointOutputFile}
\label{soilcolumn_pointoutput}
\end{longtable}
\end{center}


\subsection{Map Output}

\subsubsection{Parameters}
\begin{center}
\begin{longtable}{|p {3.4 cm}|p {4.7 cm}|p {1. cm}|p{0.8 cm}|p{1.4 cm}|p{0.8 cm}|p{1.3 cm}|}
\hline
\textbf{Keyword} & \textbf{Description} & \textbf{M. U.} & \textbf{range} & \textbf{Default Value} & \textbf{Sca / Vec} & \textbf{Str / Num / Opt} \\ \hline
\endfirsthead
\hline
\multicolumn{7}{| c |}{continued from previous page} \\
\hline
\textbf{Keyword} & \textbf{Description} & \textbf{M. U.} & \textbf{range} & \textbf{Default Value} & \textbf{Sca / Vec} & \textbf{Str / Num / Opt} \\ \hline
\endhead
\hline
\multicolumn{7}{| c |}{{continued on next page}}\\ 
\hline
\endfoot
\endlastfoot
\hline
OutputSoilMaps \index{OutputSoilMaps} & frequency (h) of printing of the results of the soil maps & h &  & 0 & sca & num \\ \hline
\caption{Keywords of frequency for printing soil output maps}
\label{soil_freq}
\end{longtable}
\end{center}


\subsection{Map names}


\begin{center}
\begin{longtable}{|p {5.3 cm}|p {9 cm}|}
\hline
\textbf{Keyword} & \textbf{Description}  \\ \hline
\endfirsthead
\hline
\multicolumn{2}{| c |}{continued from previous page} \\
\hline
\textbf{Keyword} & \textbf{Description}   \\ \hline
\endhead
\hline
\multicolumn{2}{| c |}{{continued on next page}}\\ 
\hline
\endfoot
\endlastfoot
\hline
SoilMapFile \index{SoilMapFile} & name of the file providing the soil map  \\ \hline
FirstSoilLayerLiqContentMapFile \index{FirstSoilLayerLiqContentMapFile} & name of the map of the liquird water content of the first soil layer  \\ \hline
LandSurfaceWaterDepthMapFile \index{LandSurfaceWaterDepthMapFile} & name of the map of the water height above the surface  \\ \hline
WaterTableDepthMapFile \index{WaterTableDepthMapFile} & name of the output file providing the Water table depth map  \\ \hline
SpecificPlotSurfaceWaterContentMapFile \index{SpecificPlotSurfaceWaterContentMapFile} & name of the output file providing the surface water content map at high temporal resolution during specific days \\ \hline
\caption{Keywords of print output maps for soil and rock thermal and hydraulic variables}
\label{table_out_map_soil}
\end{longtable}
\end{center}

\subsection{Tensor names}


\begin{center}
\begin{longtable}{|p {3.9 cm}|p {9 cm}|}
\hline
\textbf{Keyword} & \textbf{Description}  \\ \hline
\endfirsthead
\hline
\multicolumn{2}{| c |}{continued from previous page} \\
\hline
\textbf{Keyword} & \textbf{Description}   \\ \hline
\endhead
\hline
\multicolumn{2}{| c |}{{continued on next page}}\\ 
\hline
\endfoot
\endlastfoot
\hline
SoilLiqContentTensorFile \index{SoilLiqContentTensorFile} & Name of the ensamble of raster maps corresponding to the liquid water content of each layer (if PlotSoilDepth$\neq$0 it writes the value at the corresponding depths)  \\ \hline
SoilLiqWaterPressTensorFile \index{SoilLiqWaterPressTensorFile} & Name of the ensamble of raster maps corresponding to the water pressure of each layer (if PlotSoilDepth$\neq$0 it writes the value at the corresponding depths)  \\ \hline
\caption{Keywords of print output tensor maps for soil and rock thermal and hydraulic variables}
\label{table_out_tensor_soil}
\end{longtable}
\end{center}



%
%%%%%%%%%%%%%%%%%%%%%%%%%%%%%%%%%%%%%%%%%%%%%%%%%%%%%%%%%%%%%%%%
%\section{Soil hydraulic proprieties}\label{sec:soil}
%%%%%%%%%%%%%%%%%%%%%%%%%%%%%%%%%%%%%%%%%%%%%%%%%%%%%%%%%%%%%%%%

%

%\begin{table}[!h]
%\begin{footnotesize}
%\begin{center}
%\begin{tabular}{lcccccc}
%  \hline
%  \hline
%  \textsl{Texture}&	\textsl{$K_s$}		   & \textsl{$\theta_{res}$}	& \textsl{$\theta_{sat}$}	& \textsl{$\alpha$}	& \textsl{n}	\\
%  \textsl{class}  &	\textsl{[$mm$ $sec^{-1}$]}  &\textsl{[-]}			& \textsl{[-]}			& \textsl{[$mm^{-1}$]}	& \textsl{[-]}	\\
%  \hline
%Clay		&	0.00171		&		0.098		&	0.459			&		0.00149		&1.25		\\
%Silt		&	0.00506		&		0.050		&	0.489			&		0.00065		&1.67		\\
%Sand		&	0.07439		&		0.053		&	0.375			&		0.00352		&3.17		\\
%Loam		&	0.00139		&		0.061		&	0.399			&		0.00111		&1.47		\\
%Silty clay	&	0.00111		&		0.111		&	0.481			&		0.00162		&1.32		\\
%Sandy clay	&	0.00131		&		0.117		&	0.385			&		0.00334		&1.20		\\
%Clay loam	&	0.00095		&		0.079		&	0.442			&		0.00158		&1.41		\\
%Silt loam	&	0.00211		&		0.065		&	0.439			&		0.00050		&1.66		\\
%Sandy Loam	&	0.00443		&		0.039		&	0.387			&		0.00266		&1.44		\\
%Loamy sand	&	0.01218		&		0.049		&	0.390			&		0.00347		&1.74		\\
%Silty clay loam	&	0.00153		&		0.063		&	0.384			&		0.00210		&1.33		\\
%Sandy clay loam	&	0.00129		&		0.090		&	0.482			&		0.00083		&1.52		\\
%\hline
%\hline
%\end{tabular}
%\textsl{\caption{ROSETTA Class Average Hydraulic Parameters}}\label{rosetta}
% \end{center}
%\end{footnotesize}
%\end{table}

%

%%%%%%%%%%%%%%%%%%%%%%%%%%%%%%%%%%%%%%%%%%%%%%%%%%%%%%%%%%%%%%%%
%\section{Soil thermal properties}
%%%%%%%%%%%%%%%%%%%%%%%%%%%%%%%%%%%%%%%%%%%%%%%%%%%%%%%%%%%%%%%%

%
%\noindent Example of thermal properties of natural materials, \textsl{Boundary Layer Climates - T.R.Oke}
%\begin{table}[!h]
%\begin{footnotesize}
%\begin{center}
%\begin{tabular}{llllllll}
%  \hline
%  \hline
%  \textsl{}             	& \textsl{}	  	& \textsl{$\rho$} 	         	& \textsl{$c$}      				     &   \textsl{C}         				& \textsl{d}  			& \textsl{$\kappa$} 	 		& \textsl{$\mu$}  \\
%  \textsl{Material}         	& \textsl{Remarks}  	& \textsl{Density} 	         	& \textsl{Specific} 			             &   \textsl{Heat}         			        & \textsl{Thermal}  		& \textsl{Thermal} 	 		& \textsl{Thermal}  \\
%  \textsl{}	         	& \textsl{}	  	& \textsl{}	 	         	& \textsl{heat} 				     &   \textsl{capacity}         			& \textsl{conductivity}  	& \textsl{diffusivity} 	 		& \textsl{admittance}  \\
%  \textsl{}	         	& \textsl{}	  	& \textsl{$kg$ $m^{-3}$}		& \textsl{$J$ $Kg^{-1}$ $k^{-1}$}   		     &   \textsl{$J$ $Kg^{-1}$ $k^{-1}$}    	   	& \textsl{$W$ $m^{-1}$ $k^{-1}$}& \textsl{$m^2$ $s^{-1}$}	 	& \textsl{$J$ $m^{-2}$ $s^{-1/2}$ $K^{-1}$}  \\
%  \textsl{}	         	& \textsl{}	  	& \textsl{$\times 10^3$}		& \textsl{$\times 10^3$}  			     &   \textsl{$ \times 10^6$}       			& 			  	& \textsl{$\times 10^{-6}$} 		&  \\
%  \hline
%  Sandy soil			& Dry			&	1.60 & 0.80  & 1.28 & 0.30 & 0.24 & 620\\
%    ($40\%$ pore		&  			&	\\
%     space)			& Saturated		&	2.00 & 1.48 & 2.96 & 2.20 & 0.74 & 2550	\\
%  Clay soil			& Dry			&	1.60 & 0.89 & 1.42 & 0.25 & 0.18 & 600 \\
%    ($40\%$ pore		& 			&	\\
%     space)			& Saturated		&	2.00 & 1.55 & 3.10 & 1.58 & 0.51 & 2210 \\
%  Peat soil			& Dry			&	0.30 & 1.92 & 0.58 & 0.06 & 0.10 & 190 \\
%    ($80\%$ pore		& 			&	\\
%     space)			& Saturated		&	1.10 & 3.65 & 4.02 & 0.50 & 0.12 & 1420 \\
%  Snow				& Fresh			&	0.10 & 2.09 & 0.21 & 0.08 & 0.10 & 130 \\
%  				& Old			&	0.48 & 2.09 & 0.84 & 0.42 & 0.40 & 595 \\
%  Ice				& 0\textdegree C,pure	&	0.92 & 2.10 & 1.93 & 2.24 & 1.16 & 2080\\
%  Water				& 4\textdegree C,still	&	1.00 & 4.18 & 4.18 & 0.57 & 0.14 & 1545\\
%  Air 				& 10\textdegree C,still	&	0.0012 & 1.01 & 0.0012 & 0.025 & 21.50 & 5\\
%   				& Turbolent		&	0.0012 & 1.01 & 0.0012 & $\sim$ 125 & $\sim$ 10 $\times 10^{6}$ & 390\\
%\hline
%\hline
%\end{tabular}
%\textsl{\caption{Thermal properties of natural materials, \textsl{Boundary Layer Climates - T.R.Oke}}}\label{ddd}
% \end{center}
%\end{footnotesize}
%\end{table}

%
%ROSETTA Class Average Hydraulic Parameters \\
%The table below gives class-average values of the seven hydraulic parameters for the 
%twelve USDA textural classes. Effectively, this table represents the first model of the hierarchical
% sequence. For the qr, q s, a, n and Ks parameters, the values have been generated by computing the average 
%values for each textural class. For Ko and L the values were generated by inserting the class average values 
%of qr, qs, a, n into Model C2 (see Rosetta's help file). This means that Ko and L are based on predicted parameters 
%and may not be very reliable. The values in parentheses give the one standard deviation uncertainties of the class 
%average values.\\
%\begin{table}[!h]
%\begin{footnotesize}
%\begin{center}
%\begin{tabular}{lccccccc}
%  \hline
%  \hline
%  \textsl{Texture}&	\textsl{$K_s$}		   & \textsl{$K_0$}		& \textsl{$\theta_{res}$}	& \textsl{$\theta_{sat}$}	& \textsl{$\alpha$}	& \textsl{n}	& \textsl{v}\\
%  \textsl{class}  &	\textsl{[$mm$ $sec^{-1}$]} & \textsl{[$mm$ $sec^{-1}$]} &\textsl{[-]}			& \textsl{[-]}			& \textsl{[$mm^{-1}$]}	& \textsl{[-]}	& \textsl{[-]}\\
%  \hline
%Clay		&	0.00171		&		0.00034			&	0.098		&	0.459			&		0.00149		&1.25		&-1.56	\\
%Silt		&	0.00506		&		0.00039			&	0.050		&	0.489			&		0.00065		&1.67		&0.62	\\
%Sand		&	0.07439		&		0.00283			&	0.053		&	0.375			&		0.00352		&3.17		&-0.93	\\
%Loam		&	0.00139		&		0.00043			&	0.061		&	0.399			&		0.00111		&1.47		&-0.37	\\
%Silty clay	&	0.00111		&		0.00037			&	0.111		&	0.481			&		0.00162		&1.32		&-1.28	\\
%Sandy clay	&	0.00131		&		0.00050			&	0.117		&	0.385			&		0.00334		&1.20		&-3.66	\\
%Clay loam	&	0.00095		&		0.00058			&	0.079		&	0.442			&		0.00158		&1.41		&-0.76	\\
%Silt loam	&	0.00211		&		0.00020			&	0.065		&	0.439			&		0.00050		&1.66		&0.36	\\
%Sandy Loam	&	0.00443		&		0.00179			&	0.039		&	0.387			&		0.00266		&1.44		&-0.86	\\
%Loamy sand	&	0.01218		&		0.00282			&	0.049		&	0.390			&		0.00347		&1.74		&-0.87	\\
%Silty clay loam	&	0.00153		&		0.00080			&	0.063		&	0.384			&		0.00210		&1.33		&-1.28	\\
%Sandy clay loam	&	0.00129		&		0.00026			&	0.090		&	0.482			&		0.00083		&1.52		&-0.15	\\
%\hline
%\hline
%\end{tabular}
%\textsl{\caption{ROSETTA Class Average Hydraulic Parameters}}\label{}
% \end{center}
%\end{footnotesize}
%\end{table}



