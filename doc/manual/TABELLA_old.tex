%*******************Dichiarazione della classe di documento e pacchetti usati***********
\documentclass[11pt, a4paper, landscape]{article}  %uso carta a3 e dimnesione font: 11 punti,  tipo di documento
\usepackage[latin1]{inputenc}
%\usepackage[utf8x]{inputenc}                    %
%\usepackage[italian]{babel}
\usepackage{fancyhdr}
\usepackage{longtable}
\usepackage{caption}
\usepackage[usenames]{color}
\usepackage{times}
\usepackage{setspace}
\usepackage{amsmath}
\usepackage{textcomp}
\usepackage{float}
\usepackage{fancyhdr}
\usepackage{amsfonts}
\usepackage{amssymb}
\usepackage{float}
\usepackage{natbib}
\usepackage{rotating}
\usepackage{graphicx}
%
\usepackage[babel]{csquotes}                    %
\usepackage[T1]{fontenc}                        %
\usepackage{indentfirst}                        %
\usepackage{lmodern}                            %
%\usepackage[hang,  footnotesize,  it]{caption}    % 
%\usepackage{tikz}                               %
%\usepackage{vmargin}                            %
%\usepackage{booktabs}                           %
%\usepackage{cite}                         
%%%%%%%%%%%%%%%%%%%%%%%%%%%%%%%%%%%%%%%%%%%%%%%%%%%

\newif\ifpdf
\ifx\pdfoutput\undefined
\pdffalse % we are not running PDFLaTeX
\else
\pdfoutput=1 % we are running PDFLaTeX
\pdftrue
\fi

%\ifpdf
%\usepackage[pdftex]{graphicx}
%\else
%\usepackage{graphicx}
%\usepackage{graphicx}
%\fi

%*****************Tutta una serie di comandi di formattazione**************************
\textwidth=25cm %larghezza corpo testo (l'intero paragrafo,  non le parole singole)
\textheight=15cm %altezza del corpo del testo (spazio occupato)
\headsep=1cm %distanza del testo dall'intestazione in alto
\topmargin=1.5cm %distanza dell'intestazione dal limite superiore (che non e' il margine)*
\footskip=1cm %distanza tra il testo e il piede di pagina inferiore
\voffset=-2cm %zona tra il limite superiore e il margine del foglio (vedi manuale)
\evensidemargin=-0.50cm
\oddsidemargin=0cm


%\providecommand{\geotop}{\lower-.25em\hbox{G}E\lower.25em\hbox{O}\lower-.25em\hbox{T}O\lower-.25em\hbox{P}\@}
\providecommand{\geotop}{\lower-.25em\hbox{GEO}\lower-.25em\hbox{top}\@}

%*****************Inizio descrizione del testo: indice,  capitoli,  elenchi**************

\begin{document}
%%% TITOLO%%%%%%%%%%%%%%%%%%%%%%%%%%%%%%%

\begin{titlepage}
\thispagestyle{empty}
\topmargin=0cm

%\begin{figure}[htbp]
%\begin{center}
%\includegraphics[width=5cm]{./pic/logoPATgeologico.png}\\
%\vspace{0.2cm}
%\end{center}
%\end{figure}

%\begin{figure}
%\begin{minipage}[htp]{0.5\linewidth} % A minipage that covers half the page
%\centering
%\includegraphics[height=2.5cm]{./pic/logopermanet.pdf}
%\end{minipage}
%\hspace{0.1 cm} % To get a little bit of space between the figures
%\begin{minipage}[!tbp]{0.5\linewidth}
%\centering
%\includegraphics[height=2.5cm]{./pic/logoAlpineSpace.png}
%\vspace{0.5cm}
%\end{minipage}
%{\it The PermaNET project is part of the European Territorial Cooperation and co-funded by the European Regional Development Fund (ERDF) in the scope of the Alpine Space Programme www.alpine-space.eu}
%%\label{U_psi}
%\end{figure}


%\vspace{0.2cm}
%\begin{center}
%{\LARGE Report on the activity on permafrost modeling in the Autonomous Province of Trento}\\
%\end{center}
%\vspace{0.2 cm}
%\begin{center}
%{ \large Realized by} \\
%\vspace{0.5cm}
%{\LARGE \bf{Mountain-eering S.r.l.} }\\
%\vspace{1cm}
% ing. Filippo Zambon\\
% \vspace{0.1cm}
% Dr. Matteo Dall'Amico,  PhD\\
%\begin{figure}[t, b]
%\includegraphics[width=0.2\textwidth]{./firma_matteo.png}
%\end{figure}
%\end{center}


%\noindent{\large Report on the determination of the thermal parameters of the Cime Bianche borehole through physically based hydroL modeling}

\vspace{0.2cm}
\hfill{May 2011}\\


%\newpage
\mbox{}
\clearpage

%\thispagestyle{empty}
%\vspace{17cm}
%%\begin{figure}[H]
%%\includegraphics[width=8cm]{./pic/logo_R.pdf}\\
%%\end{figure}
%\noindent {\bf Mountain-eering S.r.l.} Societ\`a di Ingegneria \\
%Spin Off Universit\`a degli Studi di Trento \\
%Start Up del Technology Innovation Suedtirol (TIS) di Bolzano \\
%Sede legale: Siemens str. 19 via Siemens,  Bozen 39100 Bolzano - Italy \\
%Tel: 0471 068226 Fax: 0471 068229 \\
%Ufficio Tecnico: Via Giusti 10,  38100 Trento - Italy \\


\mbox{ } \thispagestyle{empty}

\end{titlepage}


%*********************Fine modifica titolo******************************************

\baselineskip=5.5mm %formattazione distanza tra le linee del testo

%\newpage
%\mbox{}
%\clearpage

%\pagenumbering{roman} \setcounter{page}{1}
%\tableofcontents %questo genera l'indice
%\clearpage


%**************************NUOVA FORMATTAZIONE TITOLI E PARAGRAFI E SPAZIATURA TRA PARAGRAFI*****
%\makeatletter %setto la @ come font comando

%\renewcommand{\chapter}{\@startsection  %formattazione titolo capitolo
 %      {chapter}                            %nome
%       {0}                                  %livello
%       {0mm}                                %rientro
%       {10cm}                               %spazio prima
%       {1cm}                              %spazio dopo
%       {\normalfont\huge\bfseries\upshape}} %formattazione carattere

%\renewcommand{\section}{\@startsection  %formattazione titolo sezione
%       {section}                            %nome
%       {1}                                  %livello
%       {0mm}                                %rientro
%       {2cm}                                %spazio prima
%       {1cm}                                %spazio dopo
%       {\normalfont\LARGE\bfseries\upshape}}%formattazione carattere

%\renewcommand{\subsection}{\@startsection  %formattazione titolo subsezione
%       {subsection}                           %nome
%       {2}                                         %livello
%       {0mm}                                  %rientro
%      {1.5cm}                                %spazio prima
%       {0.5cm}                                %spazio dopo
 %      {\normalfont\Large\bfseries\upshape}}  %formattazione carattere

%\renewcommand{\subsubsection}{\@startsection  %formattazione titolo subsubsezione
%       {subsubsection}                        %nome
%       {3}                                    %livello
%       {0mm}                                  %rientro
%       {1.5cm}                                %spazio prima
%       {0.5cm}                                %spazio dopo
%       {\normalfont\large\bfseries\upshape}}  %formattazione carattere

%\renewcommand{\paragraph}{\@startsection      %formattazione titolo paragraph
%       {paragraph}                            %nome
%       {4}                                    %livello
%       {0mm}                                  %rientro
%       {1cm}                                  %spazio prima
%       {0.7cm}                                %spazio dopo
%       {\normalfont\normalsize\itshape}}      %formattazione carattere

%\makeatother %riporto la @ a font normale
%**************************FINE FORMATTAZIONE TITOLI E PARAGRAFI E SPAZIATURA TRA PARAGRAFI*****

\parindent=0.5cm %formattaz. rientro dell'inizio paragrafo
%\newpage
%\mbox{ }  \thispagestyle{empty}
\pagenumbering{arabic} \setcounter{page}{0} %numerazioni pagine (testo)

%\clearpage
%\newpage
\pagestyle{fancy} %stile di formattazione di pagina (per l'intestazione eil pi� del documento)
\fancyhead[LE, RO]{\footnotesize \slshape {}} %formattazione intestazione (vedi man.)
\fancyhead[LO, RE]{\footnotesize \slshape \leftmark} %come sopra (vedi manuale)
%\renewcommand{\chaptermark}[1]{\markboth{\thechapter.\ #1}{}} %come sopra
\renewcommand{\sectionmark}[1]{\markright{\thesection.\ #1}}
\renewcommand{\captionfont}{\footnotesize} %formattazione didascalie foto e tabelle
\renewcommand{\headrulewidth}{0.2pt}

%\part{Relazione}
\tableofcontents %questo genera l'indice
\clearpage

\section{1D INPUT NUMERIC}

\begin{center}
\begin{longtable}{|p {6.5 cm}|p {4.5 cm}|p {3 cm}|p{3 cm}|p{1.5 cm}|p{1.5 cm}|p{2 cm}|}
\hline
\textbf{Keyword} & \textbf{Description} & \textbf{M. U.} & \textbf{range} & \textbf{Default Value} & \textbf{Scalar / Vector} & \textbf{Logical / Numeric} \\ \hline
\endfirsthead
\hline
\multicolumn{7}{| c |}{continued from previous page} \\
\hline
\textbf{Keyword} & \textbf{Description} & \textbf{M. U.} & \textbf{range} & \textbf{Default Value} & \textbf{Scalar / Vector} & \textbf{Logical / Numeric} \\ \hline
\endhead
\hline
\multicolumn{7}{| c |}{{continued on next page}}\\ 
\hline
\endfoot
\endlastfoot
\hline
PointHorizon & ID number to which corresponds the horizon file & - &  & NA & vec & num \\ \hline
\caption{Table of topographic parameters  (numeric)}
\label{topo1d_parameters}
\end{longtable}
\end{center}

\clearpage







\begin{center}
\begin{longtable}{|p {6.5 cm}|p {4.5 cm}|p {3 cm}|p{3 cm}|p{1.5 cm}|p{1.5 cm}|p{2 cm}|}
\hline
\textbf{Keyword} & \textbf{Description} & \textbf{M. U.} & \textbf{range} & \textbf{Default Value} & \textbf{Scalar / Vector} & \textbf{Logical / Numeric} \\ \hline
\endfirsthead
\hline
\multicolumn{7}{| c |}{continued from previous page} \\
\hline
\textbf{Keyword} & \textbf{Description} & \textbf{M. U.} & \textbf{range} & \textbf{Default Value} & \textbf{Scalar / Vector} & \textbf{Logical / Numeric} \\ \hline
\endhead
\hline
\multicolumn{7}{| c |}{{continued on next page}}\\ 
\hline
\endfoot
\endlastfoot
\hline
FreeDrainageAtBottom & Boundary condition on Richards' equation at the bottom border (1: free drainage, 0: no flux) & - & 0,1 & 0 & sca & num \\ \hline
ZeroTempAmplitDepth & Depth [mm] at which the annual temperature remains constant. It is used as the bottom boundary condition of the heat equation. The Zero flux condition can be assigned setting this parameter at a very high value & mm &  & 1.00E+20 & sca & num \\ \hline
ZeroTempAmplitTemp & Temperature at the depth assigned above & $^\circ$C &  & 20 & sca & num \\ \hline
BottomBoundaryHeatFlux & Incoming heat flux at the bottom boundary of the soil domain (geothermal heat flux) & W m$^{-2}$ &  & 0 & sca & num \\ \hline
\caption{Table of boundary condition  (numeric)}
\label{BC1d}
\end{longtable}
\end{center}


\clearpage
\begin{center}
\begin{longtable}{|p {6.5 cm}|p {4.5 cm}|p {3 cm}|p{3 cm}|p{1.5 cm}|p{1.5 cm}|p{2 cm}|}
\hline
\textbf{Keyword} & \textbf{Description} & \textbf{M. U.} & \textbf{range} & \textbf{Default Value} & \textbf{Scalar / Vector} & \textbf{Logical / Numeric} \\ \hline
\endfirsthead
\hline
\multicolumn{7}{| c |}{continued from previous page} \\
\hline
\textbf{Keyword} & \textbf{Description} & \textbf{M. U.} & \textbf{range} & \textbf{Default Value} & \textbf{Scalar / Vector} & \textbf{Logical / Numeric} \\ \hline
\endhead
\hline
\multicolumn{7}{| c |}{{continued on next page}}\\ 
\hline
\endfoot
\endlastfoot
\hline
InitDateDDMMYYYYhhmm & Date and time of the simulation start in date12 format (MANDATORY) & format DDMMYYhhmm & 01/01/1800 00:00, 01/01/2500 00:00 & NA & vec & str \\ \hline
EndDateDDMMYYYYhhmm & Date and time of the simulation start in date12 format (MANDATORY) & format DDMMYYhhmm & 01/01/1800 00:00, 01/01/2500 00:00 & NA & vec & str \\ \hline
NumSimulationTimes & How many times the simulation is run (if >1, it uses the final condition as initial conditions of the new simulation) & - & 0, inf & 1 & vec & num \\ \hline
StandardTimeSimulation & Standard time to which all the output data are referred (difference respect UMT, in hours): GMT + x [h] & h & 0, 12 & 0 & sca & num \\ \hline
PointSim & Point simulation (=1), distributed simulation (=0) & - & 0, 1 & 0 & sca & opt \\ \hline
RecoverSim & Simulation recovered (=number of saving point you want to start from), otherwise (=0) & - & 0, 1 & 0 & sca & opt \\ \hline
WaterBalance & Activate water balance (Yes=1, No=0) & - &  & 0 & sca & opt \\ \hline
EnergyBalance & Activate energy balance (Yes=1, No=0) &  &  & 0 & sca & opt \\ \hline
PixelCoordinates & Write 1 IF ALL point COORDINATES ARE IN FORMAT (EAST,NORTH) in meters, Or 0 IF IN FORMAT ROW AND COLUMS (r,c) of the dem map & - &  & 1 & sca & opt \\ \hline
SavingPoints &  & - &  & NA & vec & num \\ \hline
SoilLayerTypes & Number of types of soil types, corresponding to different soil stratigraphies & - &  & 1 & sca & num \\ \hline
DefaultSoilTypeLand & given a multiple number of type of soil, this relates to the default given to the land type type & - &  & 1 & sca & num \\ \hline
DefaultSoilTypeChannel & given a multiple number of type of soil, this relates to the default given to the channel type & - &  & 1 & sca & num \\ \hline
\caption{Table of general parameters  (numeric)}
\label{general1d_numeric}
\end{longtable}
\end{center}


\clearpage
\begin{center}
\begin{longtable}{|p {6.5 cm}|p {4.5 cm}|p {3 cm}|p{3 cm}|p{1.5 cm}|p{1.5 cm}|p{2 cm}|}
\hline
\textbf{Keyword} & \textbf{Description} & \textbf{M. U.} & \textbf{range} & \textbf{Default Value} & \textbf{Scalar / Vector} & \textbf{Logical / Numeric} \\ \hline
\endfirsthead
\hline
\multicolumn{7}{| c |}{continued from previous page} \\
\hline
\textbf{Keyword} & \textbf{Description} & \textbf{M. U.} & \textbf{range} & \textbf{Default Value} & \textbf{Scalar / Vector} & \textbf{Logical / Numeric} \\ \hline
\endhead
\hline
\multicolumn{7}{| c |}{{continued on next page}}\\ 
\hline
\endfoot
\endlastfoot
\hline
IrriducibleWatSatGlacier & IRREDUCIBLE WATER SATURATION FOR GLACIER & - &  & 0.02 & sca & num \\ \hline
MaxWaterEqGlacLayerContent & maximum water equivalent admitted in a snow layer &  &  & 5 & sca & num \\ \hline
MaxGlacLayerNumber & maximum layers of snow to use (suggested $>$5) &  &  & 0 & sca & num \\ \hline
ThickerGlacLayers & Layer numbers that can become thicker than admitted by the threshold given by MaxGlacLayerNumber (from the bottom up). They can be more than one &  &  & Max Glac Layer Number/2 & vec & num \\ \hline
\caption{Table of glacier parameters  (numeric)}
\label{glacier1d_numeric}
\end{longtable}
\end{center}


\clearpage
\begin{center}
\begin{longtable}{|p {6.5 cm}|p {4.5 cm}|p {3 cm}|p{3 cm}|p{1.5 cm}|p{1.5 cm}|p{2 cm}|}
\hline
\textbf{Keyword} & \textbf{Description} & \textbf{M. U.} & \textbf{range} & \textbf{Default Value} & \textbf{Scalar / Vector} & \textbf{Logical / Numeric} \\ \hline
\endfirsthead
\hline
\multicolumn{7}{| c |}{continued from previous page} \\
\hline
\textbf{Keyword} & \textbf{Description} & \textbf{M. U.} & \textbf{range} & \textbf{Default Value} & \textbf{Scalar / Vector} & \textbf{Logical / Numeric} \\ \hline
\endhead
\hline
\multicolumn{7}{| c |}{{continued on next page}}\\ 
\hline
\endfoot
\endlastfoot
\hline
InitSWE & Initial snow water equivalent (SWE) - used if no snow map is given & kg m$^{-2}$ &  & 0 & sca & num \\ \hline
InitSnowDensity & INITIAL SNOW DENSITY - uniform with depth & kg m$^{-3}$ &  & 200 & sca & num \\ \hline
InitSnowTemp & INITIAL SNOW TEMPERATURE - uniform with depth & $^\circ$C &  & -3 & sca & num \\ \hline
InitSnowAge & INITIAL SNOW AGE & days &  & 0 & sca & num \\ \hline
InitGlacierDepth & GLACIER DEPTH - used if no snow map is given & mm &  & 0 & sca & num \\ \hline
InitGlacierDensity & INITIAL GLACIER DENSITY - uniform with depth & kg m$^{-3}$ &  & 800 & sca & num \\ \hline
InitGlacierTemp & INITIAL GLACIER TEMPERATURE - uniform with depth & $^\circ$C &  & -3 & sca & num \\ \hline
InitWaterTableHeightOverTopoSurface & initial condition on water table depth (positive downwards from ground surface). Used if InitSoilPressure is void & mm &  & 0 & sca & num \\ \hline
InitSoilPressure &  & mm &  & NA & vec & num \\ \hline
InitSoilTemp &  & $^\circ$C &  & 5 & vec & num \\ \hline
InitSoilPressureBedrock &  & mm &  & NA & vec & num \\ \hline
InitSoilTempBedrock &  & $^\circ$C &  & 5 & vec & num \\ \hline
\caption{Table of initial condition  (numeric)}
\label{IC1d_numeric}
\end{longtable}
\end{center}



\clearpage
\begin{center}
\begin{longtable}{|p {6.5 cm}|p {4.5 cm}|p {3 cm}|p{3 cm}|p{1.5 cm}|p{1.5 cm}|p{2 cm}|}
\hline
\textbf{Keyword} & \textbf{Description} & \textbf{M. U.} & \textbf{range} & \textbf{Default Value} & \textbf{Scalar / Vector} & \textbf{Logical / Numeric} \\ \hline
\endfirsthead
\hline
\multicolumn{7}{| c |}{continued from previous page} \\
\hline
\textbf{Keyword} & \textbf{Description} & \textbf{M. U.} & \textbf{range} & \textbf{Default Value} & \textbf{Scalar / Vector} & \textbf{Logical / Numeric} \\ \hline
\endhead
\hline
\multicolumn{7}{| c |}{{continued on next page}}\\ 
\hline
\endfoot
\endlastfoot
\hline
TimeStepEnergyAndWater & THE INTEGRATION time step [s] for energy and water balance equation (MANDATORY) & s & 0, inf & NA & vec & num \\ \hline
RichardTol & Absolute Tolerance for the integration of Richards' equation (on the Euclidean norm of residuals)  & mm & 1E-20, inf & 1.00E-08 & sca & num \\ \hline
RichardMaxIter & Max iterations for the integration of Richards' equation & - & 1, inf & 100 & sca & num \\ \hline
RichardInitForc & Initial forcing term of Newton method  & - &  & 0.01 & sca & num \\ \hline
MinTimeStepSupFlow & minimum integration time step for the integration of the surface runoff equation &  &  & 0.01 & sca & num \\ \hline
HeatEqTol & Max norm of the residuals for heat equation & J m$^{-2}$ &  & 1.00E-04 & sca & num \\ \hline
HeatEqMaxIter & Max number of iterations for heat equation & - &  & 500 & sca & num \\ \hline
CanopyMaxIter & Max number of iterations for canopy (heat) equation &  &  & 3 & sca & num \\ \hline
BusingerMaxIter & Max number of iterations for Monin-Obulhov stability algorithm (Businger parameterization) & - &  & 5 & sca & num \\ \hline
TsMaxIter & Max number of iterations for the calculation of canopy air temperature & - &  & 2 & sca & num \\ \hline
LocMaxIter & Max number of iterations for the calculation of the within-canopy Monin-Obukhov length & - &  & 3 & sca & num \\ \hline
CanopyStabCorrection & Use of the stability corrections within canopy (=1), otherwise (=0) & - &  & 1 & sca & num \\ \hline
\caption{Table of numeric parameters  (numeric)}
\label{numeric1d_numeric}
\end{longtable}
\end{center}



\clearpage
\begin{center}
\begin{longtable}{|p {6.5 cm}|p {4.5 cm}|p {3 cm}|p{3 cm}|p{1.5 cm}|p{1.5 cm}|p{2 cm}|}
\hline
\textbf{Keyword} & \textbf{Description} & \textbf{M. U.} & \textbf{range} & \textbf{Default Value} & \textbf{Scalar / Vector} & \textbf{Logical / Numeric} \\ \hline
\endfirsthead
\hline
\multicolumn{7}{| c |}{continued from previous page} \\
\hline
\textbf{Keyword} & \textbf{Description} & \textbf{M. U.} & \textbf{range} & \textbf{Default Value} & \textbf{Scalar / Vector} & \textbf{Logical / Numeric} \\ \hline
\endhead
\hline
\multicolumn{7}{| c |}{{continued on next page}}\\ 
\hline
\endfoot
\endlastfoot
\hline
ThetaResBedrock &  & - &  & 0.05 & vec & num \\ \hline
WiltingPointBedrock &  & - &  & 0.15 & vec & num \\ \hline
FieldCapacityBedrock &  & - &  & 0.25 & vec & num \\ \hline
ThetaSatBedrock &  & - &  & 0.5 & vec & num \\ \hline
AlphaVanGenuchtenBedrock &  & mm$^{-1}$ &  & 0.004 & vec & num \\ \hline
NVanGenuchtenBedrock &  & - &  & 1.3 & vec & num \\ \hline
VMualemBedrock &  & - &  & 0.5 & vec & num \\ \hline
ThermalConductivitySoilSolidsBedrock & thermal conductivity of the bedrock & W m$^{-1}$ K$^{-1}$ &  & 2.5 & vec & num \\ \hline
ThermalCapacitySoilSolidsBedrock & thermal capacity of the bedrock & J m$^{-3}$ K$^{-1}$ &  & 1.00E+06 & vec & num \\ \hline
SpecificStorativityBedrock &  & mm$^{-1}$ &  & 1.00E-07 & vec & num \\ \hline
\caption{Table of rock parameters  (numeric)}
\label{rock1d_numeric}
\end{longtable}
\end{center}


\clearpage
\begin{center}
\begin{longtable}{|p {6.5 cm}|p {4.5 cm}|p {3 cm}|p{3 cm}|p{1.5 cm}|p{1.5 cm}|p{2 cm}|}
\hline
\textbf{Keyword} & \textbf{Description} & \textbf{M. U.} & \textbf{range} & \textbf{Default Value} & \textbf{Scalar / Vector} & \textbf{Logical / Numeric} \\ \hline
\endfirsthead
\hline
\multicolumn{7}{| c |}{continued from previous page} \\
\hline
\textbf{Keyword} & \textbf{Description} & \textbf{M. U.} & \textbf{range} & \textbf{Default Value} & \textbf{Scalar / Vector} & \textbf{Logical / Numeric} \\ \hline
\endhead
\hline
\multicolumn{7}{| c |}{{continued on next page}}\\ 
\hline
\endfoot
\endlastfoot
\hline
ThresSnowSoilRough & Threshold on snow depth to change roughness to snow roughness values with d0 set at 0, for bare soil fraction & mm & 0, 1000 & 10 & sca & num \\ \hline
ThresSnowVegUp & Threshold on snow depth above which the roughness is snow roughness, for vegetation fraction & mm & 0, 20000 & 1000 & sca & num \\ \hline
ThresSnowVegDown & Threshold on snow depth below which the roughness is vegetation roughness, for vegetation fraction & mm & 0, 20000 & 1000 & sca & num \\ \hline
RoughElemXUnitArea & Number of roughness elements (=vegetation) per unit area - used only for blowing snow subroutines & Number m$^{-2}$ & 0, inf & 0 & sca & num \\ \hline
RoughElemDiam & Diameter [mm] of the roughness elements (=vegetation) - used only for blowing snow subroutines & mm & 0, inf & 50 & sca & num \\ \hline
AlphaSnow & Alpha (SNTHERM parameter) for the freezing characteristic soil for snow, the bigger, the steeper the curve around 0 degrees & - &  & 1.00E+05 & sca & num \\ \hline
ThresTempRain & DEW or AIR TEMPERATURE ABOVE WHICH ALL PRECIPITATION IS RAIN & $^\circ$C &  & 3 & sca & num \\ \hline
ThresTempSnow & DEW or AIR TEMPERATURE BELOW WHICH ALL PRECIPITATION IS SNOW & $^\circ$C &  & -1 & sca & num \\ \hline
DewTempOrNormTemp & Use dew temperature (1) or air temperature (0) to discriminate between snowfall and rainfall & - & 1 or 0 & 0 & sca & opt \\ \hline
AlbExtParSnow & ALBEDO EXTINCTION PARAMETER - if snow depth < aep, albedo is interpolated between soil and snow & mm &  & 10 & sca & num \\ \hline
FreshSnowReflVis & VISIBLE BAND REFLECTANCE OF fresh SNOW  & - &  & 0.9 & sca & num \\ \hline
FreshSnowReflNIR & near INFRARED BAND REFLECTANCE OF fresh SNOW & - &  & 0.65 & sca & num \\ \hline
IrriducibleWatSatSnow & IRREDUCIBLE WATER SATURATION - from Colbeck (0.02 - 0.07). It is the ratio of the capillarity-hold water to ice content in the snow & - & 0.02 - 0.07 & 0.02 & sca & num \\ \hline
SnowEmissiv & SNOW LONGWAVE EMISSIVITY [-] & - &  & 0.98 & sca & num \\ \hline
SnowRoughness & Roughness length over snow (mm) & mm &  & 0.1 & sca & num \\ \hline
SnowCorrFactor & correction factor on fresh snow accumulation &  &  & 1 & sca & num \\ \hline
MaxSnowPorosity & MAXIMUM SNOW POROSITY ALLOWED (-). This parameter prevents excessive snow densification & - &  & 0.7 & sca & num \\ \hline
DrySnowDefRate & SNOW COMPACTION (\% per hour) DUE TO DESTRUCTIVE METAMORPHISM for SNOW DENSITY<snow\_density\_ cutoff and DRY SNOW  & - &  & 1 & sca & num \\ \hline
SnowDensityCutoff & SNOW DENSITY CUTOFF (kg m$^{-3}$) TO CHANGE SNOW DEFORMATION RATE & kg m$^{-3}$ &  & 100 & sca & num \\ \hline
WetSnowDefRate & ENHANCEMENT FACTOR IN PRESENCE OF WET SNOW & - &  & 1.5 & sca & num \\ \hline
SnowViscosity & SNOW VISCOSITY COEFFICIENT (kg s m$^{-2}$) at T=0 C and snow density=0 & N s m$^{-2}$ &  & 1.00E+06 & sca & num \\ \hline
FetchUp & SCALING FETCH in case snow wind transport in increasing [m] & m &  & 1000 & sca & num \\ \hline
FetchDown & SCALING FETCH in case snow wind transport in decreasing [m] & m &  & 100 & sca & num \\ \hline
BlowingSnowSoftLayerIceContent & Snow depth (in ice water equivalent), the averaged density of which is used for blowing snow wind thresholds & kg m$^{-2}$ &  & 0 & sca & num \\ \hline
TimeStepBlowingSnow & Time step [s] at which the Prairie Blowing Snow Model is run & s &  & TimeStep Energy AndWater & sca & num \\ \hline
SnowSMIN & MINIMUM SLOPE [degree] TO ADJUST PRECIPITATION REDUCTION & degree &  & 30 & sca & num \\ \hline
SnowSMAX & MAXIMUM SLOPE [degree] TO ADJUST PRECIPITATION REDUCTION & degree &  & 80 & sca & num \\ \hline
SnowCURV & SHAPE PARAMETER FOR PRECIPITATION REDUCTION (if <0 the adjustment is not applied) & - &  & -200 & sca & num \\ \hline
MaxWaterEqSnowLayerContent & maximum water equivalent admitted in a snow layer & kg m$^{-2}$ &  & 5 & sca & num \\ \hline
MaxSnowLayerNumber & maximum layers of snow to use (suggested $>$10) &  &  & 10 & sca & num \\ \hline
ThickerSnowLayers & Layer numbers that can become thicker than admitted by the threshold given by MaxSnowLayerNumber (from the bottom up). They can be more than one &  &  & Max Snow Layer Number/2 & vec & num \\ \hline
BlowingSnow & Activate blowing snow module (yes=1, no=0) & - &  & 0 & sca & opt \\ \hline
PointMaxSWE & Max snow water equivalent that can be reached in the simulation point & kg m$^{-2}$ &  & NA & vec & num \\ \hline
SnowAgingCoeffVis & reflectance of the new snow in the visible wave length & - &  & 0.2 & sca & num \\ \hline
SnowAgingCoeffNIR & reflectance of the new snow in the infrared wave length & - &  & 0.5 & sca & num \\ \hline
\caption{Table of snow parameters  (numeric)}
\label{snow1d_numeric}
\end{longtable}
\end{center}



\clearpage
\begin{center}
\begin{longtable}{|p {6.5 cm}|p {4.5 cm}|p {3 cm}|p{3 cm}|p{1.5 cm}|p{1.5 cm}|p{2 cm}|}
\hline
\textbf{Keyword} & \textbf{Description} & \textbf{M. U.} & \textbf{range} & \textbf{Default Value} & \textbf{Scalar / Vector} & \textbf{Logical / Numeric} \\ \hline
\endfirsthead
\hline
\multicolumn{7}{| c |}{continued from previous page} \\
\hline
\textbf{Keyword} & \textbf{Description} & \textbf{M. U.} & \textbf{range} & \textbf{Default Value} & \textbf{Scalar / Vector} & \textbf{Logical / Numeric} \\ \hline
\endhead
\hline
\multicolumn{7}{| c |}{{continued on next page}}\\ 
\hline
\endfoot
\endlastfoot
\hline
FrozenSoilHydrCondReduction & Reduction factor of the hydraulic conductivity in partially frozen soil ($K=K_{no\_ice}*10^{impedence Q}$, where Q is the ice ratio & - & 0, 7 & 2 & sca & num \\ \hline
PointSoilType & Soil type of the simulation point & - &  & NA & vec & num \\ \hline
SoilLayerThicknesses & vector defining the thickness of the various soil layers. If not present, a column of 5 layers 100 mm thick will be assumed & mm &  & 100 & vec & num \\ \hline
SoilLayerNumber & number of soil layers (is calculated after the number of components of the vector SoilLayerNumber) & - &  & 5 & sca & num \\ \hline
NormalHydrConductivity &  & mm s$^{-1}$ &  & 1.00E-04 & vec & num \\ \hline
LateralHydrConductivity &  & mm s$^{-1}$ &  & 1.00E-04 & vec & num \\ \hline
ThetaRes &  & - &  & 0.05 & vec & num \\ \hline
WiltingPoint &  & - &  & 0.15 & vec & num \\ \hline
FieldCapacity &  & - &  & 0.25 & vec & num \\ \hline
ThetaSat &  & - &  & 0.5 & vec & num \\ \hline
AlphaVanGenuchten &  & mm$^{-1}$ &  & 0.004 & vec & num \\ \hline
NVanGenuchten &  & - &  & 1.3 & vec & num \\ \hline
VMualem &  & - &  & 0.5 & vec & num \\ \hline
ThermalConductivitySoilSolids & thermal conductivity of the soil particles & W m$^{-1}$ K$^{-1}$ &  & 2.5 & vec & num \\ \hline
ThermalCapacitySoilSolids & thermal capacity of the soil particles & J m$^{-3}$ K$^{-1}$ &  & 1.00E+06 & vec & num \\ \hline
SpecificStorativity &  & mm$^{-1}$ &  & 1.00E-07 & vec & num \\ \hline
NormalHydrConductivityBedrock &  & mm s$^{-1}$ &  & 1.00E-04 & vec & num \\ \hline
LateralHydrConductivityBedrock &  & mm s$^{-1}$ &  & 1.00E-04 & vec & num \\ \hline
\caption{Table of soil parameters  (numeric)}
\label{soil1d_numeric}
\end{longtable}
\end{center}

\clearpage
\begin{center}
\begin{longtable}{|p {6.5 cm}|p {4.5 cm}|p {3 cm}|p{3 cm}|p{1.5 cm}|p{1.5 cm}|p{2 cm}|}
\hline
\textbf{Keyword} & \textbf{Description} & \textbf{M. U.} & \textbf{range} & \textbf{Default Value} & \textbf{Scalar / Vector} & \textbf{Logical / Numeric} \\ \hline
\endfirsthead
\hline
\multicolumn{7}{| c |}{continued from previous page} \\
\hline
\textbf{Keyword} & \textbf{Description} & \textbf{M. U.} & \textbf{range} & \textbf{Default Value} & \textbf{Scalar / Vector} & \textbf{Logical / Numeric} \\ \hline
\endhead
\hline
\multicolumn{7}{| c |}{{continued on next page}}\\ 
\hline
\endfoot
\endlastfoot
\hline
NumLandCoverTypes & Number of Classes of land cover. Each land cover type corresponds to a particular land-cover state, described by a specific set of values of the parameters listed below. Each set of land cover parameters will be distributively assigned according to the land cover map, which relates each pixel with a land cover type number. This number corresponds to the number of component in the numerical vector that is assigned to any land cover parameters listed below. & - & 1, inf & 1 & sca & num \\ \hline
SoilRoughness & Roughness length of soil surface & mm & 0, 1000 & 10 & sca & num \\ \hline
SoilAlbVisDry & Ground albedo without snow in the visible � dry & - & 0, 1 & 0.2 & sca & num \\ \hline
SoilAlbNIRDry & Ground albedo without snow in the near infrared � dry & - & 0, 1 & 0.2 & sca & num \\ \hline
SoilAlbVisWet & Ground albedo without snow in the visible � saturated & - & 0, 1 & 0.2 & sca & num \\ \hline
SoilAlbNIRWet & Ground albedo without snow in the near infrared � saturated & - & 0, 1 & 0.2 & sca & num \\ \hline
SoilEmissiv & Soil emissivity & - & 0, 1 & 0.96 & sca & num \\ \hline
PointLandCoverType & Land Cover type of the simulation point & - &  & NA & vec & num \\ \hline
\caption{Table of soil surface parameters  (numeric)}
\label{soilsurface1d_numeric}
\end{longtable}
\end{center}


\clearpage
\begin{center}
\begin{longtable}{|p {6.5 cm}|p {4.5 cm}|p {3 cm}|p{3 cm}|p{1.5 cm}|p{1.5 cm}|p{2 cm}|}
\hline
\textbf{Keyword} & \textbf{Description} & \textbf{M. U.} & \textbf{range} & \textbf{Default Value} & \textbf{Scalar / Vector} & \textbf{Logical / Numeric} \\ \hline
\endfirsthead
\hline
\multicolumn{7}{| c |}{continued from previous page} \\
\hline
\textbf{Keyword} & \textbf{Description} & \textbf{M. U.} & \textbf{range} & \textbf{Default Value} & \textbf{Scalar / Vector} & \textbf{Logical / Numeric} \\ \hline
\endhead
\hline
\multicolumn{7}{| c |}{{continued on next page}}\\ 
\hline
\endfoot
\endlastfoot
\hline
LWinParameterization & Which formula for incoming longwave radiation:  1 (Brutsaert, 1975), 2 (Satterlund, 1979), 3 (Idso, 1981), 4(Idso+Hodges),  5 (Koenig-Langlo \& Augstein, 1994), 6 (Andreas \& Ackley, 1982), 7 (Konzelmann, 1994), 8 (Prata, 1996), 9 (Dilley 1998) &  & 1, 2, .., 9 & 9 & sca & opt \\ \hline
MoninObukhov & Atmospherical stability parameter: 1 stability and instability considered, 2 stability not considered, 3 instability not considered, 4 always neutrality &  &  & 1 & sca & num \\ \hline
Surroundings & Yes(1), No(0) & - &  & 0 & sca & opt \\ \hline
\caption{Table of surface energy flux parameters  (numeric)}
\label{surfaceenergyfluxes1d_numeric}
\end{longtable}
\end{center}


\clearpage
\begin{center}
\begin{longtable}{|p {6.5 cm}|p {4.5 cm}|p {3 cm}|p{3 cm}|p{1.5 cm}|p{1.5 cm}|p{2 cm}|}
\hline
\textbf{Keyword} & \textbf{Description} & \textbf{M. U.} & \textbf{range} & \textbf{Default Value} & \textbf{Scalar / Vector} & \textbf{Logical / Numeric} \\ \hline
\endfirsthead
\hline
\multicolumn{7}{| c |}{continued from previous page} \\
\hline
\textbf{Keyword} & \textbf{Description} & \textbf{M. U.} & \textbf{range} & \textbf{Default Value} & \textbf{Scalar / Vector} & \textbf{Logical / Numeric} \\ \hline
\endhead
\hline
\multicolumn{7}{| c |}{{continued on next page}}\\ 
\hline
\endfoot
\endlastfoot
\hline
Latitude & Average latitude of the basin, positive means north, negative means south (MANDATORY) & degree & -90, 90 & 45 & sca & num \\ \hline
Longitude & Average longitude of the basin, eastwards from 0 meridiane (MANDATORY) & degree & 0, 180 & 0 & sca & num \\ \hline
PointID & identification code for the point of simulation &  &  & NA & sca & num \\ \hline
CoordinatePointX & coordinate X if PixelCoordinates is 1, number of row of the matrix if PixelCoordinates is 0 & m (according to the geographical projection of the maps) &  & NA & vec & num \\ \hline
CoordinatePointY & coordinate Y if PixelCoordinates is 1, number of column of the matrix if PixelCoordinates is 1 & m (according to the geographical projection of the maps) &  & NA & vec & num \\ \hline
PointElevation & elevation of the point of simulation & m a.s.l. &  & NA & vec & num \\ \hline
PointSlope & Slope steepness of the simulation point & degree &  & NA & vec & num \\ \hline
PointAspect & Aspect of the simulation point & degree &  & NA & vec & num \\ \hline
PointSkyViewFactor & Sky View Factor of the simulation point & - &  & NA & vec & num \\ \hline
PointCurvatureNorthSouthDirection & N-S curvature of the simulation point & m$^{-1}$ &  & NA & vec & num \\ \hline
PointCurvatureWestEastDirection & W-E curvature of the simulation point & m$^{-1}$ &  & NA & vec & num \\ \hline
PointCurvatureNorthwest SoutheastDirection & N-W curvature of the simulation point & m$^{-1}$ &  & NA & vec & num \\ \hline
PointCurvatureNortheast SouthwestDirection & N-E curvature of the simulation point & m$^{-1}$ &  & NA & vec & num \\ \hline
PointDrainageLateralDistance & Lateral Drainage distance of the simulation point & m &  & NA & vec & num \\ \hline
PointLatitude & Latitude of the simulation point & degree &  & NA & sca & num \\ \hline
PointLongitude & Longitude of the simulation point & degree &  & NA & sca & num \\ \hline
\caption{Table of topographic parameters  (numeric)}
\label{topo1d_numeric}
\end{longtable}
\end{center}

\clearpage
\begin{center}
\begin{longtable}{|p {6.5 cm}|p {4.5 cm}|p {3 cm}|p{3 cm}|p{1.5 cm}|p{1.5 cm}|p{2 cm}|}
\hline
\textbf{Keyword} & \textbf{Description} & \textbf{M. U.} & \textbf{range} & \textbf{Default Value} & \textbf{Scalar / Vector} & \textbf{Logical / Numeric} \\ \hline
\endfirsthead
\hline
\multicolumn{7}{| c |}{continued from previous page} \\
\hline
\textbf{Keyword} & \textbf{Description} & \textbf{M. U.} & \textbf{range} & \textbf{Default Value} & \textbf{Scalar / Vector} & \textbf{Logical / Numeric} \\ \hline
\endhead
\hline
\multicolumn{7}{| c |}{{continued on next page}}\\ 
\hline
\endfoot
\endlastfoot
\hline
VegHeight & vegetation height & mm & 0, 20000 & 1000 & sca & num \\ \hline
LSAI & Leaf and Stem Area Index [$L^2/L^2$] & - & 0, 1 & 1 & sca & numeric \\ \hline
CanopyFraction & Canopy fraction [0: no canopy in the pixel, 1: pixel fully covered by canopy] & - & 0, 1 & 0 & sca & numeric \\ \hline
DecayCoeffCanopy & Decay coefficient of the eddy diffusivity profile in the canopy & - & 0, inf & 2.5 & sca & numeric \\ \hline
VegSnowBurying & Coefficient of the exponential snow burying of vegetation & - & 0, inf & 1 & sca & numeric \\ \hline
RootDepth & Root depth [mm] (it is used to calculate root\_fraction for each layer, it must be positive) & mm & 0, inf & 300 & sca & numeric \\ \hline
MinStomatalRes & Minimum stomatal resistance & s $m^{-1}$ & 0, inf & 60 & sca & numeric \\ \hline
VegReflectVis & Vegetation reflectivity in the visible & - & 0, 1 & 0.2 & sca & numeric \\ \hline
VegReflNIR & Vegetation reflectivity in the near infrared & - & 0, 1 & 0.2 & sca & numeric \\ \hline
VegTransVis & Vegetation transmissimity in the visible & - & 0, 1 & 0.2 & sca & numeric \\ \hline
VegTransNIR & Vegetation transmissimity in the near infrared & - & 0, 1 & 0.2 & sca & numeric \\ \hline
LeafAngles & Departure of leaf angles from a random distribution (1 horizontal, 0 random, -1 vertical) & - & -1, 0, 1 & 0 & sca & opt \\ \hline
CanDensSurface & Surface density of canopy & kg m$^{-2}$ LSAI$^{-1}$ & 0, inf & 2 & sca & num \\ \hline
\caption{Table of vegetation parameters  (numeric)}
\label{vege1d_numeric}
\end{longtable}
\end{center}






\section{3D INPUT NUMERIC}

\begin{center}
\begin{longtable}{|p {6.5 cm}|p {4.5 cm}|p {3 cm}|p{3 cm}|p{1.5 cm}|p{1.5 cm}|p{2 cm}|}
\hline
\textbf{Keyword} & \textbf{Description} & \textbf{M. U.} & \textbf{range} & \textbf{Default Value} & \textbf{Scalar / Vector} & \textbf{Logical / Numeric} \\ \hline
\endfirsthead
\hline
\multicolumn{7}{| c |}{continued from previous page} \\
\hline
\textbf{Keyword} & \textbf{Description} & \textbf{M. U.} & \textbf{range} & \textbf{Default Value} & \textbf{Scalar / Vector} & \textbf{Logical / Numeric} \\ \hline
\endhead
\hline
\multicolumn{7}{| c |}{{continued on next page}}\\ 
\hline
\endfoot
\endlastfoot
\hline
Vmin & MINIMUM WIND VELOCITY (m/s) [wind speeds that are too low create numerical problems] & m s$^{-1}$ & 0, 100 & 0.5 & sca & num \\ \hline
RHmin & MINIMUM RELATIVE HUMIDITY (\%) [relative humidities that are too low create numerical problems] & \% & 0, 100 & 10 & sca & num \\ \hline
LapseRateTemp & Lapse rate of air temperature with elevation & $^\circ$C km$^{-1}$ &  & NA & vec & num \\ \hline
LapseRateDewTemp & Lapse rate of dew temperature with elevation & $^\circ$C km$^{-1}$ &  & NA & vec & num \\ \hline
LapseRatePrec & Lapse rate of precipitation with elevation & mm h$^{-1}$ km$^{-1}$ &  & NA & vec & num \\ \hline
\caption{Table of meteorological forcing (meteo data - numeric)}
\label{meteo_data}
\end{longtable}
\end{center}

\clearpage

\begin{center}
\begin{longtable}{|p {6.5 cm}|p {4.5 cm}|p {3 cm}|p{3 cm}|p{1.5 cm}|p{1.5 cm}|p{2 cm}|}
\hline
\textbf{Keyword} & \textbf{Description} & \textbf{M. U.} & \textbf{range} & \textbf{Default Value} & \textbf{Scalar / Vector} & \textbf{Logical / Numeric} \\ \hline
\endfirsthead
\hline
\multicolumn{7}{| c |}{continued from previous page} \\
\hline
\textbf{Keyword} & \textbf{Description} & \textbf{M. U.} & \textbf{range} & \textbf{Default Value} & \textbf{Scalar / Vector} & \textbf{Logical / Numeric} \\ \hline
\endhead
\hline
\multicolumn{7}{| c |}{{continued on next page}}\\ 
\hline
\endfoot
\endlastfoot
\hline
MeteoStationsID & Identification code for the meteo station & - &  & NA & vec & num \\ \hline
NumberOfMeteoStations & number of soil Meteo Stations (is calculated after the number of components of the vector NumberOfMeteoStations) & - &  & 1 & sca & num \\ \hline
MeteoStationCoordinateX & coordinate X of the meteo station & m &  & NA & vec & num \\ \hline
MeteoStationCoordinateY & coordinate Y of the meteo station & m &  & NA & vec & num \\ \hline
MeteoStationLatitude & Latitude of the meteo station & degree &  & Latitude & vec & num \\ \hline
MeteoStationLongitude & Longitude of the meteo station & degree &  & Longitude & vec & num \\ \hline
MeteoStationElevation & Latitude of the meteo station & m a.s.l. &  & 0 & vec & num \\ \hline
MeteoStationSkyViewFactor & Sky view factor of the meteo station & - &  & 1 & vec & num \\ \hline
MeteoStationStandardTime & Standard time to which the meteo records are referred to (difference respect UMT, in hours): GMT + x [h] & h &  & Standard Time Simulation & vec & num \\ \hline
MeteoStationWindVelocitySensorHeight & Height of the wind velocity sensor of the meteo station & m a.g.l &  & 10 & vec & num \\ \hline
MeteoStationTemperatureSensorHeight & Height of the air temperature sensor of the meteo station & m a.g.l &  & 2 & vec & num \\ \hline
\caption{Table of meteorological forcing (meteo station - numeric)}
\label{meteo_station}
\end{longtable}
\end{center}





\begin{center}
\begin{longtable}{|p {6.5 cm}|p {4.5 cm}|p {3 cm}|p{3 cm}|p{1.5 cm}|p{1.5 cm}|p{2 cm}|}
\hline
\textbf{Keyword} & \textbf{Description} & \textbf{M. U.} & \textbf{range} & \textbf{Default Value} & \textbf{Scalar / Vector} & \textbf{Logical / Numeric} \\ \hline
\endfirsthead
\hline
\multicolumn{7}{| c |}{continued from previous page} \\
\hline
\textbf{Keyword} & \textbf{Description} & \textbf{M. U.} & \textbf{range} & \textbf{Default Value} & \textbf{Scalar / Vector} & \textbf{Logical / Numeric} \\ \hline
\endhead
\hline
\multicolumn{7}{| c |}{{continued on next page}}\\ 
\hline
\endfoot
\endlastfoot
\hline
FreeDrainageAtLateralBorder & Boundary condition on Richards' equation at the lateral border (1: free drainage, 0: no flux) & - & 0,1 & 1 & sca & num \\ \hline
FreeDrainageAtBottom & Boundary condition on Richards' equation at the bottom border (1: free drainage, 0: no flux) & - & 0,1 & 0 & sca & num \\ \hline
ZeroTempAmplitDepth & Depth [mm] at which the annual temperature remains constant. It is used as the bottom boundary condition of the heat equation. The Zero flux condition can be assigned setting this parameter at a very high value & mm &  & 1.00E+20 & sca & num \\ \hline
ZeroTempAmplitTemp & Temperature at the depth assigned above & $^\circ$C &  & 20 & sca & num \\ \hline
BottomBoundaryHeatFlux & Incoming heat flux at the bottom boundary of the soil domain (geothermal heat flux) & W m$^{-2}$ &  & 0 & sca & num \\ \hline
\caption{Table of boundary condition  (numeric)}
\label{BC}
\end{longtable}
\end{center}

\clearpage
\begin{center}
\begin{longtable}{|p {6.5 cm}|p {4.5 cm}|p {3 cm}|p{3 cm}|p{1.5 cm}|p{1.5 cm}|p{2 cm}|}
\hline
\textbf{Keyword} & \textbf{Description} & \textbf{M. U.} & \textbf{range} & \textbf{Default Value} & \textbf{Scalar / Vector} & \textbf{Logical / Numeric} \\ \hline
\endfirsthead
\hline
\multicolumn{7}{| c |}{continued from previous page} \\
\hline
\textbf{Keyword} & \textbf{Description} & \textbf{M. U.} & \textbf{range} & \textbf{Default Value} & \textbf{Scalar / Vector} & \textbf{Logical / Numeric} \\ \hline
\endhead
\hline
\multicolumn{7}{| c |}{{continued on next page}}\\ 
\hline
\endfoot
\endlastfoot
\hline
RatioChannelWidthPixelWidth & Fraction of channel width in the pixel width & - &  & 0.1 & sca & num \\ \hline
ChannelDepression & Depression of the channel bed with respect to the neighboring slopes [mm] - this is used to change between free and submerged weir flow model to represent to surface flow to the channel & mm &  & 500 & sca & num \\ \hline
MinSupWaterDepthLand & minimum surface water depth on the earth below which the Courant condition is not applied & mm &  & 1 & sca & num \\ \hline
MinSupWaterDepthChannel & minimum surface water depth on the channel below which the Courant condition is not applied & mm &  & 1 & sca & num \\ \hline
\caption{Table of channel flow parameters  (numeric)}
\label{channel_flow_numeric}
\end{longtable}
\end{center}

\clearpage
\begin{center}
\begin{longtable}{|p {6.5 cm}|p {4.5 cm}|p {3 cm}|p{3 cm}|p{1.5 cm}|p{1.5 cm}|p{2 cm}|}
\hline
\textbf{Keyword} & \textbf{Description} & \textbf{M. U.} & \textbf{range} & \textbf{Default Value} & \textbf{Scalar / Vector} & \textbf{Logical / Numeric} \\ \hline
\endfirsthead
\hline
\multicolumn{7}{| c |}{continued from previous page} \\
\hline
\textbf{Keyword} & \textbf{Description} & \textbf{M. U.} & \textbf{range} & \textbf{Default Value} & \textbf{Scalar / Vector} & \textbf{Logical / Numeric} \\ \hline
\endhead
\hline
\multicolumn{7}{| c |}{{continued on next page}}\\ 
\hline
\endfoot
\endlastfoot
\hline
FlagSkyViewFactor & If not present, the sky view factor can be calculated (=1), or just be considered only equal to 1 (=0) & - & 0, 1 & 0 & sca & opt \\ \hline
InitDateDDMMYYYYhhmm & Date and time of the simulation start in date12 format (MANDATORY) & format DDMMYYhhmm & 01/01/1800 00:00, 01/01/2500 00:00 & NA & vec & str \\ \hline
EndDateDDMMYYYYhhmm & Date and time of the simulation start in date12 format (MANDATORY) & format DDMMYYhhmm & 01/01/1800 00:00, 01/01/2500 00:00 & NA & vec & str \\ \hline
NumSimulationTimes & How many times the simulation is run (if >1, it uses the final condition as initial conditions of the new simulation) & - & 0, inf & 1 & vec & num \\ \hline
StandardTimeSimulation & Standard time to which all the output data are referred (difference respect UMT, in hours): GMT + x [h] & h & 0, 12 & 0 & sca & num \\ \hline
PointSim & Point simulation (=1), distributed simulation (=0) & - & 0, 1 & 0 & sca & opt \\ \hline
RecoverSim & Simulation recovered (=number of saving point you want to start from), otherwise (=0) & - & 0, 1 & 0 & sca & opt \\ \hline
WaterBalance & Activate water balance (Yes=1, No=0) & - &  & 0 & sca & opt \\ \hline
EnergyBalance & Activate energy balance (Yes=1, No=0) &  &  & 0 & sca & opt \\ \hline
PixelCoordinates & Write 1 IF ALL point COORDINATES ARE IN FORMAT (EAST,NORTH) in meters, Or 0 IF IN FORMAT ROW AND COLUMS (r,c) of the dem map & - &  & 1 & sca & opt \\ \hline
SavingPoints &  & - &  & NA & vec & num \\ \hline
SoilLayerTypes & Number of types of soil types, corresponding to different soil stratigraphies & - &  & 1 & sca & num \\ \hline
DefaultSoilTypeLand & given a multiple number of type of soil, this relates to the default given to the land type type & - &  & 1 & sca & num \\ \hline
DefaultSoilTypeChannel & given a multiple number of type of soil, this relates to the default given to the channel type & - &  & 1 & sca & num \\ \hline
\caption{Table of general parameters  (numeric)}
\label{general_numeric}
\end{longtable}
\end{center}


\clearpage
\begin{center}
\begin{longtable}{|p {6.5 cm}|p {4.5 cm}|p {3 cm}|p{3 cm}|p{1.5 cm}|p{1.5 cm}|p{2 cm}|}
\hline
\textbf{Keyword} & \textbf{Description} & \textbf{M. U.} & \textbf{range} & \textbf{Default Value} & \textbf{Scalar / Vector} & \textbf{Logical / Numeric} \\ \hline
\endfirsthead
\hline
\multicolumn{7}{| c |}{continued from previous page} \\
\hline
\textbf{Keyword} & \textbf{Description} & \textbf{M. U.} & \textbf{range} & \textbf{Default Value} & \textbf{Scalar / Vector} & \textbf{Logical / Numeric} \\ \hline
\endhead
\hline
\multicolumn{7}{| c |}{{continued on next page}}\\ 
\hline
\endfoot
\endlastfoot
\hline
IrriducibleWatSatGlacier & IRREDUCIBLE WATER SATURATION FOR GLACIER & - &  & 0.02 & sca & num \\ \hline
MaxWaterEqGlacLayerContent & maximum water equivalent admitted in a snow layer &  &  & 5 & sca & num \\ \hline
MaxGlacLayerNumber & maximum layers of snow to use (suggested $>$5) &  &  & 0 & sca & num \\ \hline
ThickerGlacLayers & Layer numbers that can become thicker than admitted by the threshold given by MaxGlacLayerNumber (from the bottom up). They can be more than one &  &  & Max Glac Layer Number/2 & vec & num \\ \hline
\caption{Table of glacier parameters  (numeric)}
\label{glacier_numeric}
\end{longtable}
\end{center}


\clearpage
\begin{center}
\begin{longtable}{|p {6.5 cm}|p {4.5 cm}|p {3 cm}|p{3 cm}|p{1.5 cm}|p{1.5 cm}|p{2 cm}|}
\hline
\textbf{Keyword} & \textbf{Description} & \textbf{M. U.} & \textbf{range} & \textbf{Default Value} & \textbf{Scalar / Vector} & \textbf{Logical / Numeric} \\ \hline
\endfirsthead
\hline
\multicolumn{7}{| c |}{continued from previous page} \\
\hline
\textbf{Keyword} & \textbf{Description} & \textbf{M. U.} & \textbf{range} & \textbf{Default Value} & \textbf{Scalar / Vector} & \textbf{Logical / Numeric} \\ \hline
\endhead
\hline
\multicolumn{7}{| c |}{{continued on next page}}\\ 
\hline
\endfoot
\endlastfoot
\hline
InitSWE & Initial snow water equivalent (SWE) - used if no snow map is given & kg m$^{-2}$ &  & 0 & sca & num \\ \hline
InitSnowDensity & INITIAL SNOW DENSITY - uniform with depth & kg m$^{-3}$ &  & 200 & sca & num \\ \hline
InitSnowTemp & INITIAL SNOW TEMPERATURE - uniform with depth & $^\circ$C &  & -3 & sca & num \\ \hline
InitSnowAge & INITIAL SNOW AGE & days &  & 0 & sca & num \\ \hline
InitGlacierDepth & GLACIER DEPTH - used if no snow map is given & mm &  & 0 & sca & num \\ \hline
InitGlacierDensity & INITIAL GLACIER DENSITY - uniform with depth & kg m$^{-3}$ &  & 800 & sca & num \\ \hline
InitGlacierTemp & INITIAL GLACIER TEMPERATURE - uniform with depth & $^\circ$C &  & -3 & sca & num \\ \hline
InitWaterTableHeightOverTopoSurface & initial condition on water table depth (positive downwards from ground surface). Used if InitSoilPressure is void & mm &  & 0 & sca & num \\ \hline
InitSoilPressure &  & mm &  & NA & vec & num \\ \hline
InitSoilTemp &  & $^\circ$C &  & 5 & vec & num \\ \hline
InitSoilPressureBedrock &  & mm &  & NA & vec & num \\ \hline
InitSoilTempBedrock &  & $^\circ$C &  & 5 & vec & num \\ \hline
\caption{Table of initial condition  (numeric)}
\label{IC_numeric}
\end{longtable}
\end{center}



\clearpage
\begin{center}
\begin{longtable}{|p {6.5 cm}|p {4.5 cm}|p {3 cm}|p{3 cm}|p{1.5 cm}|p{1.5 cm}|p{2 cm}|}
\hline
\textbf{Keyword} & \textbf{Description} & \textbf{M. U.} & \textbf{range} & \textbf{Default Value} & \textbf{Scalar / Vector} & \textbf{Logical / Numeric} \\ \hline
\endfirsthead
\hline
\multicolumn{7}{| c |}{continued from previous page} \\
\hline
\textbf{Keyword} & \textbf{Description} & \textbf{M. U.} & \textbf{range} & \textbf{Default Value} & \textbf{Scalar / Vector} & \textbf{Logical / Numeric} \\ \hline
\endhead
\hline
\multicolumn{7}{| c |}{{continued on next page}}\\ 
\hline
\endfoot
\endlastfoot
\hline
Iobsint & Let Micromet determine an appropriate "radius of influence" (=0), or define the "radius of influence" you want the model to use (=1). 1=use obs interval below, 0=use model generated interval. & - &  & 1 & sca & num \\ \hline
Dn & The "radius of influence" or "observation interval" you want the model to use for the interpolation.  In units of deltax, deltay. & - &  & 1 & sca & num \\ \hline
SlopeWeight & Weight assigned to the slope (as tangent when it is < 1) in the spatial distribution of the wind speed & - &  & 0 & sca & num \\ \hline
CurvatureWeight & Weight assigned to the curvature (as second derivative of the topographic surface) in the spatial distribution of the wind speed & - &  & 0 & sca & num \\ \hline
SlopeWeightD &  &  &  & 0 & sca & num \\ \hline
CurvatureWeightD &  &  &  & 0 & sca & num \\ \hline
SlopeWeightI &  &  &  & 0 & sca & num \\ \hline
CurvatureWeightI &  &  &  & 0 & sca & num \\ \hline
\caption{Table of spatial distribution method parameters  (numeric)}
\label{meteodistr_numeric}
\end{longtable}
\end{center}

\clearpage
\begin{center}
\begin{longtable}{|p {6.5 cm}|p {4.5 cm}|p {3 cm}|p{3 cm}|p{1.5 cm}|p{1.5 cm}|p{2 cm}|}
\hline
\textbf{Keyword} & \textbf{Description} & \textbf{M. U.} & \textbf{range} & \textbf{Default Value} & \textbf{Scalar / Vector} & \textbf{Logical / Numeric} \\ \hline
\endfirsthead
\hline
\multicolumn{7}{| c |}{continued from previous page} \\
\hline
\textbf{Keyword} & \textbf{Description} & \textbf{M. U.} & \textbf{range} & \textbf{Default Value} & \textbf{Scalar / Vector} & \textbf{Logical / Numeric} \\ \hline
\endhead
\hline
\multicolumn{7}{| c |}{{continued on next page}}\\ 
\hline
\endfoot
\endlastfoot
\hline
TimeStepEnergyAndWater & THE INTEGRATION time step [s] for energy and water balance equation (MANDATORY) & s & 0, inf & NA & vec & num \\ \hline
RichardTol & Absolute Tolerance for the integration of Richards' equation (on the Euclidean norm of residuals)  & mm & 1E-20, inf & 1.00E-08 & sca & num \\ \hline
RichardMaxIter & Max iterations for the integration of Richards' equation & - & 1, inf & 100 & sca & num \\ \hline
RichardInitForc & Initial forcing term of Newton method  & - &  & 0.01 & sca & num \\ \hline
MinTimeStepSupFlow & minimum integration time step for the integration of the surface runoff equation &  &  & 0.01 & sca & num \\ \hline
HeatEqTol & Max norm of the residuals for heat equation & J m$^{-2}$ &  & 1.00E-04 & sca & num \\ \hline
HeatEqMaxIter & Max number of iterations for heat equation & - &  & 500 & sca & num \\ \hline
CanopyMaxIter & Max number of iterations for canopy (heat) equation &  &  & 3 & sca & num \\ \hline
BusingerMaxIter & Max number of iterations for Monin-Obulhov stability algorithm (Businger parameterization) & - &  & 5 & sca & num \\ \hline
TsMaxIter & Max number of iterations for the calculation of canopy air temperature & - &  & 2 & sca & num \\ \hline
LocMaxIter & Max number of iterations for the calculation of the within-canopy Monin-Obukhov length & - &  & 3 & sca & num \\ \hline
CanopyStabCorrection & Use of the stability corrections within canopy (=1), otherwise (=0) & - &  & 1 & sca & num \\ \hline
\caption{Table of numeric parameters  (numeric)}
\label{numeric_numeric}
\end{longtable}
\end{center}



\clearpage
\begin{center}
\begin{longtable}{|p {6.5 cm}|p {4.5 cm}|p {3 cm}|p{3 cm}|p{1.5 cm}|p{1.5 cm}|p{2 cm}|}
\hline
\textbf{Keyword} & \textbf{Description} & \textbf{M. U.} & \textbf{range} & \textbf{Default Value} & \textbf{Scalar / Vector} & \textbf{Logical / Numeric} \\ \hline
\endfirsthead
\hline
\multicolumn{7}{| c |}{continued from previous page} \\
\hline
\textbf{Keyword} & \textbf{Description} & \textbf{M. U.} & \textbf{range} & \textbf{Default Value} & \textbf{Scalar / Vector} & \textbf{Logical / Numeric} \\ \hline
\endhead
\hline
\multicolumn{7}{| c |}{{continued on next page}}\\ 
\hline
\endfoot
\endlastfoot
\hline
ThetaResBedrock &  & - &  & 0.05 & vec & num \\ \hline
WiltingPointBedrock &  & - &  & 0.15 & vec & num \\ \hline
FieldCapacityBedrock &  & - &  & 0.25 & vec & num \\ \hline
ThetaSatBedrock &  & - &  & 0.5 & vec & num \\ \hline
AlphaVanGenuchtenBedrock &  & mm$^{-1}$ &  & 0.004 & vec & num \\ \hline
NVanGenuchtenBedrock &  & - &  & 1.3 & vec & num \\ \hline
VMualemBedrock &  & - &  & 0.5 & vec & num \\ \hline
ThermalConductivitySoilSolidsBedrock & thermal conductivity of the bedrock & W m$^{-1}$ K$^{-1}$ &  & 2.5 & vec & num \\ \hline
ThermalCapacitySoilSolidsBedrock & thermal capacity of the bedrock & J m$^{-3}$ K$^{-1}$ &  & 1.00E+06 & vec & num \\ \hline
SpecificStorativityBedrock &  & mm$^{-1}$ &  & 1.00E-07 & vec & num \\ \hline
\caption{Table of rock parameters  (numeric)}
\label{rock_numeric}
\end{longtable}
\end{center}


\clearpage
\begin{center}
\begin{longtable}{|p {6.5 cm}|p {4.5 cm}|p {3 cm}|p{3 cm}|p{1.5 cm}|p{1.5 cm}|p{2 cm}|}
\hline
\textbf{Keyword} & \textbf{Description} & \textbf{M. U.} & \textbf{range} & \textbf{Default Value} & \textbf{Scalar / Vector} & \textbf{Logical / Numeric} \\ \hline
\endfirsthead
\hline
\multicolumn{7}{| c |}{continued from previous page} \\
\hline
\textbf{Keyword} & \textbf{Description} & \textbf{M. U.} & \textbf{range} & \textbf{Default Value} & \textbf{Scalar / Vector} & \textbf{Logical / Numeric} \\ \hline
\endhead
\hline
\multicolumn{7}{| c |}{{continued on next page}}\\ 
\hline
\endfoot
\endlastfoot
\hline
SurFlowResLand & Coefficient ($C_m$) of of the law of uniform motion on the surface $(v_sup=C_m*h_sup^SurFlowResExp*i_{DD}^{0.5})$ [$m^{\frac{1-SurFlowResExp}{s}}$, SurFlowResExp defined later & m$^{\frac{1-SurFlowResExp}{s}}$ & 0.01, 5.0 & 0.5 & sca & num \\ \hline
SurFlowResExp & Exponent (gamma) of the law of uniform motion on the surface $v=C_m (h_{sup}^{\gamma})  i^{0.5}$ & - &  & 0.67 & sca & num \\ \hline
ThresWaterDepthLandDown & Threshold on $h_{sup}$ below which $C_m$ is 0 (water does not flow on the surface)  & mm &  & 0 & sca & num \\ \hline
ThresWaterDepthLandUp & Threshold on $h_{sup}$ above which $C_m$ is independent from $h_{sup}$ (= fully developed turbulence) & mm &  & 50 & sca & num \\ \hline
SurFlowResChannel & Resistance coefficient for the channel flow (the same gamma for land surface flow is used) & $m^{\frac{1-SurFlowResExp}{s}}$ &  & 20 & sca & num \\ \hline
ThresWaterDepthChannelUp & Threshold on $h_{sup}$ [mm] above which $C_m$ is independent from $h_{sup}$ (= fully developed turbulence).  & mm &  & 50 & sca & num \\ \hline
PointDepthFreeSurface & depth of the trench that simulates the drainage of a soil column through a weir. The deeper the trench, the higher the drainage. Valid in 1D simulations & mm &  & NA & vec & num \\ \hline
\caption{Table of runoff parameters  (numeric)}
\label{meteodistr_numeric}
\end{longtable}
\end{center}

\clearpage
\begin{center}
\begin{longtable}{|p {6.5 cm}|p {4.5 cm}|p {3 cm}|p{3 cm}|p{1.5 cm}|p{1.5 cm}|p{2 cm}|}
\hline
\textbf{Keyword} & \textbf{Description} & \textbf{M. U.} & \textbf{range} & \textbf{Default Value} & \textbf{Scalar / Vector} & \textbf{Logical / Numeric} \\ \hline
\endfirsthead
\hline
\multicolumn{7}{| c |}{continued from previous page} \\
\hline
\textbf{Keyword} & \textbf{Description} & \textbf{M. U.} & \textbf{range} & \textbf{Default Value} & \textbf{Scalar / Vector} & \textbf{Logical / Numeric} \\ \hline
\endhead
\hline
\multicolumn{7}{| c |}{{continued on next page}}\\ 
\hline
\endfoot
\endlastfoot
\hline
ThresSnowSoilRough & Threshold on snow depth to change roughness to snow roughness values with d0 set at 0, for bare soil fraction & mm & 0, 1000 & 10 & sca & num \\ \hline
ThresSnowVegUp & Threshold on snow depth above which the roughness is snow roughness, for vegetation fraction & mm & 0, 20000 & 1000 & sca & num \\ \hline
ThresSnowVegDown & Threshold on snow depth below which the roughness is vegetation roughness, for vegetation fraction & mm & 0, 20000 & 1000 & sca & num \\ \hline
RoughElemXUnitArea & Number of roughness elements (=vegetation) per unit area - used only for blowing snow subroutines & Number m$^{-2}$ & 0, inf & 0 & sca & num \\ \hline
RoughElemDiam & Diameter [mm] of the roughness elements (=vegetation) - used only for blowing snow subroutines & mm & 0, inf & 50 & sca & num \\ \hline
AlphaSnow & Alpha (SNTHERM parameter) for the freezing characteristic soil for snow, the bigger, the steeper the curve around 0 degrees & - &  & 1.00E+05 & sca & num \\ \hline
ThresTempRain & DEW or AIR TEMPERATURE ABOVE WHICH ALL PRECIPITATION IS RAIN & $^\circ$C &  & 3 & sca & num \\ \hline
ThresTempSnow & DEW or AIR TEMPERATURE BELOW WHICH ALL PRECIPITATION IS SNOW & $^\circ$C &  & -1 & sca & num \\ \hline
DewTempOrNormTemp & Use dew temperature (1) or air temperature (0) to discriminate between snowfall and rainfall & - & 1 or 0 & 0 & sca & opt \\ \hline
AlbExtParSnow & ALBEDO EXTINCTION PARAMETER - if snow depth < aep, albedo is interpolated between soil and snow & mm &  & 10 & sca & num \\ \hline
FreshSnowReflVis & VISIBLE BAND REFLECTANCE OF fresh SNOW  & - &  & 0.9 & sca & num \\ \hline
FreshSnowReflNIR & near INFRARED BAND REFLECTANCE OF fresh SNOW & - &  & 0.65 & sca & num \\ \hline
IrriducibleWatSatSnow & IRREDUCIBLE WATER SATURATION - from Colbeck (0.02 - 0.07). It is the ratio of the capillarity-hold water to ice content in the snow & - & 0.02 � 0.07 & 0.02 & sca & num \\ \hline
SnowEmissiv & SNOW LONGWAVE EMISSIVITY [-] & - &  & 0.98 & sca & num \\ \hline
SnowRoughness & Roughness length over snow (mm) & mm &  & 0.1 & sca & num \\ \hline
SnowCorrFactor & correction factor on fresh snow accumulation &  &  & 1 & sca & num \\ \hline
RainCorrFactor & correction factor precipitated rain & - &  & 1 & sca & num \\ \hline
MaxSnowPorosity & MAXIMUM SNOW POROSITY ALLOWED (-). This parameter prevents excessive snow densification & - &  & 0.7 & sca & num \\ \hline
DrySnowDefRate & SNOW COMPACTION (\% per hour) DUE TO DESTRUCTIVE METAMORPHISM for SNOW DENSITY<snow\_density\_ cutoff and DRY SNOW  & - &  & 1 & sca & num \\ \hline
SnowDensityCutoff & SNOW DENSITY CUTOFF (kg m$^{-3}$) TO CHANGE SNOW DEFORMATION RATE & kg m$^{-3}$ &  & 100 & sca & num \\ \hline
WetSnowDefRate & ENHANCEMENT FACTOR IN PRESENCE OF WET SNOW & - &  & 1.5 & sca & num \\ \hline
SnowViscosity & SNOW VISCOSITY COEFFICIENT (kg s m$^{-2}$) at T=0 C and snow density=0 & N s m$^{-2}$ &  & 1.00E+06 & sca & num \\ \hline
FetchUp & SCALING FETCH in case snow wind transport in increasing [m] & m &  & 1000 & sca & num \\ \hline
FetchDown & SCALING FETCH in case snow wind transport in decreasing [m] & m &  & 100 & sca & num \\ \hline
BlowingSnowSoftLayerIceContent & Snow depth (in ice water equivalent), the averaged density of which is used for blowing snow wind thresholds & kg m$^{-2}$ &  & 0 & sca & num \\ \hline
TimeStepBlowingSnow & Time step [s] at which the Prairie Blowing Snow Model is run & s &  & TimeStep Energy And Water & sca & num \\ \hline
SnowSMIN & MINIMUM SLOPE [degree] TO ADJUST PRECIPITATION REDUCTION & degree &  & 30 & sca & num \\ \hline
SnowSMAX & MAXIMUM SLOPE [degree] TO ADJUST PRECIPITATION REDUCTION & degree &  & 80 & sca & num \\ \hline
SnowCURV & SHAPE PARAMETER FOR PRECIPITATION REDUCTION (if <0 the adjustment is not applied) & - &  & -200 & sca & num \\ \hline
MaxWaterEqSnowLayerContent & maximum water equivalent admitted in a snow layer & kg m$^{-2}$ &  & 5 & sca & num \\ \hline
MaxSnowLayerNumber & maximum layers of snow to use (suggested $>$10) &  &  & 10 & sca & num \\ \hline
ThickerSnowLayers & Layer numbers that can become thicker than admitted by the threshold given by MaxSnow Layer Number (from the bottom up). They can be more than one &  &  & MaxSnow Layer Number/2 & vec & num \\ \hline
BlowingSnow & Activate blowing snow module (yes=1, no=0) & - &  & 0 & sca & opt \\ \hline
PointMaxSWE & Max snow water equivalent that can be reached in the simulation point & kg m$^{-2}$ &  & NA & vec & num \\ \hline
SnowAgingCoeffVis & reflectance of the new snow in the visible wave length & - &  & 0.2 & sca & num \\ \hline
SnowAgingCoeffNIR & reflectance of the new snow in the infrared wave length & - &  & 0.5 & sca & num \\ \hline
\caption{Table of snow parameters  (numeric)}
\label{snow_numeric}
\end{longtable}
\end{center}



\clearpage
\begin{center}
\begin{longtable}{|p {6.5 cm}|p {4.5 cm}|p {3 cm}|p{3 cm}|p{1.5 cm}|p{1.5 cm}|p{2 cm}|}
\hline
\textbf{Keyword} & \textbf{Description} & \textbf{M. U.} & \textbf{range} & \textbf{Default Value} & \textbf{Scalar / Vector} & \textbf{Logical / Numeric} \\ \hline
\endfirsthead
\hline
\multicolumn{7}{| c |}{continued from previous page} \\
\hline
\textbf{Keyword} & \textbf{Description} & \textbf{M. U.} & \textbf{range} & \textbf{Default Value} & \textbf{Scalar / Vector} & \textbf{Logical / Numeric} \\ \hline
\endhead
\hline
\multicolumn{7}{| c |}{{continued on next page}}\\ 
\hline
\endfoot
\endlastfoot
\hline
FrozenSoilHydrCondReduction & Reduction factor of the hydraulic conductivity in partially frozen soil ($K=K_{no\_ice}*10^{impedence Q}$, where Q is the ice ratio & - & 0, 7 & 2 & sca & num \\ \hline
PointSoilType & Soil type of the simulation point & - &  & NA & vec & num \\ \hline
SoilLayerThicknesses & vector defining the thickness of the various soil layers. If not present, a column of 5 layers 100 mm thick will be assumed & mm &  & 100 & vec & num \\ \hline
SoilLayerNumber & number of soil layers (is calculated after the number of components of the vector SoilLayerNumber) & - &  & 5 & sca & num \\ \hline
NormalHydrConductivity &  & mm s$^{-1}$ &  & 1.00E-04 & vec & num \\ \hline
LateralHydrConductivity &  & mm s$^{-1}$ &  & 1.00E-04 & vec & num \\ \hline
ThetaRes &  & - &  & 0.05 & vec & num \\ \hline
WiltingPoint &  & - &  & 0.15 & vec & num \\ \hline
FieldCapacity &  & - &  & 0.25 & vec & num \\ \hline
ThetaSat &  & - &  & 0.5 & vec & num \\ \hline
AlphaVanGenuchten &  & mm$^{-1}$ &  & 0.004 & vec & num \\ \hline
NVanGenuchten &  & - &  & 1.3 & vec & num \\ \hline
VMualem &  & - &  & 0.5 & vec & num \\ \hline
ThermalConductivitySoilSolids & thermal conductivity of the soil particles & W m$^{-1}$ K$^{-1}$ &  & 2.5 & vec & num \\ \hline
ThermalCapacitySoilSolids & thermal capacity of the soil particles & J m$^{-3}$ K$^{-1}$ &  & 1.00E+06 & vec & num \\ \hline
SpecificStorativity &  & mm$^{-1}$ &  & 1.00E-07 & vec & num \\ \hline
NormalHydrConductivityBedrock &  & mm s$^{-1}$ &  & 1.00E-04 & vec & num \\ \hline
LateralHydrConductivityBedrock &  & mm s$^{-1}$ &  & 1.00E-04 & vec & num \\ \hline
\caption{Table of soil parameters  (numeric)}
\label{soil_numeric}
\end{longtable}
\end{center}

\clearpage
\begin{center}
\begin{longtable}{|p {6.5 cm}|p {4.5 cm}|p {3 cm}|p{3 cm}|p{1.5 cm}|p{1.5 cm}|p{2 cm}|}
\hline
\textbf{Keyword} & \textbf{Description} & \textbf{M. U.} & \textbf{range} & \textbf{Default Value} & \textbf{Scalar / Vector} & \textbf{Logical / Numeric} \\ \hline
\endfirsthead
\hline
\multicolumn{7}{| c |}{continued from previous page} \\
\hline
\textbf{Keyword} & \textbf{Description} & \textbf{M. U.} & \textbf{range} & \textbf{Default Value} & \textbf{Scalar / Vector} & \textbf{Logical / Numeric} \\ \hline
\endhead
\hline
\multicolumn{7}{| c |}{{continued on next page}}\\ 
\hline
\endfoot
\endlastfoot
\hline
NumLandCoverTypes & Number of Classes of land cover. Each land cover type corresponds to a particular land-cover state, described by a specific set of values of the parameters listed below. Each set of land cover parameters will be distributively assigned according to the land cover map, which relates each pixel with a land cover type number. This number corresponds to the number of component in the numerical vector that is assigned to any land cover parameters listed below. & - & 1, inf & 1 & sca & num \\ \hline
SoilRoughness & Roughness length of soil surface & mm & 0, 1000 & 10 & sca & num \\ \hline
SoilAlbVisDry & Ground albedo without snow in the visible - dry & - & 0, 1 & 0.2 & sca & num \\ \hline
SoilAlbNIRDry & Ground albedo without snow in the near infrared - dry & - & 0, 1 & 0.2 & sca & num \\ \hline
SoilAlbVisWet & Ground albedo without snow in the visible - saturated & - & 0, 1 & 0.2 & sca & num \\ \hline
SoilAlbNIRWet & Ground albedo without snow in the near infrared - saturated & - & 0, 1 & 0.2 & sca & num \\ \hline
SoilEmissiv & Soil emissivity & - & 0, 1 & 0.96 & sca & num \\ \hline
PointLandCoverType & Land Cover type of the simulation point & - &  & NA & vec & num \\ \hline
\caption{Table of soil surface parameters  (numeric)}
\label{soilsurface_numeric}
\end{longtable}
\end{center}


\clearpage
\begin{center}
\begin{longtable}{|p {6.5 cm}|p {4.5 cm}|p {3 cm}|p{3 cm}|p{1.5 cm}|p{1.5 cm}|p{2 cm}|}
\hline
\textbf{Keyword} & \textbf{Description} & \textbf{M. U.} & \textbf{range} & \textbf{Default Value} & \textbf{Scalar / Vector} & \textbf{Logical / Numeric} \\ \hline
\endfirsthead
\hline
\multicolumn{7}{| c |}{continued from previous page} \\
\hline
\textbf{Keyword} & \textbf{Description} & \textbf{M. U.} & \textbf{range} & \textbf{Default Value} & \textbf{Scalar / Vector} & \textbf{Logical / Numeric} \\ \hline
\endhead
\hline
\multicolumn{7}{| c |}{{continued on next page}}\\ 
\hline
\endfoot
\endlastfoot
\hline
LWinParameterization & Which formula for incoming longwave radiation:  1 (Brutsaert, 1975), 2 (Satterlund, 1979), 3 (Idso, 1981), 4(Idso+Hodges),  5 (Koenig-Langlo \& Augstein, 1994), 6 (Andreas \& Ackley, 1982), 7 (Konzelmann, 1994), 8 (Prata, 1996), 9 (Dilley 1998) &  & 1, 2, .., 9 & 9 & sca & opt \\ \hline
MoninObukhov & Atmospherical stability parameter: 1 stability and instability considered, 2 stability not considered, 3 instability not considered, 4 always neutrality &  &  & 1 & sca & num \\ \hline
Surroundings & Yes(1), No(0) & - &  & 0 & sca & opt \\ \hline
\caption{Table of surface energy flux parameters  (numeric)}
\label{surfaceenergyfluxes_numeric}
\end{longtable}
\end{center}


\clearpage
\begin{center}
\begin{longtable}{|p {6.5 cm}|p {4.5 cm}|p {3 cm}|p{3 cm}|p{1.5 cm}|p{1.5 cm}|p{2 cm}|}
\hline
\textbf{Keyword} & \textbf{Description} & \textbf{M. U.} & \textbf{range} & \textbf{Default Value} & \textbf{Scalar / Vector} & \textbf{Logical / Numeric} \\ \hline
\endfirsthead
\hline
\multicolumn{7}{| c |}{continued from previous page} \\
\hline
\textbf{Keyword} & \textbf{Description} & \textbf{M. U.} & \textbf{range} & \textbf{Default Value} & \textbf{Scalar / Vector} & \textbf{Logical / Numeric} \\ \hline
\endhead
\hline
\multicolumn{7}{| c |}{{continued on next page}}\\ 
\hline
\endfoot
\endlastfoot
\hline
Latitude & Average latitude of the basin, positive means north, negative means south (MANDATORY) & degree & -90, 90 & 45 & sca & num \\ \hline
Longitude & Average longitude of the basin, eastwards from 0 meridiane (MANDATORY) & degree & 0, 180 & 0 & sca & num \\ \hline
PointID & identification code for the point of simulation &  &  & NA & sca & num \\ \hline
CoordinatePointX & coordinate X if PixelCoordinates is 1, number of row of the matrix if PixelCoordinates is 0 & m (according to the geographical projection of the maps) &  & NA & vec & num \\ \hline
CoordinatePointY & coordinate Y if PixelCoordinates is 1, number of column of the matrix if PixelCoordinates is 1 & m (according to the geographical projection of the maps) &  & NA & vec & num \\ \hline
PointElevation & elevation of the point of simulation & m a.s.l. &  & NA & vec & num \\ \hline
PointSlope & Slope steepness of the simulation point & degree &  & NA & vec & num \\ \hline
PointAspect & Aspect of the simulation point & degree &  & NA & vec & num \\ \hline
PointSkyViewFactor & Sky View Factor of the simulation point & - &  & NA & vec & num \\ \hline
PointCurvatureNorthSouthDirection & N-S curvature of the simulation point & m$^{-1}$ &  & NA & vec & num \\ \hline
PointCurvatureWestEastDirection & W-E curvature of the simulation point & m$^{-1}$ &  & NA & vec & num \\ \hline
PointCurvatureNorthwest SoutheastDirection & N-W curvature of the simulation point & m$^{-1}$ &  & NA & vec & num \\ \hline
PointCurvatureNortheast SouthwestDirection & N-E curvature of the simulation point & m$^{-1}$ &  & NA & vec & num \\ \hline
PointDrainageLateralDistance & Lateral Drainage distance of the simulation point & m &  & NA & vec & num \\ \hline
PointLatitude & Latitude of the simulation point & degree &  & NA & sca & num \\ \hline
PointLongitude & Longitude of the simulation point & degree &  & NA & sca & num \\ \hline
\caption{Table of topographic parameters  (numeric)}
\label{topo3d_numeric}
\end{longtable}
\end{center}

\clearpage
\begin{center}
\begin{longtable}{|p {6.5 cm}|p {4.5 cm}|p {3 cm}|p{3 cm}|p{1.5 cm}|p{1.5 cm}|p{2 cm}|}
\hline
\textbf{Keyword} & \textbf{Description} & \textbf{M. U.} & \textbf{range} & \textbf{Default Value} & \textbf{Scalar / Vector} & \textbf{Logical / Numeric} \\ \hline
\endfirsthead
\hline
\multicolumn{7}{| c |}{continued from previous page} \\
\hline
\textbf{Keyword} & \textbf{Description} & \textbf{M. U.} & \textbf{range} & \textbf{Default Value} & \textbf{Scalar / Vector} & \textbf{Logical / Numeric} \\ \hline
\endhead
\hline
\multicolumn{7}{| c |}{{continued on next page}}\\ 
\hline
\endfoot
\endlastfoot
\hline
VegHeight & vegetation height & mm & 0, 20000 & 1000 & sca & num \\ \hline
LSAI & Leaf and Stem Area Index [$L^2/L^2$] & - & 0, 1 & 1 & sca & numeric \\ \hline
CanopyFraction & Canopy fraction [0: no canopy in the pixel, 1: pixel fully covered by canopy] & - & 0, 1 & 0 & sca & numeric \\ \hline
DecayCoeffCanopy & Decay coefficient of the eddy diffusivity profile in the canopy & - & 0, inf & 2.5 & sca & numeric \\ \hline
VegSnowBurying & Coefficient of the exponential snow burying of vegetation & - & 0, inf & 1 & sca & numeric \\ \hline
RootDepth & Root depth [mm] (it is used to calculate root\_fraction for each layer, it must be positive) & mm & 0, inf & 300 & sca & numeric \\ \hline
MinStomatalRes & Minimum stomatal resistance & s $m^{-1}$ & 0, inf & 60 & sca & numeric \\ \hline
VegReflectVis & Vegetation reflectivity in the visible & - & 0, 1 & 0.2 & sca & numeric \\ \hline
VegReflNIR & Vegetation reflectivity in the near infrared & - & 0, 1 & 0.2 & sca & numeric \\ \hline
VegTransVis & Vegetation transmissimity in the visible & - & 0, 1 & 0.2 & sca & numeric \\ \hline
VegTransNIR & Vegetation transmissimity in the near infrared & - & 0, 1 & 0.2 & sca & numeric \\ \hline
LeafAngles & Departure of leaf angles from a random distribution (1 horizontal, 0 random, -1 vertical) & - & -1, 0, 1 & 0 & sca & opt \\ \hline
CanDensSurface & Surface density of canopy & kg m$^{-2}$ LSAI$^{-1}$ & 0, inf & 2 & sca & num \\ \hline
\caption{Table of vegetation parameters  (numeric)}
\label{vege_numeric}
\end{longtable}
\end{center}


\clearpage
\section{3D OUTPUT NUMERIC}


\clearpage
\begin{center}
\begin{longtable}{|p {6.5 cm}|p {4.5 cm}|p {3 cm}|p{3 cm}|p{1.5 cm}|p{1.5 cm}|p{2 cm}|}
\hline
\textbf{Keyword} & \textbf{Description} & \textbf{M. U.} & \textbf{range} & \textbf{Default Value} & \textbf{Scalar / Vector} & \textbf{Logical / Numeric} \\ \hline
\endfirsthead
\hline
\multicolumn{7}{| c |}{continued from previous page} \\
\hline
\textbf{Keyword} & \textbf{Description} & \textbf{M. U.} & \textbf{range} & \textbf{Default Value} & \textbf{Scalar / Vector} & \textbf{Logical / Numeric} \\ \hline
\endhead
\hline
\multicolumn{7}{| c |}{{continued on next page}}\\ 
\hline
\endfoot
\endlastfoot
\hline
DtPlotPoint & Plotting Time step (in hour) of THE OUTPUT FOR SPECIFIED PIXELS (0 means the it is not plotted) & h & 0, inf & 0 & vec & num \\ \hline
DatePoint & column number in which one would like to visualize the Date12[DDMMYYYYhhmm]    	 & - & 1, 76 & -1 & sca & num \\ \hline
JulianDayFromYear0Point & column number in which one would like to visualize the JulianDayFromYear0[days]   	 & - & 1, 76 & -1 & sca & num \\ \hline
TimeFromStartPoint & column number in which one would like to visualize the TimeFromStart[days]  & - & 1, 76 & -1 & sca & num \\ \hline
PeriodPoint & column number in which one would like to visualize the Simulation\_Period & - & 1, 76 & -1 & sca & num \\ \hline
RunPoint & column number in which one would like to visualize the Run	 & - & 1, 76 & -1 & sca & num \\ \hline
IDPointPoint & column number in which one would like to visualize the IDpoint  & - & 1, 76 & -1 & sca & num \\ \hline
PsnowPoint & column number in which one would like to visualize the Psnow\_over\_canopy[mm]      & - & 1, 76 & -1 & sca & num \\ \hline
PrainPoint & column number in which one would like to visualize the Prain\_over\_canopy[mm] 	 & - & 1, 76 & -1 & sca & num \\ \hline
PsnowNetPoint & column number in which one would like to visualize the Psnow\_under\_canopy[mm]  & - & 1, 76 & -1 & sca & num \\ \hline
PrainNetPoint & column number in which one would like to visualize the Prain\_under\_canopy[mm] 	 & - & 1, 76 & -1 & sca & num \\ \hline
PrainOnSnowPoint & column number in which one would like to visualize the Prain\_rain\_on\_snow[mm] & - & 1, 76 & -1 & sca & num \\ \hline
WindSpeedPoint & column number in which one would like to visualize the Wind\_speed[m/s]          & - & 1, 76 & -1 & sca & num \\ \hline
WindDirPoint & column number in which one would like to visualize the Wind\_direction[deg]   & - & 1, 76 & -1 & sca & num \\ \hline
RHPoint & column number in which one would like to visualize the Relative\_Humidity[-]     & - & 1, 76 & -1 & sca & num \\ \hline
AirPressPoint & column number in which one would like to visualize the Pressure[mbar]     & - & 1, 76 & -1 & sca & num \\ \hline
AirTempPoint & column number in which one would like to visualize the Tair[\textcelsius]     & - & 1, 76 & -1 & sca & num \\ \hline
TDewPoint & column number in which one would like to visualize the Tdew[\textcelsius]   & - & 1, 76 & -1 & sca & num \\ \hline
TsurfPoint & column number in which one would like to visualize the Tsurface[\textcelsius]     & - & 1, 76 & -1 & sca & num \\ \hline
TvegPoint & column number in which one would like to visualize the Tvegetation[\textcelsius]     & - & 1, 76 & -1 & sca & num \\ \hline
TCanopyAirPoint & column number in which one would like to visualize the Tcanopyair[\textcelsius]     & - & 1, 76 & -1 & sca & num \\ \hline
SurfaceEBPoint & column number in which one would like to visualize the Surface\_Energy\_balance [W/m2]     & - & 1, 76 & -1 & sca & num \\ \hline
SoilHeatFluxPoint & column number in which one would like to visualize the Soil\_heat\_flux[W/m2]      & - & 1, 76 & -1 & sca & num \\ \hline
SWinPoint & column number in which one would like to visualize the SWin[W/m2]   & - & 1, 76 & -1 & sca & num \\ \hline
SWbeamPoint & column number in which one would like to visualize the SWbeam[W/m2]    & - & 1, 76 & -1 & sca & num \\ \hline
SWdiffPoint & column number in which one would like to visualize the SWdiff[W/m2]   & - & 1, 76 & -1 & sca & num \\ \hline
LWinPoint & column number in which one would like to visualize the LWin[W/m2]  & - & 1, 76 & -1 & sca & num \\ \hline
LWinMinPoint & column number in which one would like to visualize the LWin\_min[W/m2]  & - & 1, 76 & -1 & sca & num \\ \hline
LWinMaxPoint & column number in which one would like to visualize the LWin\_max[W/m2] & - & 1, 76 & -1 & sca & num \\ \hline
SWNetPoint & column number in which one would like to visualize the SWnet[W/m2]      & - & 1, 76 & -1 & sca & num \\ \hline
LWNetPoint & column number in which one would like to visualize the LWnet[W/m2]      & - & 1, 76 & -1 & sca & num \\ \hline
HPoint & column number in which one would like to visualize the H[W/m2]       & - & 1, 76 & -1 & sca & num \\ \hline
LEPoint & column number in which one would like to visualize the LE[W/m2]      & - & 1, 76 & -1 & sca & num \\ \hline
CanopyFractionPoint & column number in which one would like to visualize the Canopy\_fraction[-]      & - & 1, 76 & -1 & sca & num \\ \hline
LSAIPoint & column number in which one would like to visualize the LSAI[m2/m2]    & - & 1, 76 & -1 & sca & num \\ \hline
z0vegPoint & column number in which one would like to visualize the z0veg[m]     & - & 1, 76 & -1 & sca & num \\ \hline
d0vegPoint & column number in which one would like to visualize the d0veg[m]     & - & 1, 76 & -1 & sca & num \\ \hline
EstoredCanopyPoint & column number in which one would like to visualize the Estored\_canopy[W/m2]    & - & 1, 76 & -1 & sca & num \\ \hline
SWvPoint & column number in which one would like to visualize the SWv[W/m2]     & - & 1, 76 & -1 & sca & num \\ \hline
LWvPoint & column number in which one would like to visualize the LWv[W/m2]     & - & 1, 76 & -1 & sca & num \\ \hline
HvPoint & column number in which one would like to visualize the Hv[W/m2]      & - & 1, 76 & -1 & sca & num \\ \hline
LEvPoint & column number in which one would like to visualize the LEv[W/m2]     & - & 1, 76 & -1 & sca & num \\ \hline
HgUnvegPoint & column number in which one would like to visualize the Hg\_unveg[W/m2]     & - & 1, 76 & -1 & sca & num \\ \hline
LEgUnvegPoint & column number in which one would like to visualize the LEg\_unveg[W/m2]    & - & 1, 76 & -1 & sca & num \\ \hline
HgVegPoint & column number in which one would like to visualize the Hg\_veg[W/m2]     & - & 1, 76 & -1 & sca & num \\ \hline
LEgVegPoint & column number in which one would like to visualize the LEg\_veg[W/m2]    & - & 1, 76 & -1 & sca & num \\ \hline
EvapSurfacePoint & column number in which one would like to visualize the Evap\_surface[mm]   & - & 1, 76 & -1 & sca & num \\ \hline
TraspCanopyPoint & column number in which one would like to visualize the Trasp\_canopy[mm]     & - & 1, 76 & -1 & sca & num \\ \hline
WaterOnCanopyPoint & column number in which one would like to visualize the Water\_on\_canopy[mm] & - & 1, 76 & -1 & sca & num \\ \hline
SnowOnCanopyPoint & column number in which one would like to visualize the Snow\_on\_canopy[mm] & - & 1, 76 & -1 & sca & num \\ \hline
QVegPoint & column number in which one would like to visualize the specific humidity near the vegetation (grams vapour/grams air)  & - & 1, 76 & -1 & sca & num \\ \hline
QSurfPoint & column number in which one would like to visualize the specific humidity at the surface (grams vapour/grams air)  & - & 1, 76 & -1 & sca & num \\ \hline
QAirPoint & column number in which one would like to visualize the specific humidity at air (grams vapour/grams air)  & - & 1, 76 & -1 & sca & num \\ \hline
QCanopyAirPoint & column number in which one would like to visualize the specific humidity at the canopy-air interface (grams vapour/grams air)  & - & 1, 76 & -1 & sca & num \\ \hline
LObukhovPoint & column number in which one would like to visualize the LObukhov[m] & - & 1, 76 & -1 & sca & num \\ \hline
LObukhovCanopyPoint & column number in which one would like to visualize the LObukhovcanopy[m] & - & 1, 76 & -1 & sca & num \\ \hline
WindSpeedTopCanopyPoint & column number in which one would like to visualize the Wind\_speed\_top\_canopy [m/s]     & - & 1, 76 & -1 & sca & num \\ \hline
DecayKCanopyPoint & column number in which one would like to visualize the Decay\_of\_K\_in\_canopy[-]    & - & 1, 76 & -1 & sca & num \\ \hline
SWupPoint & column number in which one would like to visualize the SWup[W/m$^{2}$]    & - & 1, 76 & -1 & sca & num \\ \hline
LWupPoint & column number in which one would like to visualize the LWup[W/m$^{2}$]    & - & 1, 76 & -1 & sca & num \\ \hline
HupPoint & column number in which one would like to visualize the Hup[W/m$^{2}$]     & - & 1, 76 & -1 & sca & num \\ \hline
LEupPoint & column number in which one would like to visualize the LEup[W/m$^{2}$]    & - & 1, 76 & -1 & sca & num \\ \hline
SnowDepthPoint & column number in which one would like to visualize the snow\_depth[mm]  & - & 1, 76 & -1 & sca & num \\ \hline
SWEPoint & column number in which one would like to visualize the snow\_water\_equivalent [mm]  & - & 1, 76 & -1 & sca & num \\ \hline
SnowDensityPoint & column number in which one would like to visualize the snow\_density[kg/m$^{3}$]  & - & 1, 76 & -1 & sca & num \\ \hline
SnowTempPoint & column number in which one would like to visualize the snow\_temperature[\textcelsius]  & - & 1, 76 & -1 & sca & num \\ \hline
SnowMeltedPoint & column number in which one would like to visualize the snow\_melted[mm]  & - & 1, 76 & -1 & sca & num \\ \hline
SnowSublPoint & column number in which one would like to visualize the snow\_subl[mm]  & - & 1, 76 & -1 & sca & num \\ \hline
SWEBlownPoint & column number in which one would like to visualize the snow\_blown\_away[mm]  & - & 1, 76 & -1 & sca & num \\ \hline
SWESublBlownPoint & column number in which one would like to visualize the snow\_subl\_while\_blown [mm] & - & 1, 76 & -1 & sca & num \\ \hline
GlacDepthPoint & column number in which one would like to visualize the glac\_depth[mm]  & - & 1, 76 & -1 & sca & num \\ \hline
GWEPoint & column number in which one would like to visualize the glac\_water\_equivalent[mm]  & - & 1, 76 & -1 & sca & num \\ \hline
GlacDensityPoint & column number in which one would like to visualize the glac\_density[kg/m$^{3}$]  & - & 1, 76 & -1 & sca & num \\ \hline
GlacTempPoint & column number in which one would like to visualize the glac\_temperature[\textcelsius]  & - & 1, 76 & -1 & sca & num \\ \hline
GlacMeltedPoint & column number in which one would like to visualize the glac\_melted[mm]  & - & 1, 76 & -1 & sca & num \\ \hline
GlacSublPoint & column number in which one would like to visualize the glac\_subl[mm]  & - & 1, 76 & -1 & sca & num \\ \hline
ThawedSoilDepthPoint & column number in which one would like to visualize the thawed\_soil\_depth[mm]  & - & 1, 76 & -1 & sca & num \\ \hline
WaterTableDepthPoint & column number in which one would like to visualize the water\_table\_depth[mm]  & - & 1, 76 & -1 & sca & num \\ \hline
DefaultPoint & 0: use personal setting, 1:use default & - & 0, 1 & 1 & sca & opt \\ \hline
\caption{Table of point output  (numeric)}
\label{point3d_numeric}
\end{longtable}
\end{center}

\clearpage
\begin{center}
\begin{longtable}{|p {6.5 cm}|p {4.5 cm}|p {3 cm}|p{3 cm}|p{1.5 cm}|p{1.5 cm}|p{2 cm}|}
\hline
\textbf{Keyword} & \textbf{Description} & \textbf{M. U.} & \textbf{range} & \textbf{Default Value} & \textbf{Scalar / Vector} & \textbf{Logical / Numeric} \\ \hline
\endfirsthead
\hline
\multicolumn{7}{| c |}{continued from previous page} \\
\hline
\textbf{Keyword} & \textbf{Description} & \textbf{M. U.} & \textbf{range} & \textbf{Default Value} & \textbf{Scalar / Vector} & \textbf{Logical / Numeric} \\ \hline
\endhead
\hline
\multicolumn{7}{| c |}{{continued on next page}}\\ 
\hline
\endfoot
\endlastfoot
\hline
DtPlotBasin & Plotting Time step (in hour) of THE basin averaged output (0 means the it is not plotted) & h & 0, inf & 0 & vec & num \\ \hline
DateBasin & column in which one would like to visualize the Date12 [DDMMYYYYhhmm]    	 & - & 1, 24 & -1 & sca & num \\ \hline
JulianDayFromYear0Basin & column in which one would like to visualize the JulianDayFromYear0[days]   	 & - & 1, 24 & -1 & sca & num \\ \hline
TimeFromStartBasin & column in which one would like to visualize the TimeFromStart[days]     & - & 1, 24 & -1 & sca & num \\ \hline
PeriodBasin & column in which one would like to visualize the Simulation\_Period & - & 1, 24 & -1 & sca & num \\ \hline
RunBasin & column in which one would like to visualize the Run	 & - & 1, 24 & -1 & sca & num \\ \hline
PRainNetBasin & column in which one would like to visualize the Prain\_below\_canopy[mm]      & - & 1, 24 & -1 & sca & num \\ \hline
PSnowNetBasin & column in which one would like to visualize the Psnow\_below\_canopy[mm]      & - & 1, 24 & -1 & sca & num \\ \hline
PRainBasin & column in which one would like to visualize the Prain\_above\_canopy[mm]      & - & 1, 24 & -1 & sca & num \\ \hline
PSnowBasin & column in which one would like to visualize the Prain\_above\_canopy[mm]      & - & 1, 24 & -1 & sca & num \\ \hline
AirTempBasin & column in which one would like to visualize the Tair[\textcelsius]      & - & 1, 24 & -1 & sca & num \\ \hline
TSurfBasin & column in which one would like to visualize the Tsurface[\textcelsius]      & - & 1, 24 & -1 & sca & num \\ \hline
TvegBasin & column in which one would like to visualize the Tvegetation[\textcelsius]      & - & 1, 24 & -1 & sca & num \\ \hline
EvapSurfaceBasin & column in which one would like to visualize the Evap\_surface[mm]      & - & 1, 24 & -1 & sca & num \\ \hline
TraspCanopyBasin & column in which one would like to visualize the Transpiration\_canopy[mm]      & - & 1, 24 & -1 & sca & num \\ \hline
LEBasin & column in which one would like to visualize the LE[W/m$^{2}$]      & - & 1, 24 & -1 & sca & num \\ \hline
HBasin & column in which one would like to visualize the H[W/m$^{2}$]      & - & 1, 24 & -1 & sca & num \\ \hline
SWNetBasin & column in which one would like to visualize the SW[W/m$^{2}$]      & - & 1, 24 & -1 & sca & num \\ \hline
LWNetBasin & column in which one would like to visualize the LW[W/m$^{2}$]      & - & 1, 24 & -1 & sca & num \\ \hline
LEvBasin & column in which one would like to visualize the LEv[W/m$^{2}$]      & - & 1, 24 & -1 & sca & num \\ \hline
HvBasin & column in which one would like to visualize the Hv[W/m$^{2}$]      & - & 1, 24 & -1 & sca & num \\ \hline
SWvBasin & column in which one would like to visualize the SWv[W/m$^{2}$]      & - & 1, 24 & -1 & sca & num \\ \hline
LWvBasin & column in which one would like to visualize the LWv[W/m$^{2}$]      & - & 1, 24 & -1 & sca & num \\ \hline
SWinBasin & column in which one would like to visualize the SWin[W/m$^{2}$]      & - & 1, 24 & -1 & sca & num \\ \hline
LWinBasin & column in which one would like to visualize the LWin[W/m$^{2}$]      & - & 1, 24 & -1 & sca & num \\ \hline
MassErrorBasin & column in which one would like to visualize the Mass\_balance\_error[mm]      & - & 1, 24 & -1 & sca & num \\ \hline
DefaultBasin & 0: use personal setting, 1:use default & - & 0, 1 & 1 & sca & opt \\ \hline
\caption{Table of basin output  (numeric)}
\label{basin_numeric}
\end{longtable}
\end{center}


\clearpage
\begin{center}
\begin{longtable}{|p {6.5 cm}|p {4.5 cm}|p {3 cm}|p{3 cm}|p{1.5 cm}|p{1.5 cm}|p{2 cm}|}
\hline
\textbf{Keyword} & \textbf{Description} & \textbf{M. U.} & \textbf{range} & \textbf{Default Value} & \textbf{Scalar / Vector} & \textbf{Logical / Numeric} \\ \hline
\endfirsthead
\hline
\multicolumn{7}{| c |}{continued from previous page} \\
\hline
\textbf{Keyword} & \textbf{Description} & \textbf{M. U.} & \textbf{range} & \textbf{Default Value} & \textbf{Scalar / Vector} & \textbf{Logical / Numeric} \\ \hline
\endhead
\hline
\multicolumn{7}{| c |}{{continued on next page}}\\ 
\hline
\endfoot
\endlastfoot
\hline
DtPlotDischarge & Plotting Time step (in hour) of THE WATER DISCHARGE (0 means the it is not plotted) & h & 0, inf & 0 & vec & num \\ \hline
\caption{Table of channel flow output  (numeric)}
\label{channel_numeric}
\end{longtable}
\end{center}



\clearpage
\begin{center}
\begin{longtable}{|p {6.5 cm}|p {4.5 cm}|p {3 cm}|p{3 cm}|p{1.5 cm}|p{1.5 cm}|p{2 cm}|}
\hline
\textbf{Keyword} & \textbf{Description} & \textbf{M. U.} & \textbf{range} & \textbf{Default Value} & \textbf{Scalar / Vector} & \textbf{Logical / Numeric} \\ \hline
\endfirsthead
\hline
\multicolumn{7}{| c |}{continued from previous page} \\
\hline
\textbf{Keyword} & \textbf{Description} & \textbf{M. U.} & \textbf{range} & \textbf{Default Value} & \textbf{Scalar / Vector} & \textbf{Logical / Numeric} \\ \hline
\endhead
\hline
\multicolumn{7}{| c |}{{continued on next page}}\\ 
\hline
\endfoot
\endlastfoot
\hline
FormatOutputMaps & Format of the output maps (=2 grass ascii, =3 esri ascii) & - & 2, 3 & 3 & sca & num \\ \hline
\caption{Table of general output  (numeric)}
\label{general_numeric}
\end{longtable}
\end{center}



\clearpage
\begin{center}
\begin{longtable}{|p {6.5 cm}|p {4.5 cm}|p {3 cm}|p{3 cm}|p{1.5 cm}|p{1.5 cm}|p{2 cm}|}
\hline
\textbf{Keyword} & \textbf{Description} & \textbf{M. U.} & \textbf{range} & \textbf{Default Value} & \textbf{Scalar / Vector} & \textbf{Logical / Numeric} \\ \hline
\endfirsthead
\hline
\multicolumn{7}{| c |}{continued from previous page} \\
\hline
\textbf{Keyword} & \textbf{Description} & \textbf{M. U.} & \textbf{range} & \textbf{Default Value} & \textbf{Scalar / Vector} & \textbf{Logical / Numeric} \\ \hline
\endhead
\hline
\multicolumn{7}{| c |}{{continued on next page}}\\ 
\hline
\endfoot
\endlastfoot
\hline
OutputGlacierMaps & frequency (h) of printing of the results of the glacier maps & h &  & 0 & sca & num \\ \hline
DateGlac & column number in which one would like to visualize the Date12 [DDMMYYYYhhmm]    	 & - &  & -1 & sca & num \\ \hline
JulianDayFromYear0Glac & column number in which one would like to visualize the JulianDayFromYear0[days]   	 & - &  & -1 & sca & num \\ \hline
TimeFromStartGlac & column in which one would like to visualize the TimeFromStart[days]     & - &  & -1 & sca & num \\ \hline
PeriodGlac & Column number to write the period number & - &  & -1 & sca & num \\ \hline
RunGlac & Column number to write the run number & - &  & -1 & sca & num \\ \hline
IDPointGlac & column number in which one would like to visualize the IDpoint  & - &  & -1 & sca & num \\ \hline
WaterEquivalentGlac & column number in which one would like the water equivalent of the glacier & - &  & -1 & sca & num \\ \hline
DepthGlac & column number in which one would like to visualize the depth of the glacier & - &  & -1 & sca & num \\ \hline
DensityGlac & column number in which one would like to visualize the density of the glacier & - &  & -1 & sca & num \\ \hline
TempGlac & column number in which one would like to visualize the temperature of the glacier  & - &  & -1 & sca & num \\ \hline
IceContentGlac & column number in which one would like to visualize the ice content of the glacier  & - &  & -1 & sca & num \\ \hline
WatContentGlac & column number in which one would like to visualize the water content of the glacier  & - &  & -1 & sca & num \\ \hline
DefaultGlac & 0: use personal setting, 1:use default & - & 0, 1 & 1 & sca & opt \\ \hline
GlacPlotDepths & depths of the glacier where one wants to write the results & - &  & NA & vec & num \\ \hline
\caption{Table of glacier output  (numeric)}
\label{glacier_numeric}
\end{longtable}
\end{center}

\clearpage
\begin{center}
\begin{longtable}{|p {6.5 cm}|p {4.5 cm}|p {3 cm}|p{3 cm}|p{1.5 cm}|p{1.5 cm}|p{2 cm}|}
\hline
\textbf{Keyword} & \textbf{Description} & \textbf{M. U.} & \textbf{range} & \textbf{Default Value} & \textbf{Scalar / Vector} & \textbf{Logical / Numeric} \\ \hline
\endfirsthead
\hline
\multicolumn{7}{| c |}{continued from previous page} \\
\hline
\textbf{Keyword} & \textbf{Description} & \textbf{M. U.} & \textbf{range} & \textbf{Default Value} & \textbf{Scalar / Vector} & \textbf{Logical / Numeric} \\ \hline
\endhead
\hline
\multicolumn{7}{| c |}{{continued on next page}}\\ 
\hline
\endfoot
\endlastfoot
\hline
OutputMeteoMaps & frequency (h) of printing of the results of the meteo maps & h &  & 0 & sca & num \\ \hline
SpecialPlotBegin & date of begin of plotting of the special output & format DDMMYYhhmm & 01/01/1800 00:00, 01/01/2500 00:00 & 0 & vec & str \\ \hline
SpecialPlotEnd & date of end of plotting of the special output & format DDMMYYhhmm & 01/01/1800 00:00, 01/01/2500 00:00 & 0 & vec & str \\ \hline
\caption{Table of meteo output  (numeric)}
\label{meteo_numeric}
\end{longtable}
\end{center}


\clearpage
\begin{center}
\begin{longtable}{|p {6.5 cm}|p {4.5 cm}|p {3 cm}|p{3 cm}|p{1.5 cm}|p{1.5 cm}|p{2 cm}|}
\hline
\textbf{Keyword} & \textbf{Description} & \textbf{M. U.} & \textbf{range} & \textbf{Default Value} & \textbf{Scalar / Vector} & \textbf{Logical / Numeric} \\ \hline
\endfirsthead
\hline
\multicolumn{7}{| c |}{continued from previous page} \\
\hline
\textbf{Keyword} & \textbf{Description} & \textbf{M. U.} & \textbf{range} & \textbf{Default Value} & \textbf{Scalar / Vector} & \textbf{Logical / Numeric} \\ \hline
\endhead
\hline
\multicolumn{7}{| c |}{{continued on next page}}\\ 
\hline
\endfoot
\endlastfoot
\hline
OutputSnowMaps & frequency (h) of printing of the results of the snow maps & h &  & 0 & sca & num \\ \hline
DateSnow & column number in which one would like to visualize the Date12[DDMMYYYY hhmm]    	 & - &  & -1 & sca & num \\ \hline
JulianDayFromYear0Snow & column number in which one would like to visualize the JulianDayFromYear0[days]   	 & - &  & -1 & sca & num \\ \hline
TimeFromStartSnow & column in which one would like to visualize the TimeFromStart[days]     & - &  & -1 & sca & num \\ \hline
PeriodSnow & Column number to write the period number & - &  & -1 & sca & num \\ \hline
RunSnow & Column number to write the run number & - &  & -1 & sca & num \\ \hline
IDPointSnow & column number in which one would like to visualize the IDpoint  & - &  & -1 & sca & num \\ \hline
WaterEquivalentSnow & column number in which one would like the water equivalent of the snow & - &  & -1 & sca & num \\ \hline
DepthSnow & column number in which one would like to visualize the depth of the snow & - &  & -1 & sca & num \\ \hline
DensitySnow & column number in which one would like to visualize the density of the snow & - &  & -1 & sca & num \\ \hline
TempSnow & column number in which one would like to visualize the temperature of the snow  & - &  & -1 & sca & num \\ \hline
IceContentSnow & column number in which one would like to visualize the ice content of the snow  & - &  & -1 & sca & num \\ \hline
WatContentSnow & column number in which one would like to visualize the water content of the snow  & - &  & -1 & sca & num \\ \hline
DefaultSnow & 0: use personal setting, 1:use default & - & 0, 1 & 1 & sca & opt \\ \hline
SnowPlotDepths & depth at which one wants the data on the snow to be plotted & - &  & NA & vec & num \\ \hline
\caption{Table of snow output  (numeric)}
\label{snow_numeric}
\end{longtable}
\end{center}


\clearpage
\begin{center}
\begin{longtable}{|p {6.5 cm}|p {4.5 cm}|p {3 cm}|p{3 cm}|p{1.5 cm}|p{1.5 cm}|p{2 cm}|}
\hline
\textbf{Keyword} & \textbf{Description} & \textbf{M. U.} & \textbf{range} & \textbf{Default Value} & \textbf{Scalar / Vector} & \textbf{Logical / Numeric} \\ \hline
\endfirsthead
\hline
\multicolumn{7}{| c |}{continued from previous page} \\
\hline
\textbf{Keyword} & \textbf{Description} & \textbf{M. U.} & \textbf{range} & \textbf{Default Value} & \textbf{Scalar / Vector} & \textbf{Logical / Numeric} \\ \hline
\endhead
\hline
\multicolumn{7}{| c |}{{continued on next page}}\\ 
\hline
\endfoot
\endlastfoot
\hline
OutputSoilMaps & frequency (h) of printing of the results of the soil maps & h &  & 0 & sca & num \\ \hline
DateSoil & column number in which one would like to visualize the Date12[DDMMYYYY hhmm]    	 & - &  & -1 & sca & num \\ \hline
JulianDayFromYear0Soil & column number in which one would like to visualize the JulianDayFromYear0[days]   	 & - &  & -1 & sca & num \\ \hline
TimeFromStartSoil & column number in which one would like to visualize the time from the start of the soil & - &  & -1 & sca & num \\ \hline
PeriodSoil & Column number to write the period number & - &  & -1 & sca & num \\ \hline
RunSoil & Column number to write the run number & - &  & -1 & sca & num \\ \hline
IDPointSoil & column number in which one would like to visualize the IDpoint  & - &  & -1 & sca & num \\ \hline
DefaultSoil & 0: use personal setting, 1:use default & - & 0, 1 & 1 & sca & opt \\ \hline
SoilPlotDepths & depth at which one wants the data on the snow to be plotted & m &  & NA & vec & num \\ \hline
\caption{Table of soil output  (numeric)}
\label{soil_numeric}
\end{longtable}
\end{center}


\clearpage
\begin{center}
\begin{longtable}{|p {6.5 cm}|p {4.5 cm}|p {3 cm}|p{3 cm}|p{1.5 cm}|p{1.5 cm}|p{2 cm}|}
\hline
\textbf{Keyword} & \textbf{Description} & \textbf{M. U.} & \textbf{range} & \textbf{Default Value} & \textbf{Scalar / Vector} & \textbf{Logical / Numeric} \\ \hline
\endfirsthead
\hline
\multicolumn{7}{| c |}{continued from previous page} \\
\hline
\textbf{Keyword} & \textbf{Description} & \textbf{M. U.} & \textbf{range} & \textbf{Default Value} & \textbf{Scalar / Vector} & \textbf{Logical / Numeric} \\ \hline
\endhead
\hline
\multicolumn{7}{| c |}{{continued on next page}}\\ 
\hline
\endfoot
\endlastfoot
\hline
OutputSurfEBALMaps & frequency (h) of printing of the results of the Surface energy balance maps & h &  & 0 & sca & num \\ \hline
\caption{Table of soil surface output  (numeric)}
\label{soilsurf_numeric}
\end{longtable}
\end{center}

\clearpage
\begin{center}
\begin{longtable}{|p {6.5 cm}|p {4.5 cm}|p {3 cm}|p{3 cm}|p{1.5 cm}|p{1.5 cm}|p{2 cm}|}
\hline
\textbf{Keyword} & \textbf{Description} & \textbf{M. U.} & \textbf{range} & \textbf{Default Value} & \textbf{Scalar / Vector} & \textbf{Logical / Numeric} \\ \hline
\endfirsthead
\hline
\multicolumn{7}{| c |}{continued from previous page} \\
\hline
\textbf{Keyword} & \textbf{Description} & \textbf{M. U.} & \textbf{range} & \textbf{Default Value} & \textbf{Scalar / Vector} & \textbf{Logical / Numeric} \\ \hline
\endhead
\hline
\multicolumn{7}{| c |}{{continued on next page}}\\ 
\hline
\endfoot
\endlastfoot
\hline
OutputVegetationMaps & frequency (h) of printing of the results of the vegetation maps & h &  & 0 & sca & num \\ \hline
\caption{Table of vegetation output  (numeric)}
\label{vegetation_numeric}
\end{longtable}
\end{center}

\clearpage
\section{1D OUTPUT NUMERIC}




\clearpage
\begin{center}
\begin{longtable}{|p {6.5 cm}|p {4.5 cm}|p {3 cm}|p{3 cm}|p{1.5 cm}|p{1.5 cm}|p{2 cm}|}
\hline
\textbf{Keyword} & \textbf{Description} & \textbf{M. U.} & \textbf{range} & \textbf{Default Value} & \textbf{Scalar / Vector} & \textbf{Logical / Numeric} \\ \hline
\endfirsthead
\hline
\multicolumn{7}{| c |}{continued from previous page} \\
\hline
\textbf{Keyword} & \textbf{Description} & \textbf{M. U.} & \textbf{range} & \textbf{Default Value} & \textbf{Scalar / Vector} & \textbf{Logical / Numeric} \\ \hline
\endhead
\hline
\multicolumn{7}{| c |}{{continued on next page}}\\ 
\hline
\endfoot
\endlastfoot
\hline
DateGlac & column number in which one would like to visualize the Date12[DDMMYYYY hhmm]    	 & - &  & -1 & sca & num \\ \hline
JulianDayFromYear0Glac & column number in which one would like to visualize the JulianDayFromYear0[days]   	 & - &  & -1 & sca & num \\ \hline
TimeFromStartGlac & column in which one would like to visualize the TimeFromStart[days]     & - &  & -1 & sca & num \\ \hline
PeriodGlac & Column number to write the period number & - &  & -1 & sca & num \\ \hline
RunGlac & Column number to write the run number & - &  & -1 & sca & num \\ \hline
IDPointGlac & column number in which one would like to visualize the IDpoint  & - &  & -1 & sca & num \\ \hline
WaterEquivalentGlac & column number in which one would like the water equivalent of the glacier & - &  & -1 & sca & num \\ \hline
DepthGlac & column number in which one would like to visualize the depth of the glacier & - &  & -1 & sca & num \\ \hline
DensityGlac & column number in which one would like to visualize the density of the glacier & - &  & -1 & sca & num \\ \hline
TempGlac & column number in which one would like to visualize the temperature of the glacier  & - &  & -1 & sca & num \\ \hline
IceContentGlac & column number in which one would like to visualize the ice content of the glacier  & - &  & -1 & sca & num \\ \hline
WatContentGlac & column number in which one would like to visualize the water content of the glacier  & - &  & -1 & sca & num \\ \hline
DefaultGlac & 0: use personal setting, 1:use default & - & 0, 1 & 1 & sca & opt \\ \hline
GlacPlotDepths & depths of the glacier where one wants to write the results & - &  & NA & vec & num \\ \hline
\caption{Table of glacier output  (numeric)}
\label{glacier1d_numeric}
\end{longtable}
\end{center}

\clearpage
\begin{center}
\begin{longtable}{|p {6.5 cm}|p {4.5 cm}|p {3 cm}|p{3 cm}|p{1.5 cm}|p{1.5 cm}|p{2 cm}|}
\hline
\textbf{Keyword} & \textbf{Description} & \textbf{M. U.} & \textbf{range} & \textbf{Default Value} & \textbf{Scalar / Vector} & \textbf{Logical / Numeric} \\ \hline
\endfirsthead
\hline
\multicolumn{7}{| c |}{continued from previous page} \\
\hline
\textbf{Keyword} & \textbf{Description} & \textbf{M. U.} & \textbf{range} & \textbf{Default Value} & \textbf{Scalar / Vector} & \textbf{Logical / Numeric} \\ \hline
\endhead
\hline
\multicolumn{7}{| c |}{{continued on next page}}\\ 
\hline
\endfoot
\endlastfoot
\hline
DtPlotPoint & Plotting Time step (in hour) of THE OUTPUT FOR SPECIFIED PIXELS (0 means the it is not plotted) & h & 0, inf & 0 & vec & num \\ \hline
DatePoint & column number in which one would like to visualize the Date12[DDMMYYYY hhmm]    	 & - & 1, 76 & -1 & sca & num \\ \hline
JulianDayFromYear0Point & column number in which one would like to visualize the JulianDayFromYear0[days]   	 & - & 1, 76 & -1 & sca & num \\ \hline
TimeFromStartPoint & column number in which one would like to visualize the TimeFromStart[days]  & - & 1, 76 & -1 & sca & num \\ \hline
PeriodPoint & column number in which one would like to visualize the Simulation\_Period & - & 1, 76 & -1 & sca & num \\ \hline
RunPoint & column number in which one would like to visualize the Run	 & - & 1, 76 & -1 & sca & num \\ \hline
IDPointPoint & column number in which one would like to visualize the IDpoint  & - & 1, 76 & -1 & sca & num \\ \hline
PsnowPoint & column number in which one would like to visualize the Psnow\_over\_canopy[mm]      & - & 1, 76 & -1 & sca & num \\ \hline
PrainPoint & column number in which one would like to visualize the Prain\_over\_canopy[mm] 	 & - & 1, 76 & -1 & sca & num \\ \hline
PsnowNetPoint & column number in which one would like to visualize the Psnow\_under\_canopy[mm]  & - & 1, 76 & -1 & sca & num \\ \hline
PrainNetPoint & column number in which one would like to visualize the Prain\_under\_canopy[mm] 	 & - & 1, 76 & -1 & sca & num \\ \hline
PrainOnSnowPoint & column number in which one would like to visualize the Prain\_rain\_on\_snow[mm] & - & 1, 76 & -1 & sca & num \\ \hline
WindSpeedPoint & column number in which one would like to visualize the Wind\_speed[m/s]          & - & 1, 76 & -1 & sca & num \\ \hline
WindDirPoint & column number in which one would like to visualize the Wind\_direction[deg]   & - & 1, 76 & -1 & sca & num \\ \hline
RHPoint & column number in which one would like to visualize the Relative\_Humidity[-]     & - & 1, 76 & -1 & sca & num \\ \hline
AirPressPoint & column number in which one would like to visualize the Pressure[mbar]     & - & 1, 76 & -1 & sca & num \\ \hline
AirTempPoint & column number in which one would like to visualize the Tair[\textcelsius]     & - & 1, 76 & -1 & sca & num \\ \hline
TDewPoint & column number in which one would like to visualize the Tdew[\textcelsius]   & - & 1, 76 & -1 & sca & num \\ \hline
TsurfPoint & column number in which one would like to visualize the Tsurface[\textcelsius]     & - & 1, 76 & -1 & sca & num \\ \hline
TvegPoint & column number in which one would like to visualize the Tvegetation[\textcelsius]     & - & 1, 76 & -1 & sca & num \\ \hline
TCanopyAirPoint & column number in which one would like to visualize the Tcanopyair[\textcelsius]     & - & 1, 76 & -1 & sca & num \\ \hline
SurfaceEBPoint & column number in which one would like to visualize the Surface\_Energy\_balance [W/m$^{2}$]     & - & 1, 76 & -1 & sca & num \\ \hline
SoilHeatFluxPoint & column number in which one would like to visualize the Soil\_heat\_flux[W/m$^{2}$]      & - & 1, 76 & -1 & sca & num \\ \hline
SWinPoint & column number in which one would like to visualize the SWin[W/m$^{2}$]   & - & 1, 76 & -1 & sca & num \\ \hline
SWbeamPoint & column number in which one would like to visualize the SWbeam[W/m$^{2}$]    & - & 1, 76 & -1 & sca & num \\ \hline
SWdiffPoint & column number in which one would like to visualize the SWdiff[W/m$^{2}$]   & - & 1, 76 & -1 & sca & num \\ \hline
LWinPoint & column number in which one would like to visualize the LWin[W/m$^{2}$]  & - & 1, 76 & -1 & sca & num \\ \hline
LWinMinPoint & column number in which one would like to visualize the LWin\_min[W/m$^{2}$]  & - & 1, 76 & -1 & sca & num \\ \hline
LWinMaxPoint & column number in which one would like to visualize the LWin\_max[W/m$^{2}$] & - & 1, 76 & -1 & sca & num \\ \hline
SWNetPoint & column number in which one would like to visualize the SWnet[W/m$^{2}$]      & - & 1, 76 & -1 & sca & num \\ \hline
LWNetPoint & column number in which one would like to visualize the LWnet[W/m$^{2}$]      & - & 1, 76 & -1 & sca & num \\ \hline
HPoint & column number in which one would like to visualize the H[W/m$^{2}$]       & - & 1, 76 & -1 & sca & num \\ \hline
LEPoint & column number in which one would like to visualize the LE[W/m$^{2}$]      & - & 1, 76 & -1 & sca & num \\ \hline
CanopyFractionPoint & column number in which one would like to visualize the Canopy\_fraction[-]      & - & 1, 76 & -1 & sca & num \\ \hline
LSAIPoint & column number in which one would like to visualize the LSAI[m$^{2}$/m$^{2}$]    & - & 1, 76 & -1 & sca & num \\ \hline
z0vegPoint & column number in which one would like to visualize the z0veg[m]     & - & 1, 76 & -1 & sca & num \\ \hline
d0vegPoint & column number in which one would like to visualize the d0veg[m]     & - & 1, 76 & -1 & sca & num \\ \hline
EstoredCanopyPoint & column number in which one would like to visualize the Estored\_canopy[W/m2]    & - & 1, 76 & -1 & sca & num \\ \hline
SWvPoint & column number in which one would like to visualize the SWv[W/m$^{2}$]     & - & 1, 76 & -1 & sca & num \\ \hline
LWvPoint & column number in which one would like to visualize the LWv[W/m$^{2}$]     & - & 1, 76 & -1 & sca & num \\ \hline
HvPoint & column number in which one would like to visualize the Hv[W/m$^{2}$]      & - & 1, 76 & -1 & sca & num \\ \hline
LEvPoint & column number in which one would like to visualize the LEv[W/m$^{2}$]     & - & 1, 76 & -1 & sca & num \\ \hline
HgUnvegPoint & column number in which one would like to visualize the Hg\_unveg[W/m$^{2}$]     & - & 1, 76 & -1 & sca & num \\ \hline
LEgUnvegPoint & column number in which one would like to visualize the LEg\_unveg[W/m$^{2}$]    & - & 1, 76 & -1 & sca & num \\ \hline
HgVegPoint & column number in which one would like to visualize the Hg\_veg[W/m$^{2}$]     & - & 1, 76 & -1 & sca & num \\ \hline
LEgVegPoint & column number in which one would like to visualize the LEg\_veg[W/m$^{2}$]    & - & 1, 76 & -1 & sca & num \\ \hline
EvapSurfacePoint & column number in which one would like to visualize the Evap\_surface[mm]   & - & 1, 76 & -1 & sca & num \\ \hline
TraspCanopyPoint & column number in which one would like to visualize the Trasp\_canopy[mm]     & - & 1, 76 & -1 & sca & num \\ \hline
WaterOnCanopyPoint & column number in which one would like to visualize the Water\_on\_canopy[mm] & - & 1, 76 & -1 & sca & num \\ \hline
SnowOnCanopyPoint & column number in which one would like to visualize the Snow\_on\_canopy[mm] & - & 1, 76 & -1 & sca & num \\ \hline
QVegPoint & column number in which one would like to visualize the specific humidity near the vegetation (grams vapour/grams air)  & - & 1, 76 & -1 & sca & num \\ \hline
QSurfPoint & column number in which one would like to visualize the specific humidity at the surface (grams vapour/grams air)  & - & 1, 76 & -1 & sca & num \\ \hline
QAirPoint & column number in which one would like to visualize the specific humidity at air (grams vapour/grams air)  & - & 1, 76 & -1 & sca & num \\ \hline
QCanopyAirPoint & column number in which one would like to visualize the specific humidity at the canopy-air interface (grams vapour/grams air)  & - & 1, 76 & -1 & sca & num \\ \hline
LObukhovPoint & column number in which one would like to visualize the LObukhov[m] & - & 1, 76 & -1 & sca & num \\ \hline
LObukhovCanopyPoint & column number in which one would like to visualize the LObukhovcanopy[m] & - & 1, 76 & -1 & sca & num \\ \hline
WindSpeedTopCanopyPoint & column number in which one would like to visualize the Wind\_speed\_top\_canopy [m/s]     & - & 1, 76 & -1 & sca & num \\ \hline
DecayKCanopyPoint & column number in which one would like to visualize the Decay\_of\_K\_in\_canopy[-]    & - & 1, 76 & -1 & sca & num \\ \hline
SWupPoint & column number in which one would like to visualize the SWup[W/m$^{2}$]    & - & 1, 76 & -1 & sca & num \\ \hline
LWupPoint & column number in which one would like to visualize the LWup[W/m$^{2}$]    & - & 1, 76 & -1 & sca & num \\ \hline
HupPoint & column number in which one would like to visualize the Hup[W/m$^{2}$]     & - & 1, 76 & -1 & sca & num \\ \hline
LEupPoint & column number in which one would like to visualize the LEup[W/m$^{2}$]    & - & 1, 76 & -1 & sca & num \\ \hline
SnowDepthPoint & column number in which one would like to visualize the snow\_depth[mm]  & - & 1, 76 & -1 & sca & num \\ \hline
SWEPoint & column number in which one would like to visualize the snow\_water\_equivalent [mm]  & - & 1, 76 & -1 & sca & num \\ \hline
SnowDensityPoint & column number in which one would like to visualize the snow\_density[kg/$^{3}$]  & - & 1, 76 & -1 & sca & num \\ \hline
SnowTempPoint & column number in which one would like to visualize the snow\_temperature[\textcelsius]  & - & 1, 76 & -1 & sca & num \\ \hline
SnowMeltedPoint & column number in which one would like to visualize the snow\_melted[mm]  & - & 1, 76 & -1 & sca & num \\ \hline
SnowSublPoint & column number in which one would like to visualize the snow\_subl[mm]  & - & 1, 76 & -1 & sca & num \\ \hline
SWEBlownPoint & column number in which one would like to visualize the snow\_blown\_away[mm]  & - & 1, 76 & -1 & sca & num \\ \hline
SWESublBlownPoint & column number in which one would like to visualize the snow\_subl\_while\_blown [mm] & - & 1, 76 & -1 & sca & num \\ \hline
GlacDepthPoint & column number in which one would like to visualize the glac\_depth[mm]  & - & 1, 76 & -1 & sca & num \\ \hline
GWEPoint & column number in which one would like to visualize the glac\_water\_equivalent[mm]  & - & 1, 76 & -1 & sca & num \\ \hline
GlacDensityPoint & column number in which one would like to visualize the glac\_density[kg/m$^{3}$]  & - & 1, 76 & -1 & sca & num \\ \hline
GlacTempPoint & column number in which one would like to visualize the glac\_temperature[\textcelsius]  & - & 1, 76 & -1 & sca & num \\ \hline
GlacMeltedPoint & column number in which one would like to visualize the glac\_melted[mm]  & - & 1, 76 & -1 & sca & num \\ \hline
GlacSublPoint & column number in which one would like to visualize the glac\_subl[mm]  & - & 1, 76 & -1 & sca & num \\ \hline
ThawedSoilDepthPoint & column number in which one would like to visualize the thawed\_soil\_depth [mm]  & - & 1, 76 & -1 & sca & num \\ \hline
WaterTableDepthPoint & column number in which one would like to visualize the water\_table\_depth [mm]  & - & 1, 76 & -1 & sca & num \\ \hline
DefaultPoint & 0: use personal setting, 1:use default & - & 0, 1 & 1 & sca & opt \\ \hline
\caption{Table of point output  (numeric)}
\label{point1d_numeric}
\end{longtable}
\end{center}

\clearpage
\begin{center}
\begin{longtable}{|p {6.5 cm}|p {4.5 cm}|p {3 cm}|p{3 cm}|p{1.5 cm}|p{1.5 cm}|p{2 cm}|}
\hline
\textbf{Keyword} & \textbf{Description} & \textbf{M. U.} & \textbf{range} & \textbf{Default Value} & \textbf{Scalar / Vector} & \textbf{Logical / Numeric} \\ \hline
\endfirsthead
\hline
\multicolumn{7}{| c |}{continued from previous page} \\
\hline
\textbf{Keyword} & \textbf{Description} & \textbf{M. U.} & \textbf{range} & \textbf{Default Value} & \textbf{Scalar / Vector} & \textbf{Logical / Numeric} \\ \hline
\endhead
\hline
\multicolumn{7}{| c |}{{continued on next page}}\\ 
\hline
\endfoot
\endlastfoot
\hline
DateSnow & column number in which one would like to visualize the Date12 [DDMMYYYYhhmm]    	 & - &  & -1 & sca & num \\ \hline
JulianDayFromYear0Snow & column number in which one would like to visualize the JulianDayFromYear0[days]   	 & - &  & -1 & sca & num \\ \hline
TimeFromStartSnow & column in which one would like to visualize the TimeFromStart[days]     & - &  & -1 & sca & num \\ \hline
PeriodSnow & Column number to write the period number & - &  & -1 & sca & num \\ \hline
RunSnow & Column number to write the run number & - &  & -1 & sca & num \\ \hline
IDPointSnow & column number in which one would like to visualize the IDpoint  & - &  & -1 & sca & num \\ \hline
WaterEquivalentSnow & column number in which one would like the water equivalent of the snow & - &  & -1 & sca & num \\ \hline
DepthSnow & column number in which one would like to visualize the depth of the snow & - &  & -1 & sca & num \\ \hline
DensitySnow & column number in which one would like to visualize the density of the snow & - &  & -1 & sca & num \\ \hline
TempSnow & column number in which one would like to visualize the temperature of the snow  & - &  & -1 & sca & num \\ \hline
IceContentSnow & column number in which one would like to visualize the ice content of the snow  & - &  & -1 & sca & num \\ \hline
WatContentSnow & column number in which one would like to visualize the water content of the snow  & - &  & -1 & sca & num \\ \hline
DefaultSnow & 0: use personal setting, 1:use default & - & 0, 1 & 1 & sca & opt \\ \hline
SnowPlotDepths & depth at which one wants the data on the snow to be plotted & - &  & NA & vec & num \\ \hline
\caption{Table of snow output  (numeric)}
\label{snow1d_numeric}
\end{longtable}
\end{center}

\clearpage
\begin{center}
\begin{longtable}{|p {6.5 cm}|p {4.5 cm}|p {3 cm}|p{3 cm}|p{1.5 cm}|p{1.5 cm}|p{2 cm}|}
\hline
\textbf{Keyword} & \textbf{Description} & \textbf{M. U.} & \textbf{range} & \textbf{Default Value} & \textbf{Scalar / Vector} & \textbf{Logical / Numeric} \\ \hline
\endfirsthead
\hline
\multicolumn{7}{| c |}{continued from previous page} \\
\hline
\textbf{Keyword} & \textbf{Description} & \textbf{M. U.} & \textbf{range} & \textbf{Default Value} & \textbf{Scalar / Vector} & \textbf{Logical / Numeric} \\ \hline
\endhead
\hline
\multicolumn{7}{| c |}{{continued on next page}}\\ 
\hline
\endfoot
\endlastfoot
\hline
DateSoil & column number in which one would like to visualize the Date12 [DDMMYYYYhhmm]    	 & - &  & -1 & sca & num \\ \hline
JulianDayFromYear0Soil & column number in which one would like to visualize the JulianDayFromYear0[days]   	 & - &  & -1 & sca & num \\ \hline
TimeFromStartSoil & column number in which one would like to visualize the time from the start of the soil & - &  & -1 & sca & num \\ \hline
PeriodSoil & Column number to write the period number & - &  & -1 & sca & num \\ \hline
RunSoil & Column number to write the run number & - &  & -1 & sca & num \\ \hline
IDPointSoil & column number in which one would like to visualize the IDpoint  & - &  & -1 & sca & num \\ \hline
DefaultSoil & 0: use personal setting, 1:use default & - & 0, 1 & 1 & sca & opt \\ \hline
SoilPlotDepths & depth at which one wants the data on the snow to be plotted & m &  & NA & vec & num \\ \hline
\caption{Table of snow output  (numeric)}
\label{soil1d_numeric}
\end{longtable}
\end{center}

\section{1D INPUT CHARACTER}

\begin{center}
\begin{longtable}{|p {7 cm}|p {7 cm}|p {3 cm}|p {4 cm}|}
\hline
\textbf{Keyword} & \textbf{Description} & \textbf{Associated file} & \textbf{type (file, header)} \\ \hline
\endfirsthead
\hline
\multicolumn{4}{| c |}{continued from previous page} \\
\hline
\textbf{Keyword} & \textbf{Description} & \textbf{Associated file} & \textbf{type (file, header)} \\ \hline
\endhead
\hline
\multicolumn{4}{| c |}{{continued on next page}}\\ 
\hline
\endfoot
\endlastfoot
\hline
HeaderSoilInitPres & column name in the file SoilParFile for the initial total pressure head & SoilParFile & header \\ \hline
HeaderSoilInitTemp & column name in the file SoilParFile for the initial temperature & SoilParFile & header \\ \hline
\caption{Table of initial conditions (character)}
\label{IC1D_data}
\end{longtable}
\end{center}
\clearpage



\begin{center}
\begin{longtable}{|p {7 cm}|p {7 cm}|p {3 cm}|p {4 cm}|}
\hline
\textbf{Keyword} & \textbf{Description} & \textbf{Associated file} & \textbf{type (file, header)} \\ \hline
\endfirsthead
\hline
\multicolumn{4}{| c |}{continued from previous page} \\
\hline
\textbf{Keyword} & \textbf{Description} & \textbf{Associated file} & \textbf{type (file, header)} \\ \hline
\endhead
\hline
\multicolumn{4}{| c |}{{continued on next page}}\\ 
\hline
\endfoot
\endlastfoot
\hline
HeaderPointDepthFreeSurface & column name in the file PointFile for the depth of the free surface of the point & PointFile & header \\ \hline
\caption{Table of runoff parameters (character)}
\label{runoff1D_data}
\end{longtable}
\end{center}
\clearpage


\begin{center}
\begin{longtable}{|p {7 cm}|p {7 cm}|p {3 cm}|p {4 cm}|}
\hline
\textbf{Keyword} & \textbf{Description} & \textbf{Associated file} & \textbf{type (file, header)} \\ \hline
\endfirsthead
\hline
\multicolumn{4}{| c |}{continued from previous page} \\
\hline
\textbf{Keyword} & \textbf{Description} & \textbf{Associated file} & \textbf{type (file, header)} \\ \hline
\endhead
\hline
\multicolumn{4}{| c |}{{continued on next page}}\\ 
\hline
\endfoot
\endlastfoot
\hline
HeaderPointMaxSWE & column name in the file PointFile for the max SWE  of the point & PointFile & header  \\ \hline
\caption{Table of snow parameters (character)}
\label{snow1D_data}
\end{longtable}
\end{center}
\clearpage

\begin{center}
\begin{longtable}{|p {7 cm}|p {7 cm}|p {3 cm}|p {4 cm}|}
\hline
\textbf{Keyword} & \textbf{Description} & \textbf{Associated file} & \textbf{type (file, header)} \\ \hline
\endfirsthead
\hline
\multicolumn{4}{| c |}{continued from previous page} \\
\hline
\textbf{Keyword} & \textbf{Description} & \textbf{Associated file} & \textbf{type (file, header)} \\ \hline
\endhead
\hline
\multicolumn{4}{| c |}{{continued on next page}}\\ 
\hline
\endfoot
\endlastfoot
\hline
HeaderPointSoilType & column name in the file PointFile for the soil type of the point & PointFile & header \\ \hline
HeaderSoilDz & column name in the file SoilParFile for the layers thickness & SoilParFile & header \\ \hline
HeaderNormalHydrConductivity & column name in the file SoilParFile for the normal hydraulic conductivity & SoilParFile & header \\ \hline
HeaderLateralHydrConductivity & column name in the file SoilParFile for the lateral hydraulic conductivity & SoilParFile & header \\ \hline
HeaderThetaRes & column name in the file SoilParFile for the residual water content & SoilParFile & header \\ \hline
HeaderWiltingPoint & column name in the file SoilParFile for the soil wilting point & SoilParFile & header \\ \hline
HeaderFieldCapacity & column name in the file SoilParFile for the field capacity & SoilParFile & header \\ \hline
HeaderThetaSat & column name in the file SoilParFile for the saturated water content & SoilParFile & header \\ \hline
HeaderAlpha & column name in the file alpha parameter of Van Genuchten & SoilParFile & header \\ \hline
HeaderN & column name in the file N parameter of Van Genuchten & SoilParFile & header \\ \hline
HeaderV & column name in the file V parameter of Van Genuchten & SoilParFile & header \\ \hline
HeaderKthSoilSolids & column name in the file thermal conductivity of the soil grains & SoilParFile & header \\ \hline
HeaderCthSoilSolids & column name in the file thermal capacity of the soil grains & SoilParFile & header \\ \hline
HeaderSpecificStorativity & column name in the file specific storativity & SoilParFile & header \\ \hline

\caption{Table of soil parameters (character)}
\label{soil1D_data}
\end{longtable}
\end{center}
\clearpage


\begin{center}
\begin{longtable}{|p {7 cm}|p {7 cm}|p {3 cm}|p {4 cm}|}
\hline
\textbf{Keyword} & \textbf{Description} & \textbf{Associated file} & \textbf{type (file, header)} \\ \hline
\endfirsthead
\hline
\multicolumn{4}{| c |}{continued from previous page} \\
\hline
\textbf{Keyword} & \textbf{Description} & \textbf{Associated file} & \textbf{type (file, header)} \\ \hline
\endhead
\hline
\multicolumn{4}{| c |}{{continued on next page}}\\ 
\hline
\endfoot
\endlastfoot
\hline
HeaderPointLandCoverType & column name in the file PointFile for the land cover of the point & PointFile & header \\ \hline
\caption{Table of soil surface parameters (character)}
\label{soilsur1D_data}
\end{longtable}
\end{center}
\clearpage


\begin{center}
\begin{longtable}{|p {7 cm}|p {7 cm}|p {3 cm}|p {4 cm}|}
\hline
\textbf{Keyword} & \textbf{Description} & \textbf{Associated file} & \textbf{type (file, header)} \\ \hline
\endfirsthead
\hline
\multicolumn{4}{| c |}{continued from previous page} \\
\hline
\textbf{Keyword} & \textbf{Description} & \textbf{Associated file} & \textbf{type (file, header)} \\ \hline
\endhead
\hline
\multicolumn{4}{| c |}{{continued on next page}}\\ 
\hline
\endfoot
\endlastfoot
\hline
PointFile & name of the file providing the properties for the simulation point & / & file \\ \hline
HorizonPoint & name of the file providing the horizon of the simulation point & / & file \\ \hline
HeaderHorizonAngle & String representing the header of the column HorizonAngle of the HorizonPoint and HorizonMeteoStation files & HorizonPoint / HorizonMeteoStation & header \\ \hline
HeaderHorizonHeight & String representing the header of the column HorizonHeight of the HorizonPoint and HorizonMeteoStation files & HorizonPoint / HorizonMeteoStation & header \\ \hline
HeaderPointElevation & column name in the file PointFile for the elevation of the point & PointFile & header \\ \hline
HeaderPointSlope & column name in the file PointFile for the slope steepness of the point & PointFile & header \\ \hline
HeaderPointAspect & column name in the file PointFile for the aspect of the point & PointFile & header \\ \hline
HeaderPointSkyViewFactor & column name in the file PointFile for the sky view factor of the point & PointFile & header \\ \hline
HeaderPointCurvatureNorthSouth Direction & column name in the file PointFile for the N-S curvature of the point & PointFile & header \\ \hline
HeaderPointCurvatureWest EastDirection & column name in the file PointFile for the E-W curvature of the point & PointFile & header \\ \hline
HeaderPointCurvatureNorthwest SoutheastDirection & column name in the file PointFile for the NW-SE curvature of the point & PointFile & header \\ \hline
HeaderPointCurvatureNortheast SouthwestDirection & column name in the file PointFile for the NE-SW curvature of the point & PointFile & header \\ \hline
HeaderPointDrainageLateral Distance & column name in the file PointFile for the distance of lateral drainage & PointFile & header \\ \hline
HeaderPointHorizon & column name in the file PointFile for the horizon ID of the point & PointFile & header \\ \hline
HeaderPointLatitude & column name in the file PointFile for the latitude of the point & PointFile & header \\ \hline
HeaderPointLongitude & column name in the file PointFile for the longitude of the point & PointFile & header \\ \hline
HeaderPointID & column name in the file PointFile for the identification ID of the point & PointFile & header \\ \hline
HeaderCoordinatePointX & column name in the file PointFile for the x coordinate of the point & PointFile & header \\ \hline
HeaderCoordinatePointY & column name in the file PointFile for the y coordinate of the point & PointFile & header \\ \hline


\caption{Table of topographic parameters (character)}
\label{topo1D_data}
\end{longtable}
\end{center}


\clearpage

\section{3D INPUT CHARACTER}



\begin{center}
\begin{longtable}{|p {7 cm}|p {7 cm}|p {3 cm}|p {4 cm}|}
\hline
\textbf{Keyword} & \textbf{Description} & \textbf{Associated file} & \textbf{type (file, header)} \\ \hline
\endfirsthead
\hline
\multicolumn{4}{| c |}{continued from previous page} \\
\hline
\textbf{Keyword} & \textbf{Description} & \textbf{Associated file} & \textbf{type (file, header)} \\ \hline
\endhead
\hline
\multicolumn{4}{| c |}{{continued on next page}}\\ 
\hline
\endfoot
\endlastfoot
\hline
InitWaterTableHeightOverTopoSurface MapFile & name of the file providing the initial condition on the water table height map & / & map \\ \hline
InitSnowDepthMapFile & name of the file providing the initial condition on the snow depth map & / & map \\ \hline
InitSnowAgeMapFile & name of the file providing the initial condition on the snow age map & / & map \\ \hline
InitGlacierDepthMapFile & name of the file providing the initial condition on the glacier depth map & / & map \\ \hline
HeaderSoilInitPres & column name in the file SoilParFile for the initial total pressure head & SoilParFile & header \\ \hline
HeaderSoilInitTemp & column name in the file SoilParFile for the initial temperature & SoilParFile & header \\ \hline


\caption{Table of initial condition (character)}
\label{init3d_data}
\end{longtable}
\end{center}
\clearpage

\begin{center}
\begin{longtable}{|p {7 cm}|p {7 cm}|p {3 cm}|p {4 cm}|}
\hline
\textbf{Keyword} & \textbf{Description} & \textbf{Associated file} & \textbf{type (file, header)} \\ \hline
\endfirsthead
\hline
\multicolumn{4}{| c |}{continued from previous page} \\
\hline
\textbf{Keyword} & \textbf{Description} & \textbf{Associated file} & \textbf{type (file, header)} \\ \hline
\endhead
\hline
\multicolumn{4}{| c |}{{continued on next page}}\\ 
\hline
\endfoot
\endlastfoot
\hline
SoilMapFile & name of the file providing the soil map & / & map \\ \hline
HeaderSoilDz & column name in the file SoilParFile for the layers thickness & SoilParFile & header \\ \hline
HeaderNormalHydrConductivity & column name in the file SoilParFile for the normal hydraulic conductivity & SoilParFile & header \\ \hline
HeaderLateralHydrConductivity & column name in the file SoilParFile for the lateral hydraulic conductivity & SoilParFile & header \\ \hline
HeaderThetaRes & column name in the file SoilParFile for the residual water content & SoilParFile & header \\ \hline
HeaderWiltingPoint & column name in the file SoilParFile for the soil wilting point & SoilParFile & header \\ \hline
HeaderFieldCapacity & column name in the file SoilParFile for the field capacity & SoilParFile & header \\ \hline
HeaderThetaSat & column name in the file SoilParFile for the saturated water content & SoilParFile & header \\ \hline
HeaderAlpha & column name in the file alpha parameter of Van Genuchten & SoilParFile & header \\ \hline
HeaderN & column name in the file N parameter of Van Genuchten & SoilParFile & header \\ \hline
HeaderV & column name in the file V parameter of Van Genuchten & SoilParFile & header \\ \hline
HeaderKthSoilSolids & column name in the file thermal conductivity of the soil grains & SoilParFile & header \\ \hline
HeaderCthSoilSolids & column name in the file thermal capacity of the soil grains & SoilParFile & header \\ \hline
HeaderSpecificStorativity & column name in the file specific storativity & SoilParFile & header \\ \hline
\caption{Table of soil (character)}
\label{soil3d_data}
\end{longtable}
\end{center}
\clearpage


\begin{center}
\begin{longtable}{|p {7 cm}|p {7 cm}|p {3 cm}|p {4 cm}|}
\hline
\textbf{Keyword} & \textbf{Description} & \textbf{Associated file} & \textbf{type (file, header)} \\ \hline
\endfirsthead
\hline
\multicolumn{4}{| c |}{continued from previous page} \\
\hline
\textbf{Keyword} & \textbf{Description} & \textbf{Associated file} & \textbf{type (file, header)} \\ \hline
\endhead
\hline
\multicolumn{4}{| c |}{{continued on next page}}\\ 
\hline
\endfoot
\endlastfoot
\hline
LandCoverMapFile & name of the file providing the land cover map & / & map \\ \hline
\caption{Table of soil surface (character)}
\label{soilsur3d_data}
\end{longtable}
\end{center}
\clearpage


\begin{center}
\begin{longtable}{|p {7 cm}|p {7 cm}|p {3 cm}|p {4 cm}|}
\hline
\textbf{Keyword} & \textbf{Description} & \textbf{Associated file} & \textbf{type (file, header)} \\ \hline
\endfirsthead
\hline
\multicolumn{4}{| c |}{continued from previous page} \\
\hline
\textbf{Keyword} & \textbf{Description} & \textbf{Associated file} & \textbf{type (file, header)} \\ \hline
\endhead
\hline
\multicolumn{4}{| c |}{{continued on next page}}\\ 
\hline
\endfoot
\endlastfoot
\hline
DemFile & name of the file providing the DEM map & / & map \\ \hline
SkyViewFactorMapFile & name of the file providing the sky view factor map & / & map \\ \hline
SlopeMapFile & name of the file providing the slope steepness map & / & map \\ \hline
RiverNetwork & name of the file providing the river network map & / & map \\ \hline
AspectMapFile & name of the file providing the aspect map & / & map \\ \hline
CurvaturesMapFile & name of the file providing the curvature map & / & map \\ \hline
BedrockDepthMapFile & name of the file providing the bedrock depth map & / & map \\ \hline
HeaderPointID & column name in the file PointFile for the identification ID of the point & PointFile & header \\ \hline
HeaderCoordinatePointX & column name in the file PointFile for the x coordinate of the point & PointFile & header \\ \hline
HeaderCoordinatePointY & column name in the file PointFile for the y coordinate of the point & PointFile & header \\ \hline

\caption{Table of topographic parameters (character)}
\label{topo3d_data}
\end{longtable}
\end{center}
\clearpage

\begin{center}
\begin{longtable}{|p {7 cm}|p {7 cm}|p {3 cm}|p {4 cm}|}
\hline
\textbf{Keyword} & \textbf{Description} & \textbf{Associated file} & \textbf{type (file, header)} \\ \hline
\endfirsthead
\hline
\multicolumn{4}{| c |}{continued from previous page} \\
\hline
\textbf{Keyword} & \textbf{Description} & \textbf{Associated file} & \textbf{type (file, header)} \\ \hline
\endhead
\hline
\multicolumn{4}{| c |}{{continued on next page}}\\ 
\hline
\endfoot
\endlastfoot
\hline
TimeStepsFile & name of the file providing the integration time steps & / & file \\ \hline
SoilParFile & name of the file providing the soil parameters & / & file \\ \hline
MeteoFile & name of the file providing the meteo forcing data & / & file \\ \hline
MeteoStationsListFile & name of the file providing the Meteo Station list & / & file \\ \hline
LapseRateFile & name of the file providing the Lapse rate & / & file \\ \hline
HeaderDateDDMMYYYYhhmm LapseRates & column name in the file LapseRateFile for the variable Date & LapseRateFile & header \\ \hline
HeaderLapseRateTemp & column name in the file LapseRateFile for the variable air temperature & LapseRateFile & header \\ \hline
HeaderLapseRateDewTemp & column name in the file LapseRateFile for the variable dew temperature & LapseRateFile & header \\ \hline
HeaderLapseRatePrec & column name in the file LapseRateFile for the variable precipitation & LapseRateFile & header \\ \hline
\caption{Table of general parameters (character)}
\label{gen3d_data}
\end{longtable}
\end{center}
\clearpage

\begin{center}
\begin{longtable}{|p {7 cm}|p {7 cm}|p {3 cm}|p {4 cm}|}
\hline
\textbf{Keyword} & \textbf{Description} & \textbf{Associated file} & \textbf{type (file, header)} \\ \hline
\endfirsthead
\hline
\multicolumn{4}{| c |}{continued from previous page} \\
\hline
\textbf{Keyword} & \textbf{Description} & \textbf{Associated file} & \textbf{type (file, header)} \\ \hline
\endhead
\hline
\multicolumn{4}{| c |}{{continued on next page}}\\ 
\hline
\endfoot
\endlastfoot
\hline
HorizonMeteoStationFile & name of the file providing the horizon of the meteo station & / & file \\ \hline
HeaderIDMeteoStation & column name in the file MeteoFile for the variable IDMeteoStation & MeteoFile & header \\ \hline
HeaderMeteoStationCoordinateX & column name in the file MeteoFile for the variable MeteoStationCoordinateX & MeteoFile & header \\ \hline
HeaderMeteoStationCoordinateY & column name in the file MeteoFile for the variable MeteoStationCoordinateY & MeteoFile & header \\ \hline
HeaderMeteoStationLatitude & column name in the file MeteoFile for the variable MeteoStationLatitude & MeteoFile & header \\ \hline
HeaderMeteoStationLongitude & column name in the file MeteoFile for the variable MeteoStationLongitude & MeteoFile & header \\ \hline
HeaderMeteoStationElevation & column name in the file MeteoFile for the variable MeteoStationElevation & MeteoFile & header \\ \hline
HeaderMeteoStationSkyViewFactor & column name in the file MeteoFile for the variable MeteoStationSkyViewFactor & MeteoFile & header \\ \hline
HeaderMeteoStationStandardTime & column name in the file MeteoFile for the variable MeteoStationStandardTime & MeteoFile & header \\ \hline
\caption{Table of meteo station (character)}
\label{meteostation_data}
\end{longtable}
\end{center}
\clearpage

\begin{center}
\begin{longtable}{|p {7 cm}|p {7 cm}|p {3 cm}|p {4 cm}|}
\hline
\textbf{Keyword} & \textbf{Description} & \textbf{Associated file} & \textbf{type (file, header)} \\ \hline
\endfirsthead
\hline
\multicolumn{4}{| c |}{continued from previous page} \\
\hline
\textbf{Keyword} & \textbf{Description} & \textbf{Associated file} & \textbf{type (file, header)} \\ \hline
\endhead
\hline
\multicolumn{4}{| c |}{{continued on next page}}\\ 
\hline
\endfoot
\endlastfoot
\hline
TimeDependentVegetationParameterFile & name of the file providing the time dependent vegetation parameters & / & file \\ \hline
\caption{Table of vegetation parameters (character)}
\label{vegetation3d_data}
\end{longtable}
\end{center}
\clearpage

\section{1D OUTPUT CHARACTER}

\begin{center}
\begin{longtable}{|p {7 cm}|p {7 cm}|p {3 cm}|p {4 cm}|}
\hline
\textbf{Keyword} & \textbf{Description} & \textbf{Associated file} & \textbf{type (file, header)} \\ \hline
\endfirsthead
\hline
\multicolumn{4}{| c |}{continued from previous page} \\
\hline
\textbf{Keyword} & \textbf{Description} & \textbf{Associated file} & \textbf{type (file, header)} \\ \hline
\endhead
\hline
\multicolumn{4}{| c |}{{continued on next page}}\\ 
\hline
\endfoot
\endlastfoot
\hline
SuccessfulRunFile & column name of the file that summarizes if the simulation has arrived to the end & / & file \\ \hline
FailedRunFile & column name of the file that summarizes if the simulation has failed & / & file \\ \hline
PointOutputFile & name of the output file providing the Point values & / & file \\ \hline
PointOutputFileWriteEnd & name of the output file providing the Point values written just once at the end & / & file \\ \hline
HeaderDatePoint & column name in the file PointOutputFile for the variable DatePoint & PointOutputFile & header \\ \hline
HeaderJulianDayFromYear0Point & column name in the file PointOutputFile for the variable JulianDayFromYear0Point & PointOutputFile & header \\ \hline
HeaderTimeFromStartPoint & column name in the file PointOutputFile for the variable TimeFromStartPoint & PointOutputFile & header \\ \hline
HeaderPeriodPoint & column name in the file PointOutputFile for the variable PeriodPoint & PointOutputFile & header \\ \hline
HeaderRunPoint & column name in the file PointOutputFile for the variable RunPoint & PointOutputFile & header \\ \hline
HeaderIDPointPoint & column name in the file PointOutputFile for the variable IDPointPoint & PointOutputFile & header \\ \hline
HeaderCanopyFractionPoint & column name in the file PointOutputFile for the variable CanopyFractionPoint & PointOutputFile & header \\ \hline
\caption{Table of general parameters (character)}
\label{general1d_data}
\end{longtable}
\end{center}
\clearpage


\begin{center}
\begin{longtable}{|p {7 cm}|p {7 cm}|p {3 cm}|p {4 cm}|}
\hline
\textbf{Keyword} & \textbf{Description} & \textbf{Associated file} & \textbf{type (file, header)} \\ \hline
\endfirsthead
\hline
\multicolumn{4}{| c |}{continued from previous page} \\
\hline
\textbf{Keyword} & \textbf{Description} & \textbf{Associated file} & \textbf{type (file, header)} \\ \hline
\endhead
\hline
\multicolumn{4}{| c |}{{continued on next page}}\\ 
\hline
\endfoot
\endlastfoot
\hline
GlacierProfileFile & name of the output file providing the glacier instantaneous values at various depths & / & file \\ \hline
GlacierProfileFileWriteEnd & name of the output file providing the glacier instantaneous values at various depths written just once at the end & / & file \\ \hline
HeaderDateGlac & column name in the file GlacierProfileFile for the variable Date & GlacierProfileFile & header \\ \hline
HeaderJulianDayFromYear0Glac & column name in the file GlacierProfileFile for the variable Julian Day from 0 & GlacierProfileFile & header \\ \hline
HeaderTimeFromStartGlac & column name in the file GlacierProfileFile for the variable Time from start & GlacierProfileFile & header \\ \hline
HeaderPeriodGlac & column name in the file GlacierProfileFile for the variable Simulation period & GlacierProfileFile & header \\ \hline
HeaderRunGlac & column name in the file GlacierProfileFile for the variable Run & GlacierProfileFile & header \\ \hline
HeaderIDPointGlac & column name in the file GlacierProfileFile for the variable IDPoint & GlacierProfileFile & header \\ \hline
HeaderTempGlac & column name in the file GlacierProfileFile for the variable temperature & GlacierProfileFile & header \\ \hline
HeaderIceContentGlac & column name in the file GlacierProfileFile for the variable ice content & GlacierProfileFile & header \\ \hline
HeaderWatContentGlac & column name in the file GlacierProfileFile for the variable liquid content & GlacierProfileFile & header \\ \hline
HeaderDepthGlac & column name in the file GlacierProfileFile for the variable Depth & GlacierProfileFile & header \\ \hline
HeaderGlacDepthPoint & column name in the file PointOutputFile for the variable GlacDepthPoint & PointOutputFile & header \\ \hline
HeaderGWEPoint & column name in the file PointOutputFile for the variable GWEPoint & PointOutputFile & header \\ \hline
HeaderGlacDensityPoint & column name in the file PointOutputFile for the variable GlacDensityPoint & PointOutputFile & header \\ \hline
HeaderGlacTempPoint & column name in the file PointOutputFile for the variable GlacTempPoint & PointOutputFile & header \\ \hline
HeaderGlacMeltedPoint & column name in the file PointOutputFile for the variable GlacMeltedPoint & PointOutputFile & header \\ \hline
HeaderGlacSublPoint & column name in the file PointOutputFile for the variable GlacSublPoint & PointOutputFile & header \\ \hline
\caption{Table of glacier parameters (character)}
\label{glacier1d_data}
\end{longtable}
\end{center}
\clearpage


\begin{center}
\begin{longtable}{|p {7 cm}|p {7 cm}|p {3 cm}|p {4 cm}|}
\hline
\textbf{Keyword} & \textbf{Description} & \textbf{Associated file} & \textbf{type (file, header)} \\ \hline
\endfirsthead
\hline
\multicolumn{4}{| c |}{continued from previous page} \\
\hline
\textbf{Keyword} & \textbf{Description} & \textbf{Associated file} & \textbf{type (file, header)} \\ \hline
\endhead
\hline
\multicolumn{4}{| c |}{{continued on next page}}\\ 
\hline
\endfoot
\endlastfoot
\hline
HeaderPsnowPoint & column name in the file PointOutputFile for the variable PsnowPoint & PointOutputFile & header \\ \hline
HeaderPrainPoint & column name in the file PointOutputFile for the variable PrainPoint & PointOutputFile & header \\ \hline
HeaderPrainNetPoint & column name in the file PointOutputFile for the variable PrainNetPoint & PointOutputFile & header \\ \hline
HeaderPrainOnSnowPoint & column name in the file PointOutputFile for the variable PrainOnSnowPoint & PointOutputFile & header \\ \hline
HeaderWindSpeedPoint & column name in the file PointOutputFile for the variable WindSpeedPoint & PointOutputFile & header \\ \hline
HeaderWindDirPoint & column name in the file PointOutputFile for the variable WindDirPoint & PointOutputFile & header \\ \hline
HeaderRHPoint & column name in the file PointOutputFile for the variable RHPoint & PointOutputFile & header \\ \hline
HeaderAirPressPoint & column name in the file PointOutputFile for the variable AirPressPoint & PointOutputFile & header \\ \hline
HeaderAirTempPoint & column name in the file PointOutputFile for the variable AirTempPoint & PointOutputFile & header \\ \hline
HeaderTDewPoint & column name in the file PointOutputFile for the variable TDewPoint & PointOutputFile & header \\ \hline
HeaderTsurfPoint & column name in the file PointOutputFile for the variable TsurfPoint & PointOutputFile & header \\ \hline
HeaderQCanopyAirPoint & column name in the file PointOutputFile for the variable specific humidity at the canopy-air interface & PointOutputFile & header \\ \hline
\caption{Table of meteorological parameters (character)}
\label{meteo1d_data}
\end{longtable}
\end{center}
\clearpage


\begin{center}
\begin{longtable}{|p {7 cm}|p {7 cm}|p {3 cm}|p {4 cm}|}
\hline
\textbf{Keyword} & \textbf{Description} & \textbf{Associated file} & \textbf{type (file, header)} \\ \hline
\endfirsthead
\hline
\multicolumn{4}{| c |}{continued from previous page} \\
\hline
\textbf{Keyword} & \textbf{Description} & \textbf{Associated file} & \textbf{type (file, header)} \\ \hline
\endhead
\hline
\multicolumn{4}{| c |}{{continued on next page}}\\ 
\hline
\endfoot
\endlastfoot
\hline
SoilTempProfileFile & name of the output file providing the Soil/rock instantaneous temperature values at various depths & / & file \\ \hline
SoilTempProfileFileWriteEnd & name of the output file providing the Soil/rock instantaneous temperature values at various depths written just once at the end & / & file \\ \hline
SoilAveragedTempProfileFile & name of the output file providing the Soil/rock average (in DtPlotPoint) temperature values at various depths & / & file \\ \hline
SoilAveragedTempProfileFileWriteEnd & name of the output file providing the Soil/rock average (in DtPlotPoint) temperature values at various depths written just once at the end & / & file \\ \hline
SoilLiqWaterPressProfileFile & name of the output file providing the Soil/rock instantaneous liquid water pressure head values at various depths & / & file \\ \hline
SoilLiqWaterPressProfileFileWriteEnd & name of the output file providing the Soil/rock instantaneous liquid water pressure head values at various depths written just once at the end & / & file \\ \hline
SoilTotWaterPressProfileFile & name of the output file providing the Soil/rock instantaneous total (water+ice) pressure head values at various depths & / & file \\ \hline
SoilTotWaterPressProfileFileWriteEnd & name of the output file providing the Soil/rock instantaneous total (water+ice) pressure head values at various depths written just once at the end & / & file \\ \hline
SoilLiqContentProfileFile & name of the output file providing the Soil/rock instantaneous liquid water content values at various depths & / & file \\ \hline
SoilLiqContentProfileFileWriteEnd & name of the output file providing the Soil/rock instantaneous liquid water content values at various depths written just once at the end & / & file \\ \hline
SoilAveragedLiqContentProfileFile & name of the output file providing the Soil/rock average (in DtPlotPoint) liquid water content values at various depths & / & file \\ \hline
SoilAveragedLiqContentProfileFile WriteEnd & name of the output file providing the Soil/rock average (in DtPlotPoint) liquid water content values at various depths written just once at the end & / & file \\ \hline
SoilIceContentProfileFile & name of the output file providing the Soil/rock instantaneous ice content values at various depths & / & file \\ \hline
SoilIceContentProfileFileWriteEnd & name of the output file providing the Soil/rock instantaneous ice content values at various depths written just once at the end & / & file \\ \hline
SoilAveragedIceContentProfileFile & name of the output file providing the Soil/rock average (in DtPlotPoint) ice content values at various depths & / & file \\ \hline
SoilAveragedIceContentProfile FileWriteEnd & name of the output file providing the Soil/rock average (in DtPlotPoint) ice content values at various depths written just once at the end & / & file \\ \hline
HeaderDateSoil & column name in the file PointOutputFile for the variable Date &  & header \\ \hline
HeaderJulianDayFromYear0Soil & column name in the file PointOutputFile for the variable Julian Day from 0 &  & header \\ \hline
HeaderTimeFromStartSoil & column name in the file PointOutputFile for the variable Time from start &  & header \\ \hline
HeaderPeriodSoil & column name in the file PointOutputFile for the variable Simulation period &  & header \\ \hline
HeaderRunSoil & column name in the file PointOutputFile for the variable Run &  & header \\ \hline
HeaderIDPointSoil & column name in the file PointOutputFile for the variable IDPoint &  & header \\ \hline
HeaderThawedSoilDepthPoint & column name in the file PointOutputFile for the variable ThawedSoilDepthPoint & PointOutputFile & header \\ \hline
HeaderWaterTableDepthPoint & column name in the file PointOutputFile for the variable WaterTableDepthPoint & PointOutputFile & header \\ \hline
\caption{Table of meteorological parameters (character)}
\label{soil1d_data}
\end{longtable}
\end{center}
\clearpage










\begin{center}
\begin{longtable}{|p {7 cm}|p {7 cm}|p {3 cm}|p {4 cm}|}
\hline
\textbf{Keyword} & \textbf{Description} & \textbf{Associated file} & \textbf{type (file, header)} \\ \hline
\endfirsthead
\hline
\multicolumn{4}{| c |}{continued from previous page} \\
\hline
\textbf{Keyword} & \textbf{Description} & \textbf{Associated file} & \textbf{type (file, header)} \\ \hline
\endhead
\hline
\multicolumn{4}{| c |}{{continued on next page}}\\ 
\hline
\endfoot
\endlastfoot
\hline
HeaderSurfaceEBPoint & column name in the file PointOutputFile for the variable SurfaceEBPoint & PointOutputFile & header \\ \hline
HeaderSoilHeatFluxPoint & column name in the file PointOutputFile for the variable SoilHeatFluxPoint & PointOutputFile & header \\ \hline
HeaderSWinPoint & column name in the file PointOutputFile for the variable SWinPoint & PointOutputFile & header \\ \hline
HeaderSWbeamPoint & column name in the file PointOutputFile for the variable SWbeamPoint & PointOutputFile & header \\ \hline
HeaderSWdiffPoint & column name in the file PointOutputFile for the variable SWdiffPoint & PointOutputFile & header \\ \hline
HeaderLWinPoint & column name in the file PointOutputFile for the variable LWinPoint & PointOutputFile & header \\ \hline
HeaderLWinMinPoint & column name in the file PointOutputFile for the variable LWinMinPoint & PointOutputFile & header \\ \hline
HeaderLWinMaxPoint & column name in the file PointOutputFile for the variable LWinMaxPoint & PointOutputFile & header \\ \hline
HeaderSWNetPoint & column name in the file PointOutputFile for the variable SWNetPoint & PointOutputFile & header \\ \hline
HeaderLWNetPoint & column name in the file PointOutputFile for the variable LWNetPoint & PointOutputFile & header \\ \hline
HeaderHPoint & column name in the file PointOutputFile for the variable HPoint & PointOutputFile & header \\ \hline
HeaderLEPoint & column name in the file PointOutputFile for the variable LEPoint & PointOutputFile & header \\ \hline
HeaderQSurfPoint & column name in the file PointOutputFile for the variable specific humidity near the soil surface & PointOutputFile & header \\ \hline
HeaderQAirPoint & column name in the file PointOutputFile for the variable specific humidity of the air & PointOutputFile & header \\ \hline
HeaderLObukhovPoint & column name in the file PointOutputFile for the variable LObukhovPoint & PointOutputFile & header \\ \hline
HeaderSWupPoint & column name in the file PointOutputFile for the variable SWupPoint & PointOutputFile & header \\ \hline
HeaderLWupPoint & column name in the file PointOutputFile for the variable LWupPoint & PointOutputFile & header \\ \hline
HeaderHupPoint & column name in the file PointOutputFile for the variable HupPoint & PointOutputFile & header \\ \hline
HeaderLEupPoint & column name in the file PointOutputFile for the variable LEupPoint & PointOutputFile & header \\ \hline
\caption{Table of surface energy flux parameters (character)}
\label{surfaceenergyflux1d_data}
\end{longtable}
\end{center}
\clearpage


\begin{center}
\begin{longtable}{|p {7 cm}|p {7 cm}|p {3 cm}|p {4 cm}|}
\hline
\textbf{Keyword} & \textbf{Description} & \textbf{Associated file} & \textbf{type (file, header)} \\ \hline
\endfirsthead
\hline
\multicolumn{4}{| c |}{continued from previous page} \\
\hline
\textbf{Keyword} & \textbf{Description} & \textbf{Associated file} & \textbf{type (file, header)} \\ \hline
\endhead
\hline
\multicolumn{4}{| c |}{{continued on next page}}\\ 
\hline
\endfoot
\endlastfoot
\hline
HeaderTvegPoint & column name in the file PointOutputFile for the variable TvegPoint & PointOutputFile & header \\ \hline
HeaderTCanopyAirPoint & column name in the file PointOutputFile for the variable TCanopyAirPoint & PointOutputFile & header \\ \hline
HeaderLSAIPoint & column name in the file PointOutputFile for the variable LSAIPoint & PointOutputFile & header \\ \hline
Headerz0vegPoint & column name in the file PointOutputFile for the variable z0vegPoint & PointOutputFile & header \\ \hline
Headerd0vegPoint & column name in the file PointOutputFile for the variable d0vegPoint & PointOutputFile & header \\ \hline
HeaderEstoredCanopyPoint & column name in the file PointOutputFile for the variable EstoredCanopyPoint & PointOutputFile & header \\ \hline
HeaderSWvPoint & column name in the file PointOutputFile for the variable SWvPoint & PointOutputFile & header \\ \hline
HeaderLWvPoint & column name in the file PointOutputFile for the variable LWvPoint & PointOutputFile & header \\ \hline
HeaderHvPoint & column name in the file PointOutputFile for the variable HvPoint & PointOutputFile & header \\ \hline
HeaderLEvPoint & column name in the file PointOutputFile for the variable LEvPoint & PointOutputFile & header \\ \hline
HeaderHgUnvegPoint & column name in the file PointOutputFile for the variable HgUnvegPoint & PointOutputFile & header \\ \hline
HeaderLEgUnvegPoint & column name in the file PointOutputFile for the variable LEgUnvegPoint & PointOutputFile & header \\ \hline
HeaderHgVegPoint & column name in the file PointOutputFile for the variable HgVegPoint & PointOutputFile & header \\ \hline
HeaderLEgVegPoint & column name in the file PointOutputFile for the variable LEgVegPoint & PointOutputFile & header \\ \hline
HeaderEvapSurfacePoint & column name in the file PointOutputFile for the variable EvapSurfacePoint & PointOutputFile & header \\ \hline
HeaderTraspCanopyPoint & column name in the file PointOutputFile for the variable TraspCanopyPoint & PointOutputFile & header \\ \hline
HeaderWaterOnCanopyPoint & column name in the file PointOutputFile for the variable WaterOnCanopyPoint & PointOutputFile & header \\ \hline
HeaderSnowOnCanopyPoint & column name in the file PointOutputFile for the variable SnowOnCanopyPoint & PointOutputFile & header \\ \hline
HeaderQVegPoint & column name in the file PointOutputFile for the variable specific humidity near the vegetation & PointOutputFile & header \\ \hline
HeaderLObukhovCanopyPoint & column name in the file PointOutputFile for the variable LObukhovCanopyPoint & PointOutputFile & header \\ \hline
HeaderWindSpeedTopCanopyPoint & column name in the file PointOutputFile for the variable WindSpeedTopCanopyPoint & PointOutputFile & header \\ \hline
HeaderDecayKCanopyPoint & column name in the file PointOutputFile for the variable DecayKCanopyPoint & PointOutputFile & header \\ \hline
\caption{Table of vegetation parameters (character)}
\label{vegetation1d_data}
\end{longtable}
\end{center}
\clearpage



\section{3D OUTPUT CHARACTER}

\begin{center}
\begin{longtable}{|p {7 cm}|p {7 cm}|p {3 cm}|p {4 cm}|}
\hline
\textbf{Keyword} & \textbf{Description} & \textbf{Associated file} & \textbf{type (file, header)} \\ \hline
\endfirsthead
\hline
\multicolumn{4}{| c |}{continued from previous page} \\
\hline
\textbf{Keyword} & \textbf{Description} & \textbf{Associated file} & \textbf{type (file, header)} \\ \hline
\endhead
\hline
\multicolumn{4}{| c |}{{continued on next page}}\\ 
\hline
\endfoot
\endlastfoot
\hline
SuccessfulRunFile & column name of the file that summarizes if the simulation has arrived to the end & / & file \\ \hline
FailedRunFile & column name of the file that summarizes if the simulation has failed & / & file \\ \hline
PointOutputFile & name of the output file providing the Point values & / & file \\ \hline
PointOutputFileWriteEnd & name of the output file providing the Point values written just once at the end & / & file \\ \hline
HeaderDatePoint & column name in the file PointOutputFile for the variable DatePoint & PointOutputFile & header \\ \hline
HeaderJulianDayFromYear0Point & column name in the file PointOutputFile for the variable JulianDayFromYear0Point & PointOutputFile & header \\ \hline
HeaderTimeFromStartPoint & column name in the file PointOutputFile for the variable TimeFromStartPoint & PointOutputFile & header \\ \hline
HeaderPeriodPoint & column name in the file PointOutputFile for the variable PeriodPoint & PointOutputFile & header \\ \hline
HeaderRunPoint & column name in the file PointOutputFile for the variable RunPoint & PointOutputFile & header \\ \hline
HeaderIDPointPoint & column name in the file PointOutputFile for the variable IDPointPoint & PointOutputFile & header \\ \hline
HeaderCanopyFractionPoint & column name in the file PointOutputFile for the variable CanopyFractionPoint & PointOutputFile & header \\ \hline
\caption{Table of general parameters (character)}
\label{general1d_data}
\end{longtable}
\end{center}
\clearpage


\begin{center}
\begin{longtable}{|p {7 cm}|p {7 cm}|p {3 cm}|p {4 cm}|}
\hline
\textbf{Keyword} & \textbf{Description} & \textbf{Associated file} & \textbf{type (file, header)} \\ \hline
\endfirsthead
\hline
\multicolumn{4}{| c |}{continued from previous page} \\
\hline
\textbf{Keyword} & \textbf{Description} & \textbf{Associated file} & \textbf{type (file, header)} \\ \hline
\endhead
\hline
\multicolumn{4}{| c |}{{continued on next page}}\\ 
\hline
\endfoot
\endlastfoot
\hline
BasinOutputFile & name of the output file providing the Basin values & / & file \\ \hline
BasinOutputFileWriteEnd & name of the output file providing the Basin values written just once at the end & / & file \\ \hline
HeaderDateBasin & column name in the file BasinOutputFile for the variable DateBasin & BasinOutputFile & header \\ \hline
HeaderJulianDayFromYear0Basin & column name in the file BasinOutputFile for the variable JulianDayFromYear0Basin & BasinOutputFile & header \\ \hline
HeaderTimeFromStartBasin & column name in the file BasinOutputFile for the variable TimeFromStartBasin & BasinOutputFile & header \\ \hline
HeaderPeriodBasin & column name in the file BasinOutputFile for the variable PeriodBasin & BasinOutputFile & header \\ \hline
HeaderRunBasin & column name in the file BasinOutputFile for the variable RunBasin & BasinOutputFile & header \\ \hline
HeaderPRainNetBasin & column name in the file BasinOutputFile for the variable PRainNetBasin & BasinOutputFile & header \\ \hline
HeaderPSnowNetBasin & column name in the file BasinOutputFile for the variable PSnowNetBasin & BasinOutputFile & header \\ \hline
HeaderPRainBasin & column name in the file BasinOutputFile for the variable PRainBasin & BasinOutputFile & header \\ \hline
HeaderPSnowBasin & column name in the file BasinOutputFile for the variable PSnowBasin & BasinOutputFile & header \\ \hline
HeaderAirTempBasin & column name in the file BasinOutputFile for the variable AirTempBasin & BasinOutputFile & header \\ \hline
HeaderTSurfBasin & column name in the file BasinOutputFile for the variable TSurfBasin & BasinOutputFile & header \\ \hline
HeaderTvegBasin & column name in the file BasinOutputFile for the variable TvegBasin & BasinOutputFile & header \\ \hline
HeaderEvapSurfaceBasin & column name in the file BasinOutputFile for the variable EvapSurfaceBasin & BasinOutputFile & header \\ \hline
HeaderTraspCanopyBasin & column name in the file BasinOutputFile for the variable TraspCanopyBasin & BasinOutputFile & header \\ \hline
HeaderLEBasin & column name in the file BasinOutputFile for the variable LEBasin & BasinOutputFile & header \\ \hline
HeaderHBasin & column name in the file BasinOutputFile for the variable HBasin & BasinOutputFile & header \\ \hline
HeaderSWNetBasin & column name in the file BasinOutputFile for the variable SWNetBasin & BasinOutputFile & header \\ \hline
HeaderLWNetBasin & column name in the file BasinOutputFile for the variable LWNetBasin & BasinOutputFile & header \\ \hline
HeaderLEvBasin & column name in the file BasinOutputFile for the variable LEvBasin & BasinOutputFile & header \\ \hline
HeaderHvBasin & column name in the file BasinOutputFile for the variable HvBasin & BasinOutputFile & header \\ \hline
HeaderSWvBasin & column name in the file BasinOutputFile for the variable SWvBasin & BasinOutputFile & header \\ \hline
HeaderLWvBasin & column name in the file BasinOutputFile for the variable LWvBasin & BasinOutputFile & header \\ \hline
HeaderSWinBasin & column name in the file BasinOutputFile for the variable SWinBasin & BasinOutputFile & header \\ \hline
HeaderLWinBasin & column name in the file BasinOutputFile for the variable LWinBasin & BasinOutputFile & header \\ \hline
HeaderMassErrorBasin & column name in the file BasinOutputFile for the variable MassErrorBasin & BasinOutputFile & header \\ \hline
\caption{Table of basin parameters (character)}
\label{basin3d_data}
\end{longtable}
\end{center}
\clearpage

\begin{center}
\begin{longtable}{|p {7 cm}|p {7 cm}|p {3 cm}|p {4 cm}|}
\hline
\textbf{Keyword} & \textbf{Description} & \textbf{Associated file} & \textbf{type (file, header)} \\ \hline
\endfirsthead
\hline
\multicolumn{4}{| c |}{continued from previous page} \\
\hline
\textbf{Keyword} & \textbf{Description} & \textbf{Associated file} & \textbf{type (file, header)} \\ \hline
\endhead
\hline
\multicolumn{4}{| c |}{{continued on next page}}\\ 
\hline
\endfoot
\endlastfoot
\hline
DischargeFile & name of the output file providing the discharge values & / & file \\ \hline
\caption{Table of channel flow parameters (character)}
\label{channelflow3d_data}
\end{longtable}
\end{center}
\clearpage

\begin{center}
\begin{longtable}{|p {7 cm}|p {7 cm}|p {3 cm}|p {4 cm}|}
\hline
\textbf{Keyword} & \textbf{Description} & \textbf{Associated file} & \textbf{type (file, header)} \\ \hline
\endfirsthead
\hline
\multicolumn{4}{| c |}{continued from previous page} \\
\hline
\textbf{Keyword} & \textbf{Description} & \textbf{Associated file} & \textbf{type (file, header)} \\ \hline
\endhead
\hline
\multicolumn{4}{| c |}{{continued on next page}}\\ 
\hline
\endfoot
\endlastfoot
\hline
GlacierDepthMapFile & name of the output file providing the Glacier depth map & / & map \\ \hline
GlacierMeltedMapFile & name of the output file providing the Glacier melted map & / & map \\ \hline
GlacierSublimatedMapFile & name of the output file providing the Glacier sublimated map & / & map \\ \hline
GlacierWaterEqMapFile & name of the output file providing the Glacier water equivalent (GWE) map & / & map \\ \hline
SnowDurationMapFile & name of the output file providing the Snow Duration map & / & map \\ \hline
GlacierProfileFile & name of the output file providing the glacier instantaneous values at various depths & / & file \\ \hline
GlacierProfileFileWriteEnd & name of the output file providing the glacier instantaneous values at various depths written just once at the end & / & file \\ \hline
HeaderDateGlac & column name in the file GlacierProfileFile for the variable Date & GlacierProfileFile & header \\ \hline
HeaderJulianDayFromYear0Glac & column name in the file GlacierProfileFile for the variable Julian Day from 0 & GlacierProfileFile & header \\ \hline
HeaderTimeFromStartGlac & column name in the file GlacierProfileFile for the variable Time from start & GlacierProfileFile & header \\ \hline
HeaderPeriodGlac & column name in the file GlacierProfileFile for the variable Simulation period & GlacierProfileFile & header \\ \hline
HeaderRunGlac & column name in the file GlacierProfileFile for the variable Run & GlacierProfileFile & header \\ \hline
HeaderIDPointGlac & column name in the file GlacierProfileFile for the variable IDPoint & GlacierProfileFile & header \\ \hline
HeaderTempGlac & column name in the file GlacierProfileFile for the variable temperature & GlacierProfileFile & header \\ \hline
HeaderIceContentGlac & column name in the file GlacierProfileFile for the variable ice content & GlacierProfileFile & header \\ \hline
HeaderWatContentGlac & column name in the file GlacierProfileFile for the variable liquid content & GlacierProfileFile & header \\ \hline
HeaderDepthGlac & column name in the file GlacierProfileFile for the variable Depth & GlacierProfileFile & header \\ \hline
HeaderGlacDepthPoint & column name in the file PointOutputFile for the variable GlacDepthPoint & PointOutputFile & header \\ \hline
HeaderGWEPoint & column name in the file PointOutputFile for the variable GWEPoint & PointOutputFile & header \\ \hline
HeaderGlacDensityPoint & column name in the file PointOutputFile for the variable GlacDensityPoint & PointOutputFile & header \\ \hline
HeaderGlacTempPoint & column name in the file PointOutputFile for the variable GlacTempPoint & PointOutputFile & header \\ \hline
HeaderGlacMeltedPoint & column name in the file PointOutputFile for the variable GlacMeltedPoint & PointOutputFile & header \\ \hline
HeaderGlacSublPoint & column name in the file PointOutputFile for the variable GlacSublPoint & PointOutputFile & header \\ \hline
\caption{Table of glacier parameters (character)}
\label{glacier3d_data}
\end{longtable}
\end{center}
\clearpage

\begin{center}
\begin{longtable}{|p {7 cm}|p {7 cm}|p {3 cm}|p {4 cm}|}
\hline
\textbf{Keyword} & \textbf{Description} & \textbf{Associated file} & \textbf{type (file, header)} \\ \hline
\endfirsthead
\hline
\multicolumn{4}{| c |}{continued from previous page} \\
\hline
\textbf{Keyword} & \textbf{Description} & \textbf{Associated file} & \textbf{type (file, header)} \\ \hline
\endhead
\hline
\multicolumn{4}{| c |}{{continued on next page}}\\ 
\hline
\endfoot
\endlastfoot
\hline
SurfaceTempMapFile & name of the output file providing the surface temperature map & / & map \\ \hline
PrecipitationMapFile & name of the output file providing the precipitation map & / & map \\ \hline
AirTempMapFile & name of the output file providing the Air temperature map & / & map \\ \hline
WindSpeedMapFile & name of the output file providing the Wind Speed map & / & map \\ \hline
WindDirMapFile & name of the output file providing the Wind Direction map & / & map \\ \hline
RelHumMapFile & name of the output file providing the Rel. Humidity map & / & map \\ \hline
SpecificPlotSurfaceTempMapFile & name of the output file providing the surface air temperature map at high temporal resolution during specific days & / & map \\ \hline
SpecificPlotWindSpeedMapFile & name of the output file providing the wind speed map at high temporal resolution during specific days & / & map \\ \hline
SpecificPlotWindDirMapFile & name of the output file providing the wind direction map at high temporal resolution during specific days & / & map \\ \hline
SpecificPlotRelHumMapFile & name of the output file providing the relative humidity map at high temporal resolution during specific days & / & map \\ \hline
HeaderPsnowPoint & column name in the file PointOutputFile for the variable PsnowPoint & PointOutputFile & header \\ \hline
HeaderPrainPoint & column name in the file PointOutputFile for the variable PrainPoint & PointOutputFile & header \\ \hline
HeaderPrainNetPoint & column name in the file PointOutputFile for the variable PrainNetPoint & PointOutputFile & header \\ \hline
HeaderPrainOnSnowPoint & column name in the file PointOutputFile for the variable PrainOnSnowPoint & PointOutputFile & header \\ \hline
HeaderWindSpeedPoint & column name in the file PointOutputFile for the variable WindSpeedPoint & PointOutputFile & header \\ \hline
HeaderWindDirPoint & column name in the file PointOutputFile for the variable WindDirPoint & PointOutputFile & header \\ \hline
HeaderRHPoint & column name in the file PointOutputFile for the variable RHPoint & PointOutputFile & header \\ \hline
HeaderAirPressPoint & column name in the file PointOutputFile for the variable AirPressPoint & PointOutputFile & header \\ \hline
HeaderAirTempPoint & column name in the file PointOutputFile for the variable AirTempPoint & PointOutputFile & header \\ \hline
HeaderTDewPoint & column name in the file PointOutputFile for the variable TDewPoint & PointOutputFile & header \\ \hline
HeaderTsurfPoint & column name in the file PointOutputFile for the variable TsurfPoint & PointOutputFile & header \\ \hline
HeaderQCanopyAirPoint & column name in the file PointOutputFile for the variable specific humidity at the canopy-air interface & PointOutputFile & header \\ \hline
\caption{Table of meteorological parameters (character)}
\label{meteo3d_data}
\end{longtable}
\end{center}
\clearpage

\begin{center}
\begin{longtable}{|p {7 cm}|p {7 cm}|p {3 cm}|p {4 cm}|}
\hline
\textbf{Keyword} & \textbf{Description} & \textbf{Associated file} & \textbf{type (file, header)} \\ \hline
\endfirsthead
\hline
\multicolumn{4}{| c |}{continued from previous page} \\
\hline
\textbf{Keyword} & \textbf{Description} & \textbf{Associated file} & \textbf{type (file, header)} \\ \hline
\endhead
\hline
\multicolumn{4}{| c |}{{continued on next page}}\\ 
\hline
\endfoot
\endlastfoot
\hline
SnowDepthMapFile & name of the output file providing the Snow depth map & / & map \\ \hline
SnowMeltedMapFile & name of the output file providing the Snow melted map & / & map \\ \hline
SnowSublMapFile & name of the output file providing the Snow sublimated map & / & map \\ \hline
SWEMapFile & name of the output file providing the Snow water equivalent (SWE) map & / & map \\ \hline
AveragedSnowDepthMapFile & name of the output file providing the Average snow depth map & / & map \\ \hline
SpecificPlotSnowDepthMapFile & name of the output file providing the snow depth map at high temporal resolution during specific days & / & map \\ \hline
SnowProfileFile & name of the output file providing the snow instantaneous values at various depths & / & file \\ \hline
SnowProfileFileWriteEnd & name of the output file providing the snow instantaneous values at various depths written just once at the end & / & file \\ \hline
SnowCoveredAreaFile & Name of the output file containing the percentage of the area covered by snow & / & file \\ \hline
HeaderDateSnow & column name in the file SnowProfileFile for the variable Date & SnowProfileFile & header \\ \hline
HeaderJulianDayFromYear0Snow & column name in the file SnowProfileFile for the variable Julian Day from 0 & SnowProfileFile & header \\ \hline
HeaderTimeFromStartSnow & column name in the file SnowProfileFile for the variable Time from start & SnowProfileFile & header \\ \hline
HeaderPeriodSnow & column name in the file SnowProfileFile for the variable Simulation period & SnowProfileFile & header \\ \hline
HeaderRunSnow & column name in the file SnowProfileFile for the variable Run & SnowProfileFile & header \\ \hline
HeaderIDPointSnow & column name in the file SnowProfileFile for the variable IDPoint & SnowProfileFile & header \\ \hline
HeaderTempSnow & column name in the file SnowProfileFile for the variable temperature & SnowProfileFile & header \\ \hline
HeaderIceContentSnow & column name in the file SnowProfileFile for the variable ice content & SnowProfileFile & header \\ \hline
HeaderWatContentSnow & column name in the file SnowProfileFile for the variable liquid content & SnowProfileFile & header \\ \hline
HeaderDepthSnow & column name in the file SnowProfileFile for the variable Depth & SnowProfileFile & header \\ \hline
HeaderPsnowNetPoint & column name in the file PointOutputFile for the variable PsnowNetPoint & PointOutputFile & header \\ \hline
HeaderSnowDepthPoint & column name in the file PointOutputFile for the variable SnowDepthPoint & PointOutputFile & header \\ \hline
HeaderSWEPoint & column name in the file PointOutputFile for the variable SWEPoint & PointOutputFile & header \\ \hline
HeaderSnowDensityPoint & column name in the file PointOutputFile for the variable SnowDensityPoint & PointOutputFile & header \\ \hline
HeaderSnowTempPoint & column name in the file PointOutputFile for the variable SnowTempPoint & PointOutputFile & header \\ \hline
HeaderSnowMeltedPoint & column name in the file PointOutputFile for the variable SnowMeltedPoint & PointOutputFile & header \\ \hline
HeaderSnowSublPoint & column name in the file PointOutputFile for the variable SnowSublPoint & PointOutputFile & header \\ \hline
HeaderSWEBlownPoint & column name in the file PointOutputFile for the variable SWEBlownPoint & PointOutputFile & header \\ \hline
HeaderSWESublBlownPoint & column name in the file PointOutputFile for the variable SWESublBlownPoint & PointOutputFile & header \\ \hline
\caption{Table of snow parameters (character)}
\label{snow3d_data}
\end{longtable}
\end{center}
\clearpage



\begin{center}
\begin{longtable}{|p {7 cm}|p {7 cm}|p {3 cm}|p {4 cm}|}
\hline
\textbf{Keyword} & \textbf{Description} & \textbf{Associated file} & \textbf{type (file, header)} \\ \hline
\endfirsthead
\hline
\multicolumn{4}{| c |}{continued from previous page} \\
\hline
\textbf{Keyword} & \textbf{Description} & \textbf{Associated file} & \textbf{type (file, header)} \\ \hline
\endhead
\hline
\multicolumn{4}{| c |}{{continued on next page}}\\ 
\hline
\endfoot
\endlastfoot
\hline
FirstSoilLayerTempMapFile & name of the map of the temperature of the first soil layer & / & map \\ \hline
SoilAveragedTempTensorFile & Name of the ensamble of raster maps corresponding to the average temperature of each layer (if PlotSoilDepth$\neq$0 it writes the value at the corresponding depths) & / & map \\ \hline
FirstSoilLayerAveragedTempMapFile & name of the map of the average temperature of the first soil layer & / & map \\ \hline
SoilLiqContentTensorFile & Name of the ensamble of raster maps corresponding to the liquid water content of each layer (if PlotSoilDepth$\neq$0 it writes the value at the corresponding depths) & / & map \\ \hline
FirstSoilLayerLiqContentMapFile & name of the map of the liquird water content of the first soil layer & / & map \\ \hline
IceLiqContentTensorFile & Name of the ensamble of raster maps corresponding to the average ice content of each layer (if PlotSoilDepth$\neq$0 it writes the value at the corresponding depths) & / & map \\ \hline
FirstSoilLayerIceContentMapFile & name of the map of the ice content of the first soil layer & / & map \\ \hline
LandSurfaceWaterDepthMapFile & name of the map of the water height above the surface & / & map \\ \hline
SoilLiqWaterPressTensorFile & Name of the ensamble of raster maps corresponding to the water pressure of each layer (if PlotSoilDepth$\neq$0 it writes the value at the corresponding depths) & / & map \\ \hline
ThawedDepthMapFile & name of the output file providing the Thawed soil depth map & / & map \\ \hline
WaterTableDepthMapFile & name of the output file providing the Water table depth map & / & map \\ \hline
FrostTableDepthMapFile & name of the output file providing the Frost table depth map & / & map \\ \hline
SpecificPlotSurfaceWaterContentMapFile & name of the output file providing the surface water content map at high temporal resolution during specific days & / & map \\ \hline
SoilTempTensorFile & Name of the ensamble of raster maps corresponding to the temperature of each layer (if PlotSoilDepth$\neq$0 it writes the value at the corresponding depths) & / & map \\ \hline
SoilTempProfileFile & name of the output file providing the Soil/rock instantaneous temperature values at various depths & / & file \\ \hline
SoilTempProfileFileWriteEnd & name of the output file providing the Soil/rock instantaneous temperature values at various depths written just once at the end & / & file \\ \hline
SoilAveragedTempProfileFile & name of the output file providing the Soil/rock average (in DtPlotPoint) temperature values at various depths & / & file \\ \hline
SoilAveragedTempProfileFileWriteEnd & name of the output file providing the Soil/rock average (in DtPlotPoint) temperature values at various depths written just once at the end & / & file \\ \hline
SoilLiqWaterPressProfileFile & name of the output file providing the Soil/rock instantaneous liquid water pressure head values at various depths & / & file \\ \hline
SoilLiqWaterPressProfileFileWriteEnd & name of the output file providing the Soil/rock instantaneous liquid water pressure head values at various depths written just once at the end & / & file \\ \hline
SoilTotWaterPressProfileFile & name of the output file providing the Soil/rock instantaneous total (water+ice) pressure head values at various depths & / & file \\ \hline
SoilTotWaterPressProfileFileWriteEnd & name of the output file providing the Soil/rock instantaneous total (water+ice) pressure head values at various depths written just once at the end & / & file \\ \hline
SoilLiqContentProfileFile & name of the output file providing the Soil/rock instantaneous liquid water content values at various depths & / & file \\ \hline
SoilLiqContentProfileFileWriteEnd & name of the output file providing the Soil/rock instantaneous liquid water content values at various depths written just once at the end & / & file \\ \hline
SoilAveragedLiqContentProfileFile & name of the output file providing the Soil/rock average (in DtPlotPoint) liquid water content values at various depths & / & file \\ \hline
SoilAveragedLiqContentProfileFile WriteEnd & name of the output file providing the Soil/rock average (in DtPlotPoint) liquid water content values at various depths written just once at the end & / & file \\ \hline
SoilIceContentProfileFile & name of the output file providing the Soil/rock instantaneous ice content values at various depths & / & file \\ \hline
SoilIceContentProfileFileWriteEnd & name of the output file providing the Soil/rock instantaneous ice content values at various depths written just once at the end & / & file \\ \hline
SoilAveragedIceContentProfileFile & name of the output file providing the Soil/rock average (in DtPlotPoint) ice content values at various depths & / & file \\ \hline
SoilAveragedIceContentProfileFile WriteEnd & name of the output file providing the Soil/rock average (in DtPlotPoint) ice content values at various depths written just once at the end & / & file \\ \hline
HeaderDateSoil & column name in the file PointOutputFile for the variable Date &  & header \\ \hline
HeaderJulianDayFromYear0Soil & column name in the file PointOutputFile for the variable Julian Day from 0 &  & header \\ \hline
HeaderTimeFromStartSoil & column name in the file PointOutputFile for the variable Time from start &  & header \\ \hline
HeaderPeriodSoil & column name in the file PointOutputFile for the variable Simulation period &  & header \\ \hline
HeaderRunSoil & column name in the file PointOutputFile for the variable Run &  & header \\ \hline
HeaderIDPointSoil & column name in the file PointOutputFile for the variable IDPoint &  & header \\ \hline
HeaderThawedSoilDepthPoint & column name in the file PointOutputFile for the variable ThawedSoilDepthPoint & PointOutputFile & header \\ \hline
HeaderWaterTableDepthPoint & column name in the file PointOutputFile for the variable WaterTableDepthPoint & PointOutputFile & header \\ \hline
\caption{Table of snow parameters (character)}
\label{soil3d_data}
\end{longtable}
\end{center}
\clearpage


\begin{center}
\begin{longtable}{|p {7 cm}|p {7 cm}|p {3 cm}|p {4 cm}|}
\hline
\textbf{Keyword} & \textbf{Description} & \textbf{Associated file} & \textbf{type (file, header)} \\ \hline
\endfirsthead
\hline
\multicolumn{4}{| c |}{continued from previous page} \\
\hline
\textbf{Keyword} & \textbf{Description} & \textbf{Associated file} & \textbf{type (file, header)} \\ \hline
\endhead
\hline
\multicolumn{4}{| c |}{{continued on next page}}\\ 
\hline
\endfoot
\endlastfoot
\hline
RadiationMapFile & name of the output file providing the Radiation map (all the type of radiations) & / & map \\ \hline
NetRadiationMapFile & name of the output file providing the Net Radiation map & / & map \\ \hline
InLongwaveRadiation MapFile & name of the output file providing the LW Radiation map & / & map \\ \hline
NetLongwaveRadiation MapFile & name of the output file providing the Net LW  Radiation map & / & map \\ \hline
NetShortwaveRadiation MapFile & name of the output file providing the Net SW  Radiation map & / & map \\ \hline
InShortwaveRadiation MapFile & name of the output file providing the Swin  Radiation map & / & map \\ \hline
DirectInShortwaveRadiation MapFile & name of the output file providing the Swdir  Radiation map & / & map \\ \hline
ShadowFractionTime MapFile & name of the output file providing the map of the  fraction of Shadow in the time & / & map \\ \hline
SurfaceHeatFluxMapFile & name of the output file providing the Surface heat flux  map & / & map \\ \hline
SurfaceSensibleHeatFlux MapFile & name of the output file providing the Surface sensible heat flux  map & / & map \\ \hline
SurfaceLatentHeatFlux MapFile & name of the output file providing the Surface latent  heat flux  map & / & map \\ \hline
SpecificPlotSurfaceHeat FluxMapFile & name of the output file providing the surface heat flux map at high temporal resolution during specific days & / & map \\ \hline
SpecificPlotTotalSensibleHeatFlux MapFile & name of the output file providing the total sensible heat flux map at high temporal resolution during specific days & / & map \\ \hline
SpecificPlotTotalLatentHeatFlux MapFile & name of the output file providing the total latent heat flux map at high temporal resolution during specific days & / & map \\ \hline
SpecificPlotSurfaceSensibleHeatFlux MapFile & name of the output file providing the surface sensible heat flux map at high temporal resolution during specific days & / & map \\ \hline
SpecificPlotSurfaceLatentHeatFlux MapFile & name of the output file providing the surface latent heat flux map at high temporal resolution during specific days & / & map \\ \hline
SpecificPlotIncomingShortwaveRad MapFile & name of the output file providing the Swin flux map at high temporal resolution during specific days & / & map \\ \hline
SpecificPlotNetSurfaceShortwaveRad MapFile & name of the output file providing the surface Swnet flux map at high temporal resolution during specific days & / & map \\ \hline
SpecificPlotIncomingLongwaveRad MapFile & name of the output file providing the Lwin flux map at high temporal resolution during specific days & / & map \\ \hline
SpecificPlotNetSurfaceLongwaveRad MapFile & name of the output file providing the surface Lwnet map at high temporal resolution during specific days & / & map \\ \hline
HeaderSurfaceEBPoint & column name in the file PointOutputFile for the variable SurfaceEBPoint & PointOutputFile & header \\ \hline
HeaderSoilHeatFluxPoint & column name in the file PointOutputFile for the variable SoilHeatFluxPoint & PointOutputFile & header \\ \hline
HeaderSWinPoint & column name in the file PointOutputFile for the variable SWinPoint & PointOutputFile & header \\ \hline
HeaderSWbeamPoint & column name in the file PointOutputFile for the variable SWbeamPoint & PointOutputFile & header \\ \hline
HeaderSWdiffPoint & column name in the file PointOutputFile for the variable SWdiffPoint & PointOutputFile & header \\ \hline
HeaderLWinPoint & column name in the file PointOutputFile for the variable LWinPoint & PointOutputFile & header \\ \hline
HeaderLWinMinPoint & column name in the file PointOutputFile for the variable LWinMinPoint & PointOutputFile & header \\ \hline
HeaderLWinMaxPoint & column name in the file PointOutputFile for the variable LWinMaxPoint & PointOutputFile & header \\ \hline
HeaderSWNetPoint & column name in the file PointOutputFile for the variable SWNetPoint & PointOutputFile & header \\ \hline
HeaderLWNetPoint & column name in the file PointOutputFile for the variable LWNetPoint & PointOutputFile & header \\ \hline
HeaderHPoint & column name in the file PointOutputFile for the variable HPoint & PointOutputFile & header \\ \hline
HeaderLEPoint & column name in the file PointOutputFile for the variable LEPoint & PointOutputFile & header \\ \hline
HeaderQSurfPoint & column name in the file PointOutputFile for the variable specific humidity near the soil surface & PointOutputFile & header \\ \hline
HeaderQAirPoint & column name in the file PointOutputFile for the variable specific humidity of the air & PointOutputFile & header \\ \hline
HeaderLObukhovPoint & column name in the file PointOutputFile for the variable LObukhovPoint & PointOutputFile & header \\ \hline
HeaderSWupPoint & column name in the file PointOutputFile for the variable SWupPoint & PointOutputFile & header \\ \hline
HeaderLWupPoint & column name in the file PointOutputFile for the variable LWupPoint & PointOutputFile & header \\ \hline
HeaderHupPoint & column name in the file PointOutputFile for the variable HupPoint & PointOutputFile & header \\ \hline
HeaderLEupPoint & column name in the file PointOutputFile for the variable LEupPoint & PointOutputFile & header \\ \hline
\caption{Table of surface energy flux parameters (character)}
\label{surenflux3d_data}
\end{longtable}
\end{center}
\clearpage


\begin{center}
\begin{longtable}{|p {7 cm}|p {7 cm}|p {3 cm}|p {4 cm}|}
\hline
\textbf{Keyword} & \textbf{Description} & \textbf{Associated file} & \textbf{type (file, header)} \\ \hline
\endfirsthead
\hline
\multicolumn{4}{| c |}{continued from previous page} \\
\hline
\textbf{Keyword} & \textbf{Description} & \textbf{Associated file} & \textbf{type (file, header)} \\ \hline
\endhead
\hline
\multicolumn{4}{| c |}{{continued on next page}}\\ 
\hline
\endfoot
\endlastfoot
\hline
CanopyInterceptedWaterMapFile & name of the output file providing the canopy intercepted water map & / & map \\ \hline
SpecificPlotVegSensibleHeatFluxMapFile & name of the output file providing the vegetation sensible heat flux map at high temporal resolution during specific days & / & map \\ \hline
SpecificPlotVegLatentHeatFluxMapFile & name of the output file providing the vegetation latent heat flux map at high temporal resolution during specific days & / & map \\ \hline
SpecificPlotNetVegShortwaveRadMapFile & name of the output file providing the vegetation Swnet flux map at high temporal resolution during specific days & / & map \\ \hline
SpecificPlotNetVegLongwaveRadMapFile & name of the output file providing the vegetation Lwnet map at high temporal resolution during specific days & / & map \\ \hline
SpecificPlotCanopyAirTempMapFile & name of the output file providing the canopy air temperature map at high temporal resolution during specific days & / & map \\ \hline
SpecificPlotVegTempMapFile & name of the output file providing the vegetation temperature map at high temporal resolution during specific days & / & map \\ \hline
SpecificPlotAboveVegAirTempMapFile & name of the output file providing the above vegetation air temperature map at high temporal resolution during specific days & / & map \\ 
HeaderTvegPoint & column name in the file PointOutputFile for the variable TvegPoint & PointOutputFile & header \\ \hline
HeaderTCanopyAirPoint & column name in the file PointOutputFile for the variable TCanopyAirPoint & PointOutputFile & header \\ \hline
HeaderLSAIPoint & column name in the file PointOutputFile for the variable LSAIPoint & PointOutputFile & header \\ \hline
Headerz0vegPoint & column name in the file PointOutputFile for the variable z0vegPoint & PointOutputFile & header \\ \hline
Headerd0vegPoint & column name in the file PointOutputFile for the variable d0vegPoint & PointOutputFile & header \\ \hline
HeaderEstoredCanopyPoint & column name in the file PointOutputFile for the variable EstoredCanopyPoint & PointOutputFile & header \\ \hline
HeaderSWvPoint & column name in the file PointOutputFile for the variable SWvPoint & PointOutputFile & header \\ \hline
HeaderLWvPoint & column name in the file PointOutputFile for the variable LWvPoint & PointOutputFile & header \\ \hline
HeaderHvPoint & column name in the file PointOutputFile for the variable HvPoint & PointOutputFile & header \\ \hline
HeaderLEvPoint & column name in the file PointOutputFile for the variable LEvPoint & PointOutputFile & header \\ \hline
HeaderHgUnvegPoint & column name in the file PointOutputFile for the variable HgUnvegPoint & PointOutputFile & header \\ \hline
HeaderLEgUnvegPoint & column name in the file PointOutputFile for the variable LEgUnvegPoint & PointOutputFile & header \\ \hline
HeaderHgVegPoint & column name in the file PointOutputFile for the variable HgVegPoint & PointOutputFile & header \\ \hline
HeaderLEgVegPoint & column name in the file PointOutputFile for the variable LEgVegPoint & PointOutputFile & header \\ \hline
HeaderEvapSurfacePoint & column name in the file PointOutputFile for the variable EvapSurfacePoint & PointOutputFile & header \\ \hline
HeaderTraspCanopyPoint & column name in the file PointOutputFile for the variable TraspCanopyPoint & PointOutputFile & header \\ \hline
HeaderWaterOnCanopyPoint & column name in the file PointOutputFile for the variable WaterOnCanopyPoint & PointOutputFile & header \\ \hline
HeaderSnowOnCanopyPoint & column name in the file PointOutputFile for the variable SnowOnCanopyPoint & PointOutputFile & header \\ \hline
HeaderQVegPoint & column name in the file PointOutputFile for the variable specific humidity near the vegetation & PointOutputFile & header \\ \hline
HeaderLObukhovCanopyPoint & column name in the file PointOutputFile for the variable LObukhovCanopyPoint & PointOutputFile & header \\ \hline
HeaderWindSpeedTopCanopyPoint & column name in the file PointOutputFile for the variable WindSpeedTopCanopyPoint & PointOutputFile & header \\ \hline
HeaderDecayKCanopyPoint & column name in the file PointOutputFile for the variable DecayKCanopyPoint & PointOutputFile & header \\ \hline
\caption{Table of vegetation parameters (character)}
\label{vegetation3d_data}
\end{longtable}
\end{center}
\clearpage

\section{RECOVERY 3D CHARACTER}

\begin{center}
\begin{longtable}{|p {7 cm}|p {7 cm}|p {3 cm}|p {4 cm}|}
\hline
\textbf{Keyword} & \textbf{Description} & \textbf{Associated file} & \textbf{type (file, header)} \\ \hline
\endfirsthead
\hline
\multicolumn{4}{| c |}{continued from previous page} \\
\hline
\textbf{Keyword} & \textbf{Description} & \textbf{Associated file} & \textbf{type (file, header)} \\ \hline
\endhead
\hline
\multicolumn{4}{| c |}{{continued on next page}}\\ 
\hline
\endfoot
\endlastfoot
\hline
RecoverSoilWatPresChannel & name of the recovery file of SoiWatPresChannel & / & file  \\ \hline
RecoverSoilIceContChannel & name of the recovery file of SoiIceContChannel & / & file  \\ \hline
RecoverSoilTempChannel & name of the recovery file of SoilTempChannel & / & file  \\ \hline
\caption{Table of recovery parameters for channel flow (character)}
\label{recoverychannelflow_data}
\end{longtable}
\end{center}
\clearpage

\begin{center}
\begin{longtable}{|p {7 cm}|p {7 cm}|p {3 cm}|p {4 cm}|}
\hline
\textbf{Keyword} & \textbf{Description} & \textbf{Associated file} & \textbf{type (file, header)} \\ \hline
\endfirsthead
\hline
\multicolumn{4}{| c |}{continued from previous page} \\
\hline
\textbf{Keyword} & \textbf{Description} & \textbf{Associated file} & \textbf{type (file, header)} \\ \hline
\endhead
\multicolumn{4}{| c |}{{continued on next page}}\\ 
\hline
\endfoot
\endlastfoot
\hline
RecoverGlacierLayerThick & name of the recovery file of GlacierLayerThick & / & file  \\ \hline
RecoverGlacierLiqMass & name of the recovery file of GlacieLiqMass & / & file \\ \hline
RecoverGlacierIceMass & name of the recovery file of GlacieIceMass & / & file  \\ \hline
RecoverGlacierTemp & name of the recovery file of GlacieTemp & / & file  \\ \hline
RecoverGlacierLayerNumber & name of the recovery file of GacierLayerNumber & / & file  \\ \hline
\caption{Table of recovery parameters for glacier (character)}
\label{recoverychannelflow_data}
\end{longtable}
\end{center}
\clearpage

\begin{center}
\begin{longtable}{|p {7 cm}|p {7 cm}|p {3 cm}|p {4 cm}|}
\hline
\textbf{Keyword} & \textbf{Description} & \textbf{Associated file} & \textbf{type (file, header)} \\ \hline
\endfirsthead
\hline
\multicolumn{4}{| c |}{continued from previous page} \\
\hline
\textbf{Keyword} & \textbf{Description} & \textbf{Associated file} & \textbf{type (file, header)} \\ \hline
\endhead
\hline
\multicolumn{4}{| c |}{{continued on next page}}\\ 
\hline
\endfoot
\endlastfoot
\hline
RecoverLandSurfaceWaterDepth & name of the recovery file of LandSurfaceWaterDepth & / & file \\ \hline
\caption{Table of recovery parameters for runoff (character)}
\label{recoveryrunoff_data}
\end{longtable}
\end{center}
\clearpage

\begin{center}
\begin{longtable}{|p {7 cm}|p {7 cm}|p {3 cm}|p {4 cm}|}
\hline
\textbf{Keyword} & \textbf{Description} & \textbf{Associated file} & \textbf{type (file, header)} \\ \hline
\endfirsthead
\hline
\multicolumn{4}{| c |}{continued from previous page} \\
\hline
\textbf{Keyword} & \textbf{Description} & \textbf{Associated file} & \textbf{type (file, header)} \\ \hline
\endhead
\hline
\multicolumn{4}{| c |}{{continued on next page}}\\ 
\hline
\endfoot
\endlastfoot
\hline
RecoverSnowLiqMass & name of the recovery file of SnowLiqMass & / & file \\ \hline
RecoverSnowIceMass & name of the recovery file of SnowIceMass & / & file \\ \hline
RecoverSnowTemp & name of the recovery file of SnowTemp & / & file \\ \hline
RecoverSnowLayerNumber & name of the recovery file of SnowLayerNumber & / & file \\ \hline
RecoverNonDimensionalSnowAge & name of the recovery file of NonDimensionalSnowAge & / & file \\ \hline
RecoverDimensionalSnowAge & name of the recovery file of DimensionalSnowAge & / & file \\ \hline
\caption{Table of recovery parameters for snow (character)}
\label{recoverysnow_data}
\end{longtable}
\end{center}
\clearpage

\begin{center}
\begin{longtable}{|p {7 cm}|p {7 cm}|p {3 cm}|p {4 cm}|}
\hline
\textbf{Keyword} & \textbf{Description} & \textbf{Associated file} & \textbf{type (file, header)} \\ \hline
\endfirsthead
\hline
\multicolumn{4}{| c |}{continued from previous page} \\
\hline
\textbf{Keyword} & \textbf{Description} & \textbf{Associated file} & \textbf{type (file, header)} \\ \hline
\endhead
\hline
\multicolumn{4}{| c |}{{continued on next page}}\\ 
\hline
\endfoot
\endlastfoot
\hline
RecoverSoilWatPres & name of the recovery file of SoilWatPres & / & file \\ \hline
RecoverSoilIceCont & name of the recovery file of SoilIceCont & / & file \\ \hline
RecoverSoilTemp & name of the recovery file of SoilTemp & / & file \\ \hline
\caption{Table of recovery parameters for soil (character)}
\label{recoverysoil_data}
\end{longtable}
\end{center}
\clearpage


\begin{center}
\begin{longtable}{|p {7 cm}|p {7 cm}|p {3 cm}|p {4 cm}|}
\hline
\textbf{Keyword} & \textbf{Description} & \textbf{Associated file} & \textbf{type (file, header)} \\ \hline
\endfirsthead
\hline
\multicolumn{4}{| c |}{continued from previous page} \\
\hline
\textbf{Keyword} & \textbf{Description} & \textbf{Associated file} & \textbf{type (file, header)} \\ \hline
\endhead
\hline
\multicolumn{4}{| c |}{{continued on next page}}\\ 
\hline
\endfoot
\endlastfoot
\hline
RecoverLiqWaterOnCanopy & name of the recovery file of LiqWaterOnCanopy & / & file \\ \hline
RecoverSnowOnCanopy & name of the recovery file of SnowOnCanopy & / & file \\ \hline
RecoverVegTemp & name of the recovery file of VegetationTemperature & / & file \\ \hline
\caption{Table of recovery parameters for vegetation (character)}
\label{recoveryvegetation_data}
\end{longtable}
\end{center}
\clearpage


\end{document}  