%%%%%%%%%%%%%%%%%%%%%%%%%%%%%%%%%%%%%%%%%%%%%%%%%%%%%%%%%%%%%%%
\chapter{Compiling Instructions}\label{chap:User Manual}
%%%%%%%%%%%%%%%%%%%%%%%%%%%%%%%%%%%%%%%%%%%%%%%%%%%%%%%%%%%%%%%

\noindent Before you begin to use GEOtop, is advisable you subscribe to the
users mailing list. By doing it you will receive informations by developers and
other users, and you will be able to post your question to the community as
well. Follow the instruction at the following link:\\

\textcolor{blue}{\underline{{https://groups.google.com/forum/?pli=1\#!forum/geotopusers}}}\\


\noindent GEOtop runs properly under:
\begin{itemize}
 \item Linux platform;
 \item Mac platform;
 \item Windows platform.
\end{itemize}


If you want to build GEOtop from sources in your own machine (Linux and MacOSX):

see here: 

\textcolor{blue}{\underline{{https://github.com/geotopmodel/geotop/blob/master/doc/Install.rst}}}\\

If you prefer to install GEOtop via Docker to avoid manual installation of packages:

see here: 

\textcolor{blue}{\underline{{https://hub.docker.com/r/omslab/geotop}}}\\


%%%%%%%%%%%%%%%%%%%%%%%%%%%%%%%%%%%%%%%%%%%%%%%%%%%%%%%%%%%%%%%
%\section{Compile GEOtop through a makefile}
%%%%%%%%%%%%%%%%%%%%%%%%%%%%%%%%%%%%%%%%%%%%%%%%%%%%%%%%%%%%%%%
%The GEOtop source code can be downloaded through a terminal (or command prompt if you are using Windows) by typing, 
%as shown in \textsl{Figure \ref{cmp}}:\\

%\textsl{"svn co https://dev.fsc.bz.it/repos/geotop/trunk/0.9375KMacKenzie"}\\
%\begin{figure}[!h]
%\begin{center}
%  \begin{minipage}[c]{.80\textwidth}
%    \includegraphics[width=1\textwidth]{./images/pic_compile/compile.png}
%    \textsl{\caption{Download GEOtop source code through a terminal} \label{cmp}}
%  \end{minipage}
%\end{center}
%\end{figure}


%\noindent The downloaded folder contains the folders:
%\begin{itemize}
% \item Debug: which contains the object file created during the compilation and the makefile
% \item geotop: which contains the code
% \item Libraries: which contains the support libraries
%\end{itemize}



%\footnotesize{
%\begin{verbatim}
%HM	= .
%LHM		= $(HM)
%BINPATH 	= $(LHM)/
%NAME		= geotop0.9375
%BINS		= $(BINPATH)/$(NAME)

%SCRSPATH1	= $(HM)/KMacKenzie
%LIBPATH1	= $(HM)/LIBRARIES/FLUIDTURTLES
%LIBPATH2	= $(HM)/LIBRARIES/ASCII
%LIBPATH3	= $(HM)/LIBRARIES/GEOMORPHOLOGYLIB
%LIBPATH4	= $(HM)/LIBRARIES/MATH2
%LIBPATH6	= $(HM)/LIBRARIES/KeyPalette
%LIBPATH7	= $(HM)/EXTERN

%OBJ		= 	$(SCRSPATH1)/energy.balance.o $(SCRSPATH1)/frost_table.o\
%		  	$(SCRSPATH1)/geotop.09375.o $(SCRSPATH1)/recovery.o\
%		  	$(SCRSPATH1)/input.09375.o $(SCRSPATH1)/meteo.09375.o\
%		  	$(SCRSPATH1)/output.09375.o $(SCRSPATH1)/pedo.funct.o\
%			$(SCRSPATH1)/radiation.o\
%			$(SCRSPATH1)/snow.09375.o $(SCRSPATH1)/times.o\
%			$(SCRSPATH1)/water.balance_1D.o $(SCRSPATH1)/turbulence.o\
%			$(SCRSPATH1)/water.balance_3D.o $(SCRSPATH1)/vegetation.o\
%			$(LIBPATH1)/alloc.o $(LIBPATH1)/error.o\
%			$(LIBPATH1)/list.o $(LIBPATH1)/t_io.o\
%			$(LIBPATH1)/tensors3D.o  $(LIBPATH1)/utilities.o\
%			$(LIBPATH1)/datamanipulation.o $(LIBPATH1)/random.o\
%			$(LIBPATH1)/linearalgebra.o $(LIBPATH1)/write_dem.o\
%			$(LIBPATH2)/import_ascii.o $(LIBPATH2)/rw_maps.o\
%			$(LIBPATH2)/write_ascii.o $(LIBPATH2)/tabs.o\
%			$(LIBPATH3)/networks.o $(LIBPATH3)/geomorphology.0875.o\
%			$(LIBPATH3)/shadows.o  $(LIBPATH3)/dtm_resolution.o\
%			$(LIBPATH4)/geo_statistic.09375.o $(LIBPATH4)/sparse_matrix.o\
%			$(LIBPATH4)/util_math.o\
%			$(LIBPATH6)/key.palette.o $(LIBPATH6)/get_filenames.o\
%			$(LIBPATH6)/additional_read_functions.o $(LIBPATH6)/read_command_line.o\
%			$(LIBPATH7)/PBSM.o $(LIBPATH7)/micromet.o

%HPATH1 	= $(LIBPATH1)
%HPATH2	= $(LIBPATH2)
%HPATH3  = $(LIBPATH3)
%HPATH4  = $(LIBPATH4)
%HPATH6  = $(LIBPATH6)
%HPATH7  = $(LIBPATH7)
%HPATH0  = $(SCRSPATH1)

%CFLAGS	= -O3 -g 
%INCLUDE = -I$(HPATH1) -I$(HPATH2) -I$(HPATH3) -I$(HPATH4)  -I$(HPATH6) -I$(HPATH7) -I$(HPATH0)

%DEBUG   = -g -Wall
%CC	= gcc $(DEBUG)
%CPP	= g++

%.cc.o: $*.cc $*.h
%	$(CPP) $(CPPFLAGS) -c $< $(INCLUDE) -o $@

%.c.o: $*.c $*.h
%	$(CC) $(CFLAGS) -c $< $(INCLUDE) -o $@

%
%all: geotop

%geotop: $(OBJ)
%	$(CC) -o $(BINS) $(OBJ) -lm -rdynamic -ldl -lstdc++ 
%clean:
%	rm -rf *.o *~ $(OBJ)
%\end{verbatim}
%}
%
%\noindent Open a terminal, go into the folder \textsl{Debug} by typing:
%
%\footnotesize{
%\begin{verbatim}
%$ cd Debug
%\end{verbatim}
%}


%By typing \textsl{"ls"} you should have the following files and folders \textsl{Figure \ref{fo}}:
%\begin{figure}[!h]
%\begin{center}
%  \begin{minipage}[c]{.80\textwidth}
%    \includegraphics[width=1\textwidth]{./images/pic_compile/folders.png}
%    \textsl{\caption{Necessary files and folders to compile GEOtop through terminal} \label{fo}}
%  \end{minipage}
%\end{center}
%\end{figure}
%
%\noindent To compile GEOtop, type:
%
%\footnotesize{
%\begin{verbatim}
%$ make all
%\end{verbatim}
%}
%
%\noindent The executable file {\it GEOtop1.2} is now created in the {\it Debug} folder.

%\textsl{Figure \ref{fo_c}}.\\
%\begin{figure}[!h]
%\begin{center}
%  \begin{minipage}[c]{.80\textwidth}
%    \includegraphics[width=1\textwidth]{./images/pic_compile/folders_compiled.png}
%    \textsl{\caption{Screenshot of the 0.9375KMackenzie folder with the executable} \label{fo_c}}
%  \end{minipage}
%\end{center}
%\end{figure}


%%%%%%%%%%%%%%%%%%%%%%%%%%%%%%%%%%%%%%%%%%%%%%%%%%%%%%%%%%%%%%%%
%\section{Compile and browse GEOtop through Eclipse}
%%%%%%%%%%%%%%%%%%%%%%%%%%%%%%%%%%%%%%%%%%%%%%%%%%%%%%%%%%%%%%%%
%Eclipse is a platform to build integrated development environments (IDEs).
%You first need to download and install it to succesfully compile 
%and browse GEOtop.\\

%\noindent For a complete tutorial download the presentation from:\\

%\textcolor{blue}{\underline{\textsl{http://www.geotop.org/cgi-bin/moin.cgi/Compile\_Instructions}{secondo}}}\\

%\noindent Download and extract the appropriate version for your operative
%system from the following link:\\

%\textcolor{blue}{\underline{\textsl{http://www.eclipse.org/downloads/}{terzo}}}

%
%%=============================================================%
%\subsection{Linux}
%%=============================================================%
%Using Linux Ubunt Karmic Koala 9.10, the Eclipse Galileo release has a bug with the Graphical Interface.
%To fix it follow this instrusctions:\\
%Supposing that the executable file is \textsl{"home\/ilaria\/Scrivania\/eclipse\_folder"},
%create a text file with the following string:\\
%\begin{center}
%  \begin{minipage}[c]{.4\textwidth}
%    \centering
%    \includegraphics[width=1\textwidth]{./images/pic_compile/0_bug_fix.png}
%  \end{minipage}
%\end{center}
%Save it with a name that you like, but with the extension \textsl{".sh"} in the
%folder where the eclipse executable is, in this example \textsl{"home\/ilaria\/Scrivania\/eclipse\_folder"}.\\
%Right click on the file and allow it to be executable.\\
%Now lunch the .sh file instead of launching the eclipse icon and the bug is
%fixed\\

%%%%%%%%%%%%%%%%%%%%%%%%%%%%%%%%%%%%%%%%%%%%%%%%%%%%%%%%%%%%%%%%
%\subsection{CDT and SVN packages}
%%%%%%%%%%%%%%%%%%%%%%%%%%%%%%%%%%%%%%%%%%%%%%%%%%%%%%%%%%%%%%%%
%You now need to install two packages to succesfully compile GEOtop:
%\begin{itemize}
% \item CDT packages for eclipse Galileo: C/C++ Development Tooling
% \item SVN packages for eclipse Galileo
%\end{itemize}

%\paragraph{CDT packages:}
%\noindent To install the CDT packages go on: \textsl{"Help $\rightarrow$ Install New Software"}
%and in \textsl{"Work with"} add the repository \textsl{"http://download.eclipse.org/releases/galileo"}
%as shown in \textsl{Figure \ref{f:1}}\\
%\begin{figure}[!h]
%\begin{center}
%  \begin{minipage}[c]{.8\textwidth}
%    \centering
%    \includegraphics[width=1\textwidth]{./images/pic_compile/2_0_1_update.png}
%    \textsl{\caption{CDT} \label{f:1}}
%  \end{minipage}
%\end{center}
%\end{figure}

%\noindent Click on \textsl{"Next"}. Figure \textsl{\ref{f:2}} shows the packages you 
%are about to install and the license agreement, that you must accept.

%\begin{figure}[!h]
%\begin{center}
%  \begin{minipage}[c]{.45\textwidth}
%    \centering
%    \includegraphics[width=1\textwidth]{./images/pic_compile/2_1_update.png}
%  \end{minipage}
%  \begin{minipage}[c]{.45\textwidth}
%    \centering
%    \includegraphics[width=1\textwidth]{./images/pic_compile/2_2_update.png}
%  \end{minipage}
%\end{center}
%    \textsl{\caption{CDT packages and License agreement} \label{f:2}}
%\end{figure}

%%%%%%%%%%%%%%%%%%%%%%%%%%%%%%%%%%%%%%%%%%%%%%%%%%%
%%%%%%%%%%%%%%%%%%%%%%%%%%%%%%%%%%%%%%%%%%%%%%%%%%%
%%%%%%%%%%%%%%%%%%%%%%%%%%%%%%%%%%%%%%%%%%%%%%%%%%%
%\newpage
%\paragraph{SVN packages:}
%\noindent To install the SVN packages go on:
%\begin{itemize}
% \item \textsl{"Help $\rightarrow$ Install New Software"}
%and in \textsl{"Work with"} add the repository \textsl{"http://download.eclipse.org/releases/galileo"}
%check the box \textsl{"Collaboration"} $\rightarrow$ \textsl{"Subversive SVN Team Provider (Incubator)"}
%as shown in \textsl{Figure\ref{f:3}}
% \item \noindent\textsl{"Help $\rightarrow$ Install New Software"}
%and in \textsl{"Work with"} add the repository \textsl{"http://www.polarion.org/projects/subversive/download/eclipse/2.0/update-site"}
%and then choose \textsl{"Subversive SVN Connectors"} $\rightarrow$ \textsl{"SVNKit Implementation (Optional)"}
%as shown in \textsl{Figure \ref{f:3_1}}
%\end{itemize}
% 

%\begin{figure}[!h]
%\begin{center}
%  \begin{minipage}[c]{.45\textwidth}
%    \centering
%    \includegraphics[width=1\textwidth]{./images/pic_compile/3_svn_1.png}
%  \end{minipage}
%%  \begin{minipage}[c]{.45\textwidth}
%%    \centering
%%    \includegraphics[width=1\textwidth]{./images/pic_compile/3_svn_2.png}
%%  \end{minipage}
%  \end{center}
%    \textsl{\caption{SVN} \label{f:3}}
%\end{figure}

%
%\begin{figure}[!h]
%\begin{center}
%  \begin{minipage}[c]{.45\textwidth}
%    \centering
%    \includegraphics[width=1\textwidth]{./images/pic_compile/3_svn_3.png}
%  \end{minipage}
%  \begin{minipage}[c]{.45\textwidth}
%    \centering
%    \includegraphics[width=1\textwidth]{./images/pic_compile/3_svn_4.png}
%  \end{minipage}
%  \end{center}
%    \textsl{\caption{SVN} \label{f:3_1}}
%\end{figure}

%

%\clearpage
%%%%%%%%%%%%%%%%%%%%%%%%%%%%%%%%%%%%%%%%%%%%%%%%%%%%%%%%%%%%%%%%
%\subsubsection{Download GEOtop code from SVN}
%%%%%%%%%%%%%%%%%%%%%%%%%%%%%%%%%%%%%%%%%%%%%%%%%%%%%%%%%%%%%%%%
%From \textsl{Window} $\rightarrow$ \textsl{Open prospective} $\rightarrow$ \textsl{Other} $\rightarrow$ \textsl{SVN Repository Exploring} $\rightarrow$ \textsl{OK}\\
%From \textsl{File} $\rightarrow$ \textsl{Import} $\rightarrow$ \textsl{SVN} $\rightarrow$ \textsl{Project from SVN}\\
%Add the Url: \textsl{"https://dev.fsc.bz.it/repos/geotop"} $\rightarrow$ \textsl{"Browse"}\\  
%From \textsl{geotop $\rightarrow$ trunk} $\rightarrow$ select \textsl{"0.9375KMackenzie \#\#"} where the \#\# is the current GEOtop version\\ 
%And then check \textsl{Save in the workspace location path} as shown in \textsl{Figure \ref{f:4}}

%\begin{figure}[!h]
%\begin{center}
%  \begin{minipage}[c]{.4\textwidth}
%    \centering
%    \includegraphics[width=1\textwidth]{./images/pic_compile/3_svn_6.png}
%  \end{minipage}
%  \begin{minipage}[c]{.4\textwidth}
%    \centering
%    \includegraphics[width=1\textwidth]{./images/pic_compile/3_svn_7.png}
%  \end{minipage}
%\end{center}
%    \textsl{\caption{SVN} \label{f:4}}
%\end{figure}

%\newpage
%\paragraph{Set the C/C++ prospective:}
%%\noindent Set the GNU compiler and math library\\
%From \textsl{Window} $\rightarrow$ \textsl{Open prospective} $\rightarrow$ \textsl{Other} $\rightarrow$ \textsl{C/C++}\\
%The \textsl{'GEOtopKMacKenzie\_SVN'} project will be in the C/C++ prospective \textsl{Figure \ref{f:5}}\\

%\noindent \underline{Linux}\\
%\noindent Right click on \textsl{GEOtopKMackenzie\_SVN folder}
% $\rightarrow$ \textsl{Proprieties}  $\rightarrow$ \textsl{C/C++ Build}  $\rightarrow$ \textsl{Tool Chain Editor}\\
% $\rightarrow$ \textsl{Current Tool chain}: Linux GCC\\
% $\rightarrow$ \textsl{Current builder}: CDT Internal Builder\\

%\noindent Right click on GEOtopKMackenzie\_SVN folder\\
% $\rightarrow$ Proprieties  $\rightarrow$ C/C++ Build  $\rightarrow$ Settings
% $\rightarrow$ GCC C Linker $\rightarrow$ Miscellaneous
% $\rightarrow$ Linker flag: type \textsl{"-lm"} to activate math library \textsl{Figure \ref{f:6}}\\

%
%\begin{figure}[!h]
%\begin{center}
%  \begin{minipage}[c]{.8\textwidth}
%    \centering
%    \includegraphics[width=1\textwidth]{./images/pic_compile/7_compile.png}
%  \end{minipage}
%\end{center}
%    \textsl{\caption{SVN} \label{f:5}}
%\end{figure}

%\begin{figure}[!h]
%\begin{center}
%  \begin{minipage}[c]{.4\textwidth}
%    \centering
%    \includegraphics[width=1\textwidth]{./images/pic_compile/Tool_Chain.png}
%  \end{minipage}
%  \begin{minipage}[c]{.4\textwidth}
%    \centering
%    \includegraphics[width=1\textwidth]{./images/pic_compile/Linker_flag.png}
%  \end{minipage}
%\end{center}
%    \textsl{\caption{Tool Chain Editor and Linker Flag} \label{f:6}}
%\end{figure}

%\newpage
%\noindent \underline{MacOSx}\\
%\noindent Right click on \textsl{GEOtopKMackenzie\_SVN folder}
% $\rightarrow$ \textsl{Proprieties}  $\rightarrow$ \textsl{C/C++ Build}  $\rightarrow$ \textsl{Tool Chain Editor}\\
% $\rightarrow$ \textsl{Current Tool chain}: MacOSx GCC\\
% $\rightarrow$ \textsl{Current builder}: GNU Make Builder \textsl{Figure \ref{f:7}}

%\begin{figure}[!h]
%\begin{center}
%  \begin{minipage}[c]{.8\textwidth}
%    \centering
%    \includegraphics[width=1\textwidth]{./images/pic_compile/screen_Mac.png}
%  \end{minipage}
%\end{center}
%    \textsl{\caption{SVN} \label{f:7}}
%\end{figure}

%\noindent Now is possible to successfully build GEOtop\\
%Click on the hammer symbol: \textsl{Build 'Debug' for project 'GEOtopKMacKenzie\_SVN2'}\\
%GEOtop executable has been built $\rightarrow$ The executable file is under the folder 'Binaries'\\
%\noindent The simulation finished succesfully  \textsl{Figure \ref{f:8}}.

%\begin{figure}[!h]
%\begin{center}
%  \begin{minipage}[c]{.6\textwidth}
%    \centering
%    \includegraphics[width=1\textwidth]{./images/pic_compile/12_end.png}
%  \end{minipage}
%\end{center}
%    \textsl{\caption{SVN} \label{f:8}}
%\end{figure}

%

%\newpage
%%%%%%%%%%%%%%%%%%%%%%%%%%%%%%%%%%%%%%%%%%%%%%%%%%%%%%%%%%%%%%%
