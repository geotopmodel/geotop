\chapter{General features}

\section{Input}

\subsection{File}
\begin{center}
\begin{longtable}{|p {4.2 cm}|p {7 cm}|p {2 cm}|p {1.5 cm}|}
\hline
\textbf{Keyword} & \textbf{Description}  \\ \hline
\endfirsthead
\hline
\multicolumn{4}{| c |}{continued from previous page} \\
\hline
\textbf{Keyword} & \textbf{Description}  \\ \hline
\endhead
\hline
\multicolumn{4}{| c |}{{continued on next page}}\\ 
\hline
\endfoot
\endlastfoot
\hline
TimeStepsFile \index{TimeStepsFile} & name of the file providing the integration time steps \\ \hline
\caption{Keyword of file related to general input}
\label{gen_file}
\end{longtable}
\end{center}


\subsection{Parameters}

\begin{center}
\begin{longtable}{|p {3.8 cm}|p {4.2 cm}|p {1 cm}|p{1.3 cm}|p{1. cm}|p{0.8 cm}|p{0.8 cm}|}
\hline
\textbf{Keyword} & \textbf{Description} & \textbf{M. U.} & \textbf{range} & \textbf{Default Value} & \textbf{Sca / Vec} & \textbf{Log / Num} \\ \hline
\endfirsthead
\hline
\multicolumn{7}{| c |}{continued from previous page} \\
\hline
\textbf{Keyword} & \textbf{Description} & \textbf{M. U.} & \textbf{range} & \textbf{Default Value} & \textbf{Sca / Vec} & \textbf{Log / Num} \\ \hline
\endhead
\hline
\multicolumn{7}{| c |}{{continued on next page}}\\ 
\hline
\endfoot
\endlastfoot
\hline
FlagSkyViewFactor \index{FlagSkyViewFactor} & If not present, the sky view factor can be calculated (=1), or just be considered only equal to 1 (=0) & - & 0, 1 & 0 & sca & opt \\ \hline
TimeStepEnergyAndWater \index{TimeStepEnergyAndWater} & Integrations time step [s] for energy and water balance equation (mandatory) & s & 0, inf & NA & vec & num \\ \hline
InitDateDDMMYYYYhhmm \index{InitDateDDMMYYYYhhmm} & Date and time of the simulation start in date12 format (mandatory) & format DDMMYYhhmm & 01/01/1800 00:00, 01/01/2500 00:00 & NA & vec & str \\ \hline
EndDateDDMMYYYYhhmm \index{EndDateDDMMYYYYhhmm} & Date and time of the simulation start in date12 format (mandatory) & format DDMMYYhhmm & 01/01/1800 00:00, 01/01/2500 00:00 & NA & vec & str \\ \hline
NumSimulationTimes \index{NumSimulationTimes} & How many times the simulation is run (if $>$1, it uses the final condition as initial conditions of the new simulation) & - & 0, inf & 1 & vec & num \\ \hline
StandardTimeSimulation \index{StandardTimeSimulation} & Standard time to which all the output data are referred (difference respect UMT, in hours): GMT + x [h] & h & 0, 12 & 0 & sca & num \\ \hline
PointSim \index{PointSim} & Point simulation (=1), distributed simulation (=0) & - & 0, 1 & 0 & sca & opt \\ \hline
RecoverSim \index{RecoverSim} & Simulation recovered (n=number of saving point you want to start from), otherwise (=0) & - & 0, max saving points & 0 & sca & opt \\ \hline
WaterBalance \index{WaterBalance} & Activate water balance (Yes=1, No=0) & - &  & 0 & sca & opt \\ \hline
EnergyBalance \index{EnergyBalance} & Activate energy balance (Yes=1, No=0) &  &  & 0 & sca & opt \\ \hline
PixelCoordinates \index{PixelCoordinates} & Write 1 IF ALL point coordinates are in format (East, North) in meter, or if in format row and colums (r,c) of the dem map & - &  & 1 & sca & opt \\ \hline
SavingPoints \index{SavingPoints} &  & - &max saving points  & NA & vec & num \\ \hline
SoilLayerTypes \index{SoilLayerTypes} & Number of types of soil types, corresponding to different soil stratigraphies & - &  & 1 & sca & num \\ \hline
DefaultSoilTypeLand \index{DefaultSoilTypeLand} & given a multiple number of type of soil, this relates to the default given to the land type & - &  & 1 & sca & num \\ \hline
DefaultSoilTypeChannel \index{DefaultSoilTypeChannel} & given a multiple number of type of soil, this relates to the default given to the channel type & - &  & 1 & sca & num \\ \hline
\caption{Keywords for the general parameters settable in geotop.inpts}
\label{general_numeric}
\end{longtable}
\end{center}


\section{Output}

\subsection{Maps parameters}

\begin{center}
\begin{longtable}{|p {3.4 cm}|p {4.5 cm}|p {1 cm}|p{1.3 cm}|p{1. cm}|p{0.8 cm}|p{0.8 cm}|}
\hline
\textbf{Keyword} & \textbf{Description} & \textbf{M. U.} & \textbf{range} & \textbf{Default Value} & \textbf{Sca / Vec} & \textbf{Log / Num} \\ \hline
\endfirsthead
\hline
\multicolumn{7}{| c |}{continued from previous page} \\
\hline
\textbf{Keyword} & \textbf{Description} & \textbf{M. U.} & \textbf{range} & \textbf{Default Value} & \textbf{Sca / Vec} & \textbf{Log / Num} \\ \hline
\endhead
\hline
\multicolumn{7}{| c |}{{continued on next page}}\\ 
\hline
\endfoot
\endlastfoot
\hline
FormatOutputMaps \index{FormatOutputMaps}& Format of the output maps (=2 grass ascii, =3 esri ascii) & - & 2, 3 & 3 & sca & opt \\ \hline
\caption{Keywords of general parameters regarding output options that may be set in geotop.inpts}
\label{gen_out_par}
\end{longtable}
\end{center}