\chapter{Discharge at the outlet}


%%%%%%%%%%%%%%%%%%%%%%%%%%%%%%%%%%%%%%%%%%%%%%%%%%%%%%%%%%%%%%%
\section{Input}
%%%%%%%%%%%%%%%%%%%%%%%%%%%%%%%%%%%%%%%%%%%%%%%%%%%%%%%%%%%%%%%

\begin{center}
\begin{longtable}{|p {3.9 cm}|p {5.1 cm}|p {1.1 cm}|p{1 cm}|p{1. cm}|p{0.8 cm}|p{1 cm}|}
\hline
\textbf{Keyword} & \textbf{Description} & \textbf{M. U.} & \textbf{range} & \textbf{Default Value} & \textbf{Sca / Vec} & \textbf{Log / Num} \\ \hline
\endfirsthead
\hline
\multicolumn{7}{| c |}{continued from previous page} \\
\hline
\textbf{Keyword} & \textbf{Description} & \textbf{M. U.} & \textbf{range} & \textbf{Default Value} & \textbf{Scalar / Vector} & \textbf{Logical / Numeric} \\ \hline
\endhead
\hline
\multicolumn{7}{| c |}{{continued on next page}}\\ 
\hline
\endfoot
\endlastfoot
\hline
SurFlowResLand \index{SurFlowResLand} & ($C_m$): coefficient of of the law of uniform motion on the surface $(v_{sup}=C_m \cdot h_{sup}^{\gamma} \cdot i_{DD}^{0.5})$, $\gamma$ defined below & m$^{1-\gamma}$ s$^{-1}$ & 0.01, 5.0 & 0.5 & sca & num \\ \hline
SurFlowResExp \index{SurFlowResExp} & ($\gamma$): Exponent of the law of uniform motion on the surface $v=C_m \cdot h_{sup}^{\gamma} \cdot  i^{0.5}$ & - & 0.25 - 0.34  & 0.67 & sca & num \\ \hline
ThresWaterDepthLandDown \index{ThresWaterDepthLandDown} & $h_{sup}$: Threshold below which $C_m$ is 0 (water does not flow on the surface)  & mm &  & 0 & sca & num \\ \hline
ThresWaterDepthLandUp \index{ThresWaterDepthLandUp} & $h_{sup}$: Threshold above which $C_m$ is independent from $h_{sup}$ (= fully developed turbulence) & mm &  & 50 & sca & num \\ \hline
SurFlowResChannel \index{SurFlowResChannel} & Resistance coefficient for the channel flow (the same $\gamma$ for land surface flow is used) & m$^{1-\gamma}$ s$^{-1}$ &  & 20 & sca & num \\ \hline
ThresWaterDepthChannelUp \index{ThresWaterDepthChannelUp} & $h_{sup}$ Threshold above which C$_m$ is independent from $h_{sup}$ (= fully developed turbulence).  & mm &  & 50 & sca & num \\ \hline
RatioChannelWidthPixelWidth \index{RatioChannelWidthPixelWidth} & Fraction of channel width in the pixel width & - &  & 0.1 & sca & num \\ \hline
ChannelDepression \index{ChannelDepression} & Depression of the channel bed with respect to the neighboring slopes. It is used to change between free and submerged weir flow model to represent to surface flow to the channel & mm &  & 500 & sca & num \\ \hline
MinSupWaterDepthLand \index{MinSupWaterDepthLand} & minimum surface water depth on the earth below which the Courant condition is not applied & mm &  & 1 & sca & num \\ \hline
MinSupWaterDepthChannel \index{MinSupWaterDepthChannel} & minimum surface water depth on the channel below which the Courant condition is not applied & mm &  & 1 & sca & num \\ \hline
\caption{Keywords on input parameters to describe surface water flow on land and channel}
\label{channel_flow_numeric}
\end{longtable}
\end{center}

\begin{center}
\begin{longtable}{|p {3.5 cm}|p {5 cm}|p {1 cm}|p{1 cm}|p{1.1 cm}|p{1.cm}|p{1 cm}|}
\hline
\textbf{Keyword} & \textbf{Description} & \textbf{M. U.} & \textbf{range} & \textbf{Default Value} & \textbf{Scalar / Vector} & \textbf{Logical / Numeric} \\ \hline
\endfirsthead
\hline
\multicolumn{7}{| c |}{continued from previous page} \\
\hline
\textbf{Keyword} & \textbf{Description} & \textbf{M. U.} & \textbf{range} & \textbf{Default Value} & \textbf{Sca / Vec} & \textbf{Log / Num} \\ \hline
\endhead
\hline
\multicolumn{7}{| c |}{{continued on next page}}\\ 
\hline
\endfoot
\endlastfoot
\hline
MinTimeStepSupFlow \index{MinTimeStepSupFlow} & minimum integration time step for the integration (surface flow equation)  &  &  & 0.01 & sca & num \\ \hline
\caption{Keywords of input numeric parameters for the surface water balance equation settable in geotop.inpts}
\label{numeric1d_num_supflow}
\end{longtable}
\end{center}


%%%%%%%%%%%%%%%%%%%%%%%%%%%%%%%%%%%%%%%%%%%%%%%%%%%%%%%%%%%%%%%
\section{Output}
%%%%%%%%%%%%%%%%%%%%%%%%%%%%%%%%%%%%%%%%%%%%%%%%%%%%%%%%%%%%%%%

\subsection{Point}

\subsubsection{Files}

\begin{center}
\begin{longtable}{|p {3.3 cm}|p {9 cm}|}
\hline
\textbf{Keyword} & \textbf{Description}  \\ \hline
\endfirsthead
\hline
\multicolumn{2}{| c |}{continued from previous page} \\
\hline
\textbf{Keyword} & \textbf{Description}   \\ \hline
\endhead
\hline
\multicolumn{2}{| c |}{{continued on next page}}\\ 
\hline
\endfoot
\endlastfoot
\hline
DischargeFile \index{DtPlotDischarge} & name of the file providing the discharge values at the outlet \\ \hline
\caption{Keywords of file related to point output variables}
\label{veget_file}
\end{longtable}
\end{center}

\subsubsection{Parameters}

\begin{center}
\begin{longtable}{|p {3.2 cm}|p {4.8 cm}|p {1 cm}|p{1.4 cm}|p{1.5 cm}|p{1. cm}|p{1 cm}|}
\hline
\textbf{Keyword} & \textbf{Description} & \textbf{M. U.} & \textbf{range} & \textbf{Default Value} & \textbf{Sca / Vec} & \textbf{Log / Num} \\ \hline
\endfirsthead
\hline
\multicolumn{7}{| c |}{continued from previous page} \\
\hline
\textbf{Keyword} & \textbf{Description} & \textbf{M. U.} & \textbf{range} & \textbf{Default Value} & \textbf{Sca / Vec} & \textbf{Log / Num} \\ \hline
\endhead
\hline
\multicolumn{7}{| c |}{{continued on next page}}\\ 
\hline
\endfoot
\endlastfoot
\hline
DtPlotDischarge \index{DtPlotDischarge} \index{DtPlotDischarge} & Plotting Time step (in hour) of the water discharge (0 means the it is not plotted) & h & 0, inf & 0 & vec & num \\ \hline 
\caption{Keywords defining which parameter to print on the DischargeFile}
\label{point1d_numeric}
\end{longtable}
\end{center}


